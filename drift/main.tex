% !TeX spellcheck = da_DK
\documentclass{article}
% Ability to write input files using utf8
\usepackage[utf8]{inputenc}

% Proper font, æ, ø and å becomes copy/paste and searchable
\usepackage[T1]{fontenc}
\usepackage{lmodern}

% Enable \includegraphics, so that images can be included
\usepackage{graphicx}
\DeclareGraphicsExtensions{.pdf, .png, .jpg, .PDF, .PNG, .JPG}

% Enables use of links, and adds ToC for your PDF-reader
\usepackage{hyperref}
% A small macro for inserting clickable email adresses, e.g.,
% \email{best@fredagscafeen.dk}
\newcommand*{\email}[1]{\href{mailto:#1}{\nolinkurl{#1}}}

% Adding visible TODO's using \todo
\usepackage[obeyFinal]{todonotes}

% Better kerning etc.
\usepackage{microtype}

% To create and control lists (itemize, enumeration, description)
\usepackage{enumitem}

% A new list environment that has both numbers (as enumeration) and
% labels (as description)
\newcounter{enumdescriptioncount}
\newlist{enumdescription}{description}{1}        % 1 means that it cannot be nested
\setlist[enumdescription]{%
  before = {\setcounter{enumdescriptioncount}{0}%
            \renewcommand*{\theenumdescriptioncount}{\arabic{enumdescriptioncount}}},
  font = {\bfseries\stepcounter{enumdescriptioncount}{\large \theenumdescriptioncount.}~},
  align = right,                % Makes the labels end, at the same position
  itemindent = 6em              % TODO: This can be done better
}

% Hyphenation for Danish words etc.
\usepackage[danish]{babel}

% Macros for common words, so that we can format them fancily. xspace
% inserts spaces when needed based on context.
\usepackage{xspace}

\newcommand*{\fotex}{F%
\kern -.25em%
{\raisebox{-.215em}{Ø}}%  % -.215em makes 2 Es match
\kern -.25em%
\TeX\xspace}


\title{Drift}
\author{Silas Bergishagen Christiansen}
\date{\formatdate{16}{2}{2026}}

\begin{document}

\maketitle

\section{Introduktion}

Som Drift i Fredagscaféen har man ansvaret for vedligehold af baren og især fadølsanlægget. Det indeholder også at tage fat i bygningsdrift i tilfælde af AU's ejendom, så som stole, borde, toiletter, skulle gå i stykker eller på anden måde tage skade under en barvagt.

Få styr på hvor forskellige ting kan findes, og hvad der kunne opstå af mangler. F.eks. kan værktøj findes i \textit{træburet}, og reservedele til fadølsanlægget \textit{bag gitteret} (mindre reservedele i den røde kasse). Opstår der tvivl om hvordan noget gøres eller repareres, så spørg mere erfarne bestyrelsesmedlemmer til råds. Søgemaskiner, YouTube og LLM'er kan også hjælpe.

\subsection{Bygningsdrift}

Nat-Tech Bygningsdrift befinder sig i Hopper 040. Fornyligt har de fået et nyt system til indmelding, og den gamle mailaddressen er blevet pillet ned. Instruktioner hænger til højre for deres kontordør.

\section{Troubleshooting af fadølsanlægget}

\begin{itemize}
    \item Er fadølsanlægget sat til strøm?
    \item Er fadkoblingen monteret rigtigt?
    \item Er slangerne sat til fadkoblingen?
    \item Er fustagen tom?
    \item Er gasflasken tom?
    \item Er der tændt for gassen?
    \item Er trykket indstillet fornuftigt?
    \item Er noget synligt i stykker?
\end{itemize}

\section{Ting at være opmærksom på}

\begin{description}
    \item[Slangerne i fadølsanlægget] skal skiftes senest hvert 5. år -- det kan anbefales at være flere om det. De er sidst blevet skiftet i april 2025.
    \item[Rensningsdunkene] har en udløbsdato. De skal skiftes inden 2027.
    \item[Gasslangerne] skal være tørlagt. Dug er normalt og gør ikke noget, men hvis der er kommet væske i skal den skiftes.
    \item[Timeren til fadølsanlægget] skal den stilles igen, hvis det tages ud af stikket. Den burde dog huske tidligere ``modes''.
    \item[Industriopvaskeren] i DatKant kan give fejlkoder, typisk grundet sensorfejl. De kan normalt cleares på `C', så en pantvagt kan forsættes. Tag dog højde for om fejlen er kritisk.
    \item[Afspændingsmiddel] til industriopvaskeren er før bestilt igennem Mads fra MatKant (\email{mac@math.au.dk}), så sørger de for at sætte en ny til.
    \item[Gas grillen] har to hovedbrændere der er i udu grundet mangel på gastilførrelse. Der er enten noget i vejen med knapperne eller kablerne.
    \item[Fadølskøleren] kan ``dryppe'' grundet kondens fra kulden når det har stået tændt tilstrækkeligt længe. 
    \item Om der skal bestilles flere reservedele til anlægget. Vi har tidligere bestilt fra \url{https://holtecsolutions.dk/}, hvor vi også har en konto. Køb skal ofte igennem Kassereren først, da de har overbliv over barens økonomi og betaling. Man kan også selv lægge ud, men her er det en god ide, at få tilladelse til dette på forhånd.
\end{description}

\end{document}
%%% Local Variables:
%%% mode: latex
%%% TeX-master: t
%%% End:

