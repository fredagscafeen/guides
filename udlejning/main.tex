% !TeX spellcheck = da_DK
\documentclass{article}
% Ability to write input files using utf8
\usepackage[utf8]{inputenc}

% Proper font, æ, ø and å becomes copy/paste and searchable
\usepackage[T1]{fontenc}
\usepackage{lmodern}

% Enable \includegraphics, so that images can be included
\usepackage{graphicx}
\DeclareGraphicsExtensions{.pdf, .png, .jpg, .PDF, .PNG, .JPG}

% Enables use of links, and adds ToC for your PDF-reader
\usepackage{hyperref}
% A small macro for inserting clickable email adresses, e.g.,
% \email{best@fredagscafeen.dk}
\newcommand*{\email}[1]{\href{mailto:#1}{\nolinkurl{#1}}}

% Adding visible TODO's using \todo
\usepackage[obeyFinal]{todonotes}

% Better kerning etc.
\usepackage{microtype}

% To create and control lists (itemize, enumeration, description)
\usepackage{enumitem}

% A new list environment that has both numbers (as enumeration) and
% labels (as description)
\newcounter{enumdescriptioncount}
\newlist{enumdescription}{description}{1}        % 1 means that it cannot be nested
\setlist[enumdescription]{%
  before = {\setcounter{enumdescriptioncount}{0}%
            \renewcommand*{\theenumdescriptioncount}{\arabic{enumdescriptioncount}}},
  font = {\bfseries\stepcounter{enumdescriptioncount}{\large \theenumdescriptioncount.}~},
  align = right,                % Makes the labels end, at the same position
  itemindent = 6em              % TODO: This can be done better
}

% Hyphenation for Danish words etc.
\usepackage[danish]{babel}

% Macros for common words, so that we can format them fancily. xspace
% inserts spaces when needed based on context.
\usepackage{xspace}

\newcommand*{\fotex}{F%
\kern -.25em%
{\raisebox{-.215em}{Ø}}%  % -.215em makes 2 Es match
\kern -.25em%
\TeX\xspace}


\title{Udlejning af \fredagscafeen's ting}
\date{\formatdate{19}{12}{2025}}
\author{Viktor Hjorth Miltersen\\
\small med tilføjelser af Anders Benjamin Clausen\\
\small \& Anders Bruun Severinsen}

\begin{document}

\maketitle

\tableofcontents \

\section{Introduktion}

I Fredagscaféen har vi følgende regelmæssigt udlejede elementer:
\begin{itemize}
	\item Lille fadølsanlæg
	\item Gril
	\item Højtalere og mixer
	\item Projekter og lærred
	\item Brætspilsvogn
	\item Festtelt
\end{itemize}

Folk kan leje dem ved at skrive en mail til \udlejningmail\ eller \bestmail.

Det er vigtigt at checke sin indbakke ofte, så hurtigst muligt kan vende tilbage
omkring en udlejning.

Derudover, skal man sørge for at holde informationen og reglerne beskrevet på hjemmesiden
så tæt på virkeligheden som muligt. Er en pris f.eks. for lav i forhold til hvad
vi betaler\footnote{Plus en rimelig margin på $\approx 20-30\%$ til baren for 
udgifter til gas og vedligehold, samt bartenderarrangementer.}, skal man sørge
for at få det opdateret, ved at lave et pull-request på hjemmesiden's GitHub:
\url{https://github.com/fredagscafeen/web}, eller skrive til den webansvarlige.

\section{Lille fadølsanlæg}
Nogle gange sender folk en mail om at de gerne vil leje det lille fadølsanlæg. 
Vi udlejer gerne ud til folk, så længe de betaler for nogle fustager. 
De betaler pr. åbnet fustage. Du kan finde alle detaljerne her: \url{https://fredagscafeen.dk/udlejning/}.

Hvis folk har spørgsmål eller ikke inkluderer nok information i deres mail, 
så henvis dem til udlejningssiden og bed dem om at læse den.

Når du har fået den information du skal bruge, kan du gå til 
\url{https://fredagscafeen.dk/admin/udlejning/udlejning/} og 
logge ind for at tilføje udlejningen. Der skal du bl.a. vælge en eller flere ``Ansvarlige''. 
Dette bringer os til næste trin, at finde en ansvarlig. For at udlejningen kan blive udført, 
skal nogle best-medlemmer sørge for at låse op og udlevere anlægget, samt at tage imod det igen. 
Derfor skal du sørge for, at finde et best-medlem der kan stå for udlejningen, 
dette er \textbf{ikke} den udlejnings-ansvarliges ansvar. 
Det er en fælles pligt der deles ligeligt mellem alle bestyrelsesmedlemmer.

Lige meget hvad, så skal det ansvarlige bestyrelsesmedlem følge proceduren i guiden, 
så sørg for at minde dem om at læse guiden.

Ligesom ved det store anlæg, så skal det lille anlæg også grundigt rengøres en gang i mellem. 
Spørg eventuelt drift/anlægsansvarlig om hjælp til hvordan anlæg skal grundigt rengøres.

\subsection{Generel info}
Denne del er både til udlejnings-ansvarlig, samt til de medlemmer der har skal stå for en 
udlejning (d.v.s. alle). En udlejning er ikke så slem, ej heller så kompliceret, 
som man måtte frygte. Man aftaler et tidspunkt med brugeren og sørger så for at møde op.

Ved aflevering af anlægget, skal man sørge for at der er en rulle plastikkopper med, 
at anlægget er i orden og man skal være forberedt på at give et lille kursus i skift af 
fustager og lignende. Hvis man gerne vil have at kunden har mulighed for at rense anlægget, 
skal man også sørge for at give dem en vanddunk med.

Nogle kunder er meget lette at udleje til, de tager bare imod anlægget og ordner det hele selv. 
Nogle gange når vi udlejer til f.eks. instituttet eller andre der står os kært, 
gør vi lidt ekstra og stiller anlægget op for dem, der hvor det skal bruges. 
Denne service yder vi ofte til de fleste ``interne'' arrangementer, d.v.s. når vi lejer ud 
til folk der har tilknytning til Fredagscaféen eller som vi gerne vil være gode venner med.

Ved modtagelse af anlægget skal man lige tjekke at det er i ordentlig stand, 
at det er blevet renset for øl (hvis ikke, så rens det lige med vand), at det ikke er klamt 
og fedtet og at det stadig virker. Det er meget vigtigt at anlægget er rent når vi udleverer 
det til den næste kunde. Når anlægget er pænt og rent låser man det bare inde i træburet 
og så er det ovre.

\subsection{Ved fejl}
Skulle det lille anlæg ske at fejle, så kontakt anlægsgartneren og se om høn kan fikse det. 
Hvis ikke, så kontakt best-listen og find hurtigst muligt ud af en ad-hoc løsning.

\section{Gril}
Mange af de samme idéer ved udlejning af fadølsanlægget gælder også her ved grillen.

Sørg for at den er i ordentlig stand ved udlevering, sørg for at den er blevet gjort ordentlig 
rent ved aflevering.

Grillen står også i træburet i Hopper-0.

Ved udlån af gril aftaler man en prisen for gassen efter forbrug.

Udlejning af gril kan findes på \\
\mbox{\url{https://fredagscafeen.dk/udlejningGrill/}},
og skal også registreres igennem admin siden.

\section{Højtalere og mixer}
Vi har fire højtalere med stativ, mixer og masser af kabler og mikrofon. 
De står i træburet. 

\section{Projektor og lærred}
Vi har ca. tre projektorer og et lærred i træburet.

Vi har før oplevet, at projektor og lærred mangler, når vi har brug for dem. 
Det er derfor vigtigt, at der laves en klar aftale om, hvornår projektor og lærred afleves tilbage. 
Vi foreslår, at dette sker \textit{max} en uge efter udlejningen sker, eller mindst et par dage 
før projektor og lærred skal bruges igen af baren eller andre.

Hvis det skulle ske, at en tilbageleveringsaftale bliver misset af lejeren, 
så får lejeren som udgangspunkt forbud for at leje projektor og lærred igen inden for 3 måneder, 
medmindre der selvfølgelig er en god årsag til at aftalen er blevet misset. 
Vær konsekvent med dette, da vi ellers ender med at mangle projektor og lærred på tidpunkter, 
hvor det er meget ubelejligt.

\section{Brætspilsvogn}
Brætspilsvognen i rummet under trappen lånes kun ud til interne arrangementer.
Kontroller ved modtagelse, at spillende ligger pænt og ikke er kastet i.
Kan en bruger ikke finde ud af at passe på spillene (eller vognen for den sags skyld),
kan det være en god idé at sige det til dem, og evt. give dem en konsekvens ved gentagende
dårlig behandling af vognen.

\section{Festtelt}
Vi har i 2025 fået et festtelt af instituttet, som hovedsageligt bruges som røgtelt til 
sommerfesten. Andre kan også låne det, men det skal være nogen man stoler på, og de skal selv 
stå for at sætte det op og tage det ned, da dette er et stort styk arbejde, som kræver mange
mennesker (minimum 4).

Det kan monteres med pløkker i jorden, eller skal på fliser fæstnes med noget tungt eller med
kamspænder.

\end{document}
