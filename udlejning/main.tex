% !TeX spellcheck = da_DK
\documentclass[danish]{article}
% Ability to write input files using utf8
\usepackage[utf8]{inputenc}

% Proper font, æ, ø and å becomes copy/paste and searchable
\usepackage[T1]{fontenc}
\usepackage{lmodern}

% Enable \includegraphics, so that images can be included
\usepackage{graphicx}
\DeclareGraphicsExtensions{.pdf, .png, .jpg, .PDF, .PNG, .JPG}

% Enables use of links, and adds ToC for your PDF-reader
\usepackage{hyperref}
% A small macro for inserting clickable email adresses, e.g.,
% \email{best@fredagscafeen.dk}
\newcommand*{\email}[1]{\href{mailto:#1}{\nolinkurl{#1}}}

% Adding visible TODO's using \todo
\usepackage[obeyFinal]{todonotes}

% Better kerning etc.
\usepackage{microtype}

% Hyphenation for Danish words etc.
\usepackage{babel}

% For conditionals in commands
\usepackage{xifthen}

% Macro for Question and answers, troubleshooting etc.
% The optional argument is a label to ref to. The label will be prefixed with "qna:"
\newcommand{\QnA}[3][]{
\paragraph{#2}
% Insert label if it is provided in the optional argument
\ifthenelse{\isempty{#1}}
  {}
  {\label{qna:#1}}
  \begin{itemize}
    #3
  \end{itemize}
}

% Macros for common words, so that we can format them fancily. xspace
% inserts spaces when needed based on context.
\usepackage{xspace}

\newcommand*{\fotex}{F%
\kern -.25em%
{\raisebox{-.215em}{Ø}}%  % -.215em makes 2 Es match
\kern -.25em%
\TeX\xspace}

% Better references, automatically uses Danish names. Use \cref
% instead of \autoref. See package documentation for more details.
%
% nameinlink makes the prefix (e.g. "kapitel") a part of the hyperlink
% (as \autoref does). This might not be a good idea if multiple labels
% are used in a \cref.
\usepackage[nameinlink]{cleveref}


\title{Udlejning af Fredagscaféens ting}
\date{\formatdate{12}{10}{2021}}

\begin{document}

\maketitle

\section{Udlejning af fadølsanlæg}

I Fredagscaféen har vi følgende regelmæssigt udlejede elementer:
\begin{itemize}
	\item Lille fadølsanlæg
	\item Gril
	\item Højtalere og mixer
	\item Projekter og lærred
\end{itemize}

\section{Lille fadølsanlæg}
Nogle gange sender folk en mail om at de gerne vil leje fadølsanlægget. Vi udlejer gerne ud til folk, så længe de betaler for nogle fustager. De betaler pr. åbnet fustage. Du kan finde alle detaljerne her: \url{https://fredagscafeen.dk/udlejning/}.

Hvis folk har spørgsmål eller ikke inkluderer nok information i deres mail, så henvis dem til udlejningssiden og bed dem om at læse den.

I bunden af siden står der hvad emailen skal indeholde, det har vist sig at det ikke helt stemmer overens med hvad vi faktisk skal bruge, så det er måske noget man gerne vil bede Web om at ændre.

Når du har fået den information du skal bruge, kan du gå til \url{https://fredagscafeen.dk/admin/} og logge ind for at tilføje udlejningen. Der skal du bl.a. vælge en "BoardMemberInCharge" dette bringer os til næste trin, at finde en ansvarlig. For at udlejningen kan blive udført, skal nogle best-medlemmer sørge for at låse op og udlevere anlægget samt at tage imod det igen. Derfor skal du sørge for at finde et best-medlem der kan stå for udlejningen, dette er \textbf{ikke} den udlejnings-ansvarliges ansvar. Det er en fælles pligt der deles ligeligt mellem alle bestyrelsesmedlemmer.

Lige meget hvad, så skal det ansvarlige bestyrrelsesmedlem følge proceduren i guiden, så sørg for at minde dem om at læse guiden.


Ligesom ved det store anlæg, så skal det lille anlæg også grundigt rengøres en gang i mellem. Spørg eventuelt drift/anlægsansvarlig om hjælp til hvordan anlæg skal grundigt rengøres.

\subsection{Udlejning}
Denne del er både til udlejnings-ansvarlig samt til de medlemmer der har skal stå for en udlejning (d.v.s. alle). En udlejning er ikke så slem, ej heller så kompliceret som man måtte frygte. Man aftaler et tidspunkt med kunden og sørger så for at møde op.

Ved aflevering af anlægget, skal man sørge for at der er en rulle plastikkrus med, at anlægget er i orden og man skal være forberedt på at give et lille kursus i skift af fustager og lignende. Hvis man gerne vil have at kunden har mulighed for at rense anlægget, skal man også sørge for at give dem en vanddunk med.

Nogle kunder er meget lette at udleje til, de tager bare imod anlægget og ordner det hele selv. Nogle gange når vi udlejer til f.eks. instituttet eller andre der står os kært, gør vi lidt ekstra og stiller anlægget op for dem der hvor det skal bruges. Denne service yder vi ofte til de fleste "interne" arrangementer, d.v.s. når vi lejer ud til folk der har tilknytning til Fredagscaféen eller som vi gerne vil være gode venner med.

Ved modtagelse af anlægget skal man lige tjekke at det er i ordentlig stand, at det er blevet renset for øl (hvis ikke, så rens det lige med vand), at det ikke er klamt og fedtet og at det stadig virker. Det er meget vigtigt at anlægget er rent når vi udleverer det til den næste kunde. Når anlægget er pænt og rent låser man det bare inde i rummet ved siden af rummet og så er det ovre.
\subsection{Ved fejl}
Skulle det lille anlæg ske at fejle, så kontakt anlægsgartneren og se om høn kan fikse det. Hvis ikke, så kontakt best-listen og find hurtigst muligt ud af en ad-hoc løsning.

\section{Gril}
Mange af de samme idéer ved udlejning af fadølsanlægget gælder også her ved grillen.

Sørg for at den er i ordentlig stand ved udlevering, sørg for at den er blevet gjort ordentlig rent ved aflevering.

Grillen står i lageret under Nygaard (man går forbi når man skal flytte fustager over i baren), og kræver derfor ikke speciel adgang udover studiekort.

Grillen mangler pt. et hjul, så bed kunden om at supplere noget håndkraft for at hjælpe med at få den derhen hvor den skal være.

Ved udlån af gril er gassen gratis.

Udlejning af gril kan findes på \url{https://fredagscafeen.dk/udlejningGrill/} og skal også registreres igennem admin siden.

\section{Specielt for projektor og lærred}
Vi har før oplevet, at projektor og lærred mangler, når vi har brug for dem. Det er derfor vigtigt, at der laves en klar aftale om, hvornår projektor og lærred afleves tilbage. Vi foreslår, at dette sker \textit{max} en uge efter udlejningen sker, eller mindst et par dage før projektor og lærred skal bruges igen af baren eller andre.

Hvis det skulle ske, at en tilbageleveringsaftale bliver misset af lejeren, så får lejeren som udgangspunkt forbud for at leje projektor og lærred igen inden for 3 måneder, medmindre der selvfølgelig er en god årsag til at aftalen er blevet misset. Vær konsekvent med dette da vi ellers ender med at mangle projektor og lærred på tidpunkter, hvor det er meget ubelejligt.

\section{Andet}
Vi har cirka fire højtalere med mixer og masser af kabler og mikrofon. De står i rummet ved siden af rummet. Der er også et lærred og en projekter, de skal også stå i rummet ved siden af rummet, men fjollede best-medlemmer har det med at stille det andre steder. Hvis dette sker, send dem da en sur-smiley og sig at de skal have mindst tre glade smileyer i streg for at genoprette deres elite-status.

Derudover gælder samme principper som ved udlejning af fadølsanlæg og gril.


\end{document}
