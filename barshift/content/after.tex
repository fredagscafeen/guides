\section{After the Bar Shift}
\label{sec:post-bar-shift}

After the bar shift, two main things need to happen: 
The space needs to be cleared and the cleaning needs to be done.

\subsection{Cleanup and shutdown}
\label{sec:post:cleanup}

\begin{itemize}
	\item Put things back where they came from (see
	\autoref{sec:pre-bar-shift}).
	\item Remember to power up the refrigerators Behind the Grille,
	but disconnect the refrigerators in the Bar Area.
	The refrigerators in the Bar Area must be left ajar,
	by inserting a wooden block in the gap when the refrigerators are locked again.
	\item Remember to connect the bar to the timer behind the grill so that it
	is ready for the next bar shift.
	\item Boxes with empty bottles must go into the Wooden Cage in Hopper -1.
	\item Deposit bottles must go into the deposit bin in the parking garage.
	\item Regular glass bottles without a deposit must be sorted into glass waste in the parking garage.
	\item The speakers are turned off with the lock on the post so that it does not buzz.
\end{itemize}

\subsection{Cleaning}
\label{sec:post:cleaning}

The purpose of the cleaning is to make it look neat, roughly as if
there had been no Friday bar, but just ``normal'' use of the
premises. There is a cleaning cart in the
cleaning room opposite the disabled toilets in Nygaard,
and you need a bar key to lock yourself in.

Here is a list of the different things that need to be cleaned:
\begin{itemize}
	\item Collect deposits and garbage. Empty the bottom slats into a slat bucket.
	\item The tables must be wiped down with soapy water.
	\begin{itemize}
		\item This applies to the tables the customers have been sitting at, 
		the two high tables in the bar, the liquor cart and the bar itself.
	\end{itemize}
	\item The system must be flushed through. See this \href{https://media.fredagscafeen.dk/guides/rensningafanlaeg.pdf}{guide} for more
	details. Remember to turn off the gas afterwards.
	\item The spill tray under the taps must be washed and wiped down.
	\item The floor must be swept and washed.
	\begin{itemize}
		\item Here you just need to focus on stains from things that have been spilled:
		Therefore, you usually don't need to wash the entire floor.
		\item However, it is important to remove beer stains, as they can become very
		sticky. Things like beer pong tend to generate such stains.
		\item It is also a good idea to wash the standing mat that has been behind the bar.
	\end{itemize}
	\item Wash the bottle stoppers with soapy water and wipe them with a rag.
	\item Wash bottle openers that are damp.
	\item The trash cans in the Nygaard basement must be emptied and replaced with new trash bags.
	\item The toilets:
	\begin{itemize}
		\item Check that they have been flushed out and any disgusting situations are cleaned.
		\item Wash the floor if necessary, and put paper on the floor in the trash can.
		\item Empty the trash cans and put new bags in.
	\end{itemize}
	\item Outside:
	\begin{itemize}
		\item Check that there are no bottles or similar on the main staircase,
		where smokers usually sit.
	\end{itemize}
\end{itemize}

\subsection{Rounding off}
\label{sec:post:rounding-off}

When you think you're done, the bartender in charge goes through the closing checklist,
which is on the bar. Here you say ``yes'' if you've done what's being read out,
and it's here that you make sure everything's done. Finally, the bartender in charge thanks you and you're free to leave.
