\section{During the Bar Shift}
\label{sec:during}

During the shift there are various tasks that need to be completed.

Remember to wash or sanitize your hands several times during the bar shift
(alcohol is available at the bar).

\subsection{Sale of Goods}
\label{sec:intra:sale}

Various products will be sold.
\begin{itemize}
    \item The price can be found in the \textit{Zettle}\index{Zettle} app. If it cannot be found here,
    then try scanning the barcode on the item, or ask a board member.
    \item We accept cash, credit cards (Dankort, VISA, MasterCard
    etc.), and ``on the list''.
    %\item All items sold must be registered through the Zettle app, so we know how much we have in stock.
    \begin{itemize}
    \item Remember to check that the payment was approved.
    \item The lowest amount you can pay is 10 kr..
    \end{itemize}
    \item Payments on the \textit{Tab}\index{Tab} are for people who have created an account
    (``tab'') at \fredagscafeen. 
    Read the corresponding \href{https://media.fredagscafeen.dk/guides/krydsliste.pdf}{guide} to the Tab.
    \item Cups and jugs from customers must not be refilled, as we cannot guarantee that they are clean.
    Therefore, fill a new one instead and put the others to wash.
    \item When selling bottles, they must be opened immediately.
\end{itemize}

Remember, we are NOT allowed to sell alcohol to children under 18.
It happens rarely, but if you have the slightest doubt, ask for ID.
They must be able to show valid photo ID, and it is better to ask once too much than once too little.

\subsection{Refill}
\label{sec:intra:refill}

During the bar shift, we have to make sure that we have something to sell. 
Therefore, the refrigerators have to be filled up along the way if they are running low. 
If there is a certain beer that is popular, customers will be happy if it is available in 
the refrigerator and not just sitting out in the warehouse. Similarly, new supplies of kegs 
often have to be picked up during the bar shift.

\begin{quote}
  Use the same principle as with toilet rolls: 
  Get some new ones while you still have one left.
\end{quote}

\subsection{Dinner}
\label{sec:intra:dinner}

\begin{center}
$\dotfill$ REMEMBER RECEIPT! $\dotfill$
\end{center}
At some point there will also be some food on board.
Together you choose a place (typically in Storcenter Nord)
where you would like to get food from.
The amount per person should be a maximum of 70kr.
This fits in with, among other things:
\begin{itemize}
    \item A sandwich from Anettes Sandwich
    \item A roll, pita or bowl from Pita Planet
    \item A falafel roll from Rulle Far
    \item Four cheeseburgers from Carls
    \item Salad or sandwich from DatKant
\end{itemize}
If the amount is over 70 kr., you pay the difference yourself.
That is, if you buy dinner for 90 kr., you pay 20 kr. at the bar.,
and this is noted on the receipt.
It is typically easiest to agree on one place to shop at,
but it is up to those on duty.

Once you have chosen where to buy,
you choose one person to note the others' preferences,
and go off and buy it.
You must remember to bring contacts from
\textit{cash register}\index{Cash register} to pay for the food.
Change and receipt are then placed back in the cash register.
\begin{center}
  $\dotfill$ REMEMBER RECEIPT! $\dotfill$
\end{center}

\subsection{Pant}
\label{sec:intra:pant}

There are different kinds of \textit{pant}\index{Pant}, and it needs to be sorted.
\begin{description}
    \item[Cans] All cans, whether regular or special
    beer/soda, must be placed in the large white deposit box.
    \item[Regular beer bottles] Sportcola, Tuborg, Top, etc.\ must be placed in a regular beer box.
    \item[Hancock soda bottles] Hancock soda must be placed in their own light green boxes. These can be found in the wooden cage.
    \item[Regular bottled soda] Regular bottled soda, such as Schweppes, must be placed in a
    red soda box. If you don't have one, you can use a regular beer box as a last resort, and later place it correctly in the Wooden Cage.
    \item[Other glass bottles \underline{with deposit}] These must be placed in a transparent
    deposit plastic bag: These are usually on top of the cooling system in the bar. If not, there are several in the Wooden Cage.
    \begin{itemize}
    \item These bags should only be filled to the point where they can still be carried, and transported to the underground parking garage,
    without tipping over or breaking.
    \item After the bar guard, the bags are emptied into the gray deposit container in the underground parking garage.
    \end{itemize}
    \item[Other] Items that do not belong to any of the above
    categories should be placed in the glass container in the underground parking garage,
    and collected in a black garbage bag by the bar. This applies, for example, to liquor bottles.
\end{description}

If the deposit box becomes full, it must be emptied and a new bag put in.
The filled bag is transported to the Wooden Cage and closed with a strip.

\subsection{Other}
\label{sec:intra:other}

Bartenders get, in addition to dinner, a bag of chips, and free soda during their bar shift.

We rent out various board games/dice/playing cards/ping pong balls/pool.
Require an ID (usually a student card) as a deposit for ping pong balls and pool,
so we are sure to get these back.
There is a box in the bar to store these IDs.

It is possible to play music during the bar shift, by using the
jack in one of the posts, connected to the iPad on the bar.
To do this, you first need a bar key
for the lock on the post, to turn on the speakers.
Spotify is installed on the iPad, so it's just a matter of choosing something to play.
The bartenders are the DJs, so it is up to them whether drunk customers are allowed to put music in line, etc.
Please wait to turn on the music until after 5 PM, so that students have the opportunity to study in Nygaard in the afternoon.

Remember to clear the tables of trash and empty bottles along the way so that new customers can come.

The bell is usually rung \textit{last round}\index{Last round}
about fifteen minutes before the bar closes
(i.e. at 9:45 PM). There is a large and beautiful bell that the lucky bartender
is allowed to ring.

At 10:00 PM we play ``sod off''-music and no more sales are made.
