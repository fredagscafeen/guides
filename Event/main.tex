% !TeX spellcheck = da_DK
\documentclass{article}
% Ability to write input files using utf8
\usepackage[utf8]{inputenc}

% Proper font, æ, ø and å becomes copy/paste and searchable
\usepackage[T1]{fontenc}
\usepackage{lmodern}

% Enable \includegraphics, so that images can be included
\usepackage{graphicx}
\DeclareGraphicsExtensions{.pdf, .png, .jpg, .PDF, .PNG, .JPG}

% Enables use of links, and adds ToC for your PDF-reader
\usepackage{hyperref}
% A small macro for inserting clickable email adresses, e.g.,
% \email{best@fredagscafeen.dk}
\newcommand*{\email}[1]{\href{mailto:#1}{\nolinkurl{#1}}}

% Adding visible TODO's using \todo
\usepackage[obeyFinal]{todonotes}

% Better kerning etc.
\usepackage{microtype}

% To create and control lists (itemize, enumeration, description)
\usepackage{enumitem}

% A new list environment that has both numbers (as enumeration) and
% labels (as description)
\newcounter{enumdescriptioncount}
\newlist{enumdescription}{description}{1}        % 1 means that it cannot be nested
\setlist[enumdescription]{%
  before = {\setcounter{enumdescriptioncount}{0}%
            \renewcommand*{\theenumdescriptioncount}{\arabic{enumdescriptioncount}}},
  font = {\bfseries\stepcounter{enumdescriptioncount}{\large \theenumdescriptioncount.}~},
  align = right,                % Makes the labels end, at the same position
  itemindent = 6em              % TODO: This can be done better
}

% Hyphenation for Danish words etc.
\usepackage[danish]{babel}

% Macros for common words, so that we can format them fancily. xspace
% inserts spaces when needed based on context.
\usepackage{xspace}

\newcommand*{\fotex}{F%
\kern -.25em%
{\raisebox{-.215em}{Ø}}%  % -.215em makes 2 Es match
\kern -.25em%
\TeX\xspace}


\title{Eventguide}
\date{\today}
\author{Gitte Pedersen}

\begin{document}

\maketitle

Kære eventansvarlige. Velkommen til en af de absolut sjoveste poster i
Fredagscaféens historie! Du kommer fremover til at være i spidsen for
de kommende arrangementer og events - og du kan godt glæde dig
\Smiley: For at lede dig på rette vej, og give dig et godt overblik
over hvilke spændende events der i løbet af året vil finde sted, kan
du i nedenstående afsnit læse omkring mine, og tidligere
Eventansvarliges erfaringer, som vi har erfaret os med denne event
post. Nedenstående afsnit vil derfor være et rigtigt godt udgangspunkt
for hvordan eventdelen af baren kan drives. Har du lyst at lave
ændringer eller tilføje ting til de forskellige events, så fyr endelig
løs!

\section{Det essentielle - lead by example, deltagelse og sjov}
\label{sec:det-essentielle}

Det overordnede mål er at holde events, som kan tage folk lidt ud af
den sædvanlige trummerum. Det betyder, at de skal være sjove,
underholdende og helst lidt anderledes end hvad der normalt foregår
nede i baren. Ironisk nok er den sværeste del ved disse events at
sørge for, at folk rent faktisk deltager, og får det ud af det de
gerne vil. Erfaring viser, at den nemmeste måde at overtale folk til
at deltage og være med, er at DU som event formand går forrest og
viser hvordan det skal gøres. Herunder skal du også sørge for altid at
være forud med planlægningen, så du i god tid har styr på de praktiske
ting til de forskellige events. Dette vil der stå mere om i det næste
afsnit omkring planlægningsværktøjer.

Hvis du som arrangør hopper i med begge ben, og i øvrigt ser ud til at
have det skide skægt mens du gør det, vil gæsterne næsten altid gøre
det samme, og så er mission lykkedes! For at gøre dette så nemt som
muligt er det derfor en smule vigtigt at du laver events som du selv
gerne vil deltage i, og møder op til dem. Så sørg for at få det
essentielle gjort ordentligt, så skal alt det andet nok falde på
plads.

\section{Det praktiske - planlægningsværktøjer}
\label{sec:det-praktiske}

Events der skal afholdes, skal planlægges først. Derfor har det de
sidste par gange været et punkt til bestyrelsesmøderne når der har
været et event der skal planlægges.

Når punktet over et event er overstået, bør følgende være klart:
\begin{itemize}
\item Hvad eventet går ud på
\item Hvornår det foregår, og
\item Hvem der står for div. praktiske ting.
\end{itemize}

Man kan med fordel uddele opgaverne, hvis man ikke ønsker eller har
mulighed for at tage dem alle, men det er dit ansvar som
eventansvarlig at have det samlede overblik og vide hvem der gør
hvad. Det kan være lettere at holde styr på, hvis man bruger sin
kalender flittigt og har en form for "to-do" liste.

Oftest, når et event er planlagt, skal der også sendes emails og
doodler ud til bartenderne, og reminder emails om at huske at svare,
samt reminder emails om at huske at møde op etc. Igen er det her smart
med et todo system, eller endnu bedre bruge en automatisk email
sender, så de alle bare kan blive skrevet på en gang.

Emails indeholder som oftes div. Info om eventet og info om hvornår
doodlen lukker. Reminder emails bør sendes dagen før.

\section{Planlægning og møder}
\label{sec:planlagning-og-moder}

Er der nogle ting du har overvejet, er i tvivl om eller generelt bare
godt kunne tænke dig at få andre fra Bestyrelsens mening omkring, kan
du tage det op til et af bestyrelsesmøderne. I så fald, husk at skriv
til formanden før et bestyrelsesmøde om der et eller flere event
punkter til dagsorden, så tilføjer han det nemlig.

\subsection{Selve mødet}
\label{sec:selve-modet}

På selve mødet skal kalenderen gennemgås for datoer, hvor der kan
holdes events (lad eksempelvis være med at hold events oven i
TAAGEKAMMERET-fester osv.). Det bedste er selv at vælge en dato,
fremfor at sende endnu flere doodles til bartenderne for at høre
hvornår de bedst kan. Når datoer til potentielle events er fundet, kan
man begynde at brainstorme idéer. Når det vurderes at der er nok gode
idéer at vælge imellem, vælges en idé, og konceptet udvides og
præciseres. Efter dette er gjort, laves en liste over praktiske ting
der skal ordnes, og de bliver fordelt hvis man ikke selv kan klare dem
alle. Alt efter hvilke events der laves, kan nogle af de praktiske
ting eksempelvis være:
\begin{itemize}
\item Køb/betaling af mad (eksempelvis sandwich hos Annettes Sandwich
  eller delikatesse mad hos \fotex)
\item Pynt
\item Transport
\end{itemize}

Sørg også for at tage referat og få dette referat sendt ud på mail
efter mødet så medlemmerne og dig selv har en overordnet plan at gå ud
fra. Derudover kan det også være en god ide at holde en liste over
idéer til events, da disse kan hjælpe hvis der ikke er nogen ideer
eller brainstormingen går lidt trægt. Husk evt. også at snakke med
propaganda-ansvarlig, så der eventuelt kan komme et opslag med
informationer på Fredagscaféens Facebook-side, så eventet kan blive
hypet og delt bredt blandt de studerende.

\subsection{Oprydning}
\label{sec:oprydning}

Oprydning efter events tager generelt omkring en time til halvanden
hvis man gør det alene. Det tager kortere tid hvis du får andre til at
hjælpe dig. \allemail\ er dine venner.

Når du beder om hjælp er det en god ide at gøre det ydmygt, men
konkret. Som med alt andet er det nødvendigt at du går forrest. Sørg
for at gøre det klart du forventer at rydde og at du gerne vil have
hjælp. Specificer et konkret tidsrum hvor oprydningen vil finde sted
(f.eks. mellem klokken 20 og 22) og hvor lang tid du forventer den vil
tage (typisk omkring en halv time). På den måde er det konkret nok til
at man bliver nødt til at tage sin tilmelding til rengøringen seriøst,
men ikke så præcist man kan bruge undskyldninger som ``der skal jeg
spise''.

\section{Årskalender}
\label{sec:arskalender}

Events er listet under det kvarter hvor de skal planlægges. Events i
paranteser betyder at der skal laves noget indledende planlægning (som
at finde dato), men ikke detaljeret planlægning (som hvad vi skal have
at spise). Der skal holdes følgende events:\todo{Fix alignment}
\begin{description}
\item[Q1]
  \begin{itemize}
  \item Fødselsdag
  \end{itemize}
\item[Q2]
  \begin{itemize}
  \item (Julefrokost)
  \item (Juleklip)
  \item Bartender-event?
  \end{itemize}
\item[Q3]
  \begin{itemize}
  \item Julefrokost
  \item Generalforsamling
  \item (Aarhus Bryghus tur)
  \end{itemize}
\item[Q4]
  \begin{itemize}
  \item Aarhus Bryghus tur
  \item Djurs Sommerland (om sommeren)
  \end{itemize}
\end{description}

Der har de sidste par år været et voldsomt overskud i baren. Det har
været foreslået at dette overskud kunne bruges til et bartender-only
event i Q2 som fx paintball eller go-kart eller lignende. Såfremt der
er overskud der skal brændes af, er dette en god kandidat. Snak med
regnskabsafdelingen og de andre i BEST inden der tages en endelig
beslutning.

\subsection{Julefrokost}
\label{sec:julefrokost}

Julefrokosten er for alle bartenderne i Fredagscaféen. Sørg derfor for
at finde en dato og send doodles ud til bartenderne.

Sidste år blev arrangementet holdt i Kaffestuen i Hopper, hvor der
blev bestilt 3. retters menu fra \fotex, hvilket vi fik relativ god
feedback på. Der var dog ret meget salat ift. kød, så bestilles der
igen ved \fotex\ kan det overvejes, om der måske skal være en del mere
kød og mindre salat. Det kan også overvejes, om der skal laves mad fra
bunden af. I så fald kræver dette god planlægning, og selvfølgelig
nogen der gider at stå for maden. Dette kan dog overvejes.

Lige meget hvad er det vigtigt at du som eventansvarlig i god tid
sørger for at få fat i hjælpere til at pynte op inden julefrokosten,
og nogen til at hjælpe med at rydde op dagen efter. Til dem der melder
sig til at rydde op dagen efter kan det være en god idé at sørge for
goodiebags (cola, slik, andre godter). Dette vil gøre hele oprydningen
lidt bedre :)

Ydermere er det traditionen tro, at der vil være pakkeleg!! Husk
derfor at informere bartenderne om dette i god tid.

\subsection{Aarhus Bryghus}
\label{sec:aarhus-bryghus}

Et af de traditionelle events er den årlige tur til Aarhus Bryghus!
Her skal du sørge for at kontakte Aarhus Bryghus og aftale en
ølsmagningstur for x antal personer. Det kan i den her case være en
god idé først at få datoen på plads med Aarhus Bryghus, før der sendes
informationer og doodles ud til bartenderne.

Når datoen er valgt, kan der eventuelt sendes en doodle ud hvor
bartenderne kan klikke ``deltager'' samt hvilken sandwich de ønsker.

Efter turen skal der desuden også være mad til bartenderne. Sidste år
blev der bestilt sandwichs fra Anettes Sandwich, hvilket var en god og
nem løsning.

\subsection{Djurs Sommerland}
\label{sec:djurs-sommerland}

Hvert år i sommerferien arrangeres der en årlig tur for bartenderne
til Djurs Sommerland. Som eventansvarlig er det dit ansvar at stå for
det praktiske. Find en dato, send en mail ud til bartenderne med
informationer (dato, tid, mad, drikke, transport). De tidligere år har
maden bestået af pizza buffet på pizzariaet i Djurs Sommerland.

Turen er gratis for bartenderne i baren, og inkluderer mad, drikke og
tur-retur transport. Du skal derfor sørge for at tjekke bustider (se
eventuelt inde på Midttrafik).

\section{Ekstern Kommunikation}
\label{sec:ekst-komm}

Som eventansvarlig er det dit ansvar at kommunikere med virksomheder
der gerne vil afholde sponsorbar. Sørg derfor for at holde en liste
med sponsorbar-kunder. Der kan nemt opstå kø blandt virksomheder, og
hvis du ikke holder styr på det, bliver det noget rod.

Listen skal indeholde følgende:
\begin{itemize}
\item Dato for første henvendelse
\item Header på e-mailen (så du kan finde den)
\item Navn på virksomhed og kontakt person
\end{itemize}

\subsection{Sponsorbarer}
\label{sec:sponsorbarer}

Studerende kan godt lide gratis øl, men gider ikke skulle betale for
dem med opmærksomhed eller tvungen modtagelse af flyers eller alt
muligt andet gøjl. Virksomheder har mange penge og ikke særligt mange
gode måder at bruge dem på, hvis de vil have opmærksomhed fra
studerende.

Målet er at begynde at holde \emph{én sponsorbar pr. semester},
hvilket gerne må foregå oven i de vagter hvor vi har længe åbent.

De umiddelbare rammer for sponsorbarer er derfor som følger:
\begin{itemize}
\item De skal give ~10 fustager fadøl af 700 kr. stykket -> 7000kr i
  alt.
\item For den pris køber de et medium langt oplæg (~20 min)
  \begin{itemize}
  \item Vi stiller PA-anlæg til rådighed og sørger for opsætning
  \item Vi sørger for PR om nødvendigt
  \end{itemize}
\item Gæstebartendere er en god ide, hvis man gerne vil have kontakt
  til de studerende.
\item Yderligere ting de kommer med kan blive tilføjet, men vi gider
  altså ikke ting som at man skal op og snakke med dem så man kan få
  en billet til den gratis øl. De giver øllen væk og for det får de
  mulighed for at holde et oplæg. Gæster skal holde deres kæft og vise
  respekt under oplægget, men når oplægget er færdigt så skal gæsterne
  ikke gøre mere for at få øl. Hvis virksomheden gerne vil stille
  underholdning til rådighed (lad være med at godtage stand-up) eller
  lignende der kunne være fedt så er det helt i orden (overvej om du
  som kunde gerne vil have det).
\item Sørg for at booke det lille anlæg da der bliver brug for dette
  (send en mail til \bestmail).
\end{itemize}


Tues (gammel eventansvarlig) standardmail var som følger:
\begin{quote}
  Hej X

  Det lyder rigtig spændende [eller whatever, vis entutiasme for
  ideen]. [Indsæt svar på eventuelle spørgsmål der ikke er relateret
  til nedenstående]. For at holde sponsorbarer som et unikt
  arrangement begrænser vi antallet til to om året og der er desværre
  y virksomheder i kø foran jer. Jeg kan tilføje jer til køen og
  kontakte jer når det bliver jeres tur.

  Rammerne for sponsorbar er som følger:

  I får mulighed for at holde et mellemlangt (~20 min) oplæg med den
  information i gerne vil ud med. Vi stiller PA anlæg til rådighed og
  sørger for PR hvis nødvendigt. Prisen for dette er at I giver øl;
  ~10 fustager. Prisen pr. fustage er 700 kr.

  Andre virksomheder der har afholdt sponsorbarer har med succes brugt
  tiden efter oplægget som hjælpe/gæste bartendere i baren, for at få
  mere personlig kontakt. I har selvfølgelig samme mulighed.

  Mvh Tue Holst, på vegne af Fredagscaféen.
\end{quote}

\section{Dødsquake}
\label{sec:dodsquake}

De studerende på IT og Datalogi kan godt lide events hvor det er
muligt at game med og mod hinanden. Der holdes derfor Dødsquake ca. én
gang hvert semester. Vær opmærksom på, at det ikke ligger for tæt på
en sponsorbar eller samtidigt med et andet event.

\section{Tour de Fredagsbar}
\label{sec:tour-de-fredagsbar}

Erfaringer har vist sig at det er en god ide at holde sig indenfor
normale studie perioder (læs: lad være med at arrangere disse i
eksamens- eller ferie-perioder og vær opmærksom på at andre fakulteter
har forskellige kalendere i forhold til eksamener osv.).

\section{Epilog}
\label{sec:epilog}

Jeg håber, at du fandt denne guide hjælpsom. Er du i tvivl eller har
spørgsmål omkring hvordan et event skal holdes, eller hvordan du får
de praktiske ting på plads, kan du altid spørge de andre
bestyrelsesmedlemmer, tage det op til næste bestyrelsesmøde, eller
sende en mail til tidligere eventansvarlig i baren på mailen
\email{gitte-pedersen@live.com}.

Jeg håber, du får holdt nogle fede events!

\end{document}

%%% Local Variables:
%%% mode: latex
%%% TeX-master: t
%%% End:
