% !TeX spellcheck = da_DK
\documentclass{article}
% Ability to write input files using utf8
\usepackage[utf8]{inputenc}

% Proper font, æ, ø and å becomes copy/paste and searchable
\usepackage[T1]{fontenc}
\usepackage{lmodern}

% Enable \includegraphics, so that images can be included
\usepackage{graphicx}
\DeclareGraphicsExtensions{.pdf, .png, .jpg, .PDF, .PNG, .JPG}

% Enables use of links, and adds ToC for your PDF-reader
\usepackage{hyperref}
% A small macro for inserting clickable email adresses, e.g.,
% \email{best@fredagscafeen.dk}
\newcommand*{\email}[1]{\href{mailto:#1}{\nolinkurl{#1}}}

% Adding visible TODO's using \todo
\usepackage[obeyFinal]{todonotes}

% Better kerning etc.
\usepackage{microtype}

% To create and control lists (itemize, enumeration, description)
\usepackage{enumitem}

% A new list environment that has both numbers (as enumeration) and
% labels (as description)
\newcounter{enumdescriptioncount}
\newlist{enumdescription}{description}{1}        % 1 means that it cannot be nested
\setlist[enumdescription]{%
  before = {\setcounter{enumdescriptioncount}{0}%
            \renewcommand*{\theenumdescriptioncount}{\arabic{enumdescriptioncount}}},
  font = {\bfseries\stepcounter{enumdescriptioncount}{\large \theenumdescriptioncount.}~},
  align = right,                % Makes the labels end, at the same position
  itemindent = 6em              % TODO: This can be done better
}

% Hyphenation for Danish words etc.
\usepackage[danish]{babel}

% Macros for common words, so that we can format them fancily. xspace
% inserts spaces when needed based on context.
\usepackage{xspace}

\newcommand*{\fotex}{F%
\kern -.25em%
{\raisebox{-.215em}{Ø}}%  % -.215em makes 2 Es match
\kern -.25em%
\TeX\xspace}


\title{Guide til ølbestilling og lager beholdning}
\author{Torben Cortnum Jørgensen}
\date{\formatdate{15}{02}{2026}}

\begin{document}

\maketitle

\section{Introduction}
Denne guide vil forklare hvordan ølbestilling og modtagelse på nuværende tidspunkt udføres, samt igennem hvilke forhandlere.
Den vil også forklare hvor meget man bestiller, ting at tage hensyn til fast inventar og diverse.

\section{Forhandlere}
Lige nu bliver der bestilt igennem 2 forhandlere, One Pint og Ølhuset. De har har forskellige styrker. 
One Pint har det faste sortiment i magners, orval og fynsk forår. De plejer at have de bedste tilbud, men det er ikke
lige så tit og så meget man har at vælge mellem.\\
Ølhuset er en utrolig lang liste, og har løbende tilbud på tingene, og har de mere spændende ting normalt.\\ \\
Generelt gælder det at hvis man bestiller inden klokken 12-13 om mandagen får man varerne i samme uge.


\subsection{One Pint}
One Pint er stedet vi køber al vores cider magners har været fast inventar længe men går lige så stille op i pris så man måske skal skifte på et tidspunkt.
Kontakt personen ved One Pint er Heine Kristensen, han sender ugentligt hvis man melder sig til hans mailingssliste, og som fanges på heine@onepint.dk.\\
Her kan man også købe ting såsom glas nogle gange. Bestillingskravet for fri levering her er 4000 kroner før moms. Trikket er du næsten altid bare kan 
fylde helt op med magners fustager, der hurtigt får en derop, da de plejer at kunne holde sig længe nok til man ikke behøves bekymre sig om dato.
Prøv at time bestilling af nye magners fustager med når der er gode ontrade tilbud.\\ \\
Levering af One Pint fungerer ved at de normalt kommer i løbet af dagen Onsdag eller Torsdag (mest Onsdag), hvor de plejer at ringe 30 minutter før. 
Nogle gange kan de også være der meget tidligt eller ret sent så nogle gange kan du høre resten af bestyrelsen om de er hjemme. Angående hvem det er 
der kommer med varene kan det også ske de bare stiller en palle lige udenfor porten af P-kælderen, så skal man huske at stille en palle derud næste gang
de kommer forbi fordi der er pant på pallerne.

\subsection{Ølhuset}
Ølhuset er der vi har bestilt mest i dette år (25/26) da der er mere udvalg og mere skiftende sortiment. Tilbuddene er der også hver uge, men man skal selv
finde dem i filen. Det vigtigste du kan finde her du ikke kan ved One Pint er sour som plejer at gå ret hurtigt. Her er kravene for gratis levering
12 kasser øl eller 6 fustager. Ølhuset er begyndt ret konsekvent at stille nede ved porten men ringer/skriver stadig, baseret på hvem der leverer. De leverer
også mest onsdag og torsdag. 

\section{Hvad/hvor meget der bør bestilles}
At bestille ind er en kunst. Den vigtigste regel er, at du næsten aldrig, skal bestille mere end 3 kasser. Så begynder det at blive for meget, med mindre du
er heldig og folk kan lide det. Der er også en ting med folk i datbar, at de gerne smager noget en enkelt gang, men efter det kommer det an på smag. Det
betyder du altid kan slippe afsted med en kasse øl. Endnu en psykologisk ting er, at hvis man fylder en masse nye ting ind, og de har gamle ting imens, 
tager folk bare det nyeste hele tiden. Derfor skal man pace hvornår man sætter nye ting ind, hvis man har noget andet der er ved at gå på dato.
Vi har også sat alt vores sour i køleskabet længst inde, da denne resten af det køleskab sælger sig selv, og så har du et helt skab med ting du skal 
blive af med, i det der står længt ud mod kunderne. \\ \\
Generelt vil du gerne bestille stort i starten af semesteret og hvis der bliver drukket meget, kan du
lave en stor bestilling mere, men man skal sørge for at lageret er ret tomt, når man når til læseferien, da det er 2-3 måneder der ikke går meget øl.

\subsection{Hvad for nogle øl? Hvordan ved jeg om de er gode, jeg er ikke alkoholiker?}
Hvis du er i tvivl om en øl er god, så kan du altid gå på https://untappd.com/ og se hvad den er ratet der. Vi har kørt med en cut off omkring 3.5, ved
meget billige special kan man sænke kravet lidt. Generelt kommer de alkoholfrie øl til at være ratet lavere.\\
Folk i datbar plejer at ville have eksotiske ting. Derfor skøre designs og smage plejer at blive modtaget godt. Selve øltyperne går vi efter at have hylder
der er en type for sig. Sour, IPA, Stout, er de vigtigste. Man kan også gå efter at have en belgisk hylde og en lager. Men lager er farlig da denne plejer
at være for normal til at folk gider drikke den for meget mere end ceres. Vi plejede at have en hylde glutenfrie øl, men da de mest prominente 
glutenallerkier ikke kommer så meget mere burde man nok ændre den, da folk antager glutenfrie øl er dårligere, og dermed sælger værre.\\
Hvis prisen du skal sælge for går over 50 kroner skal det kunne holde sig længe og du må ikke bestille meget, og skal helst være en stor dåse med en del
procenter i.\\\\
Pro tip til sidst. Hvis du synes en øl ser skør og spændende ud kan du næsten altid købe en kasse, og der er nok der har samme indstilling som dig.
 
\subsubsection{Special Fadøl}
Det her er et ret frit område, med du skal næsten aldrig bestille mørkt. De er sværre at sælge og plejer at være dyre, men kan holde sig længere. 

\subsection{Hvor meget? Jeg vil ikke skulle drikke det hele selv}
Vi plejede at gå efter at have 3-4 øl på hver hylde, og så nærmest intet på lager. Dette er mere intensivt i forhold til at skulle bestille tittere, men
sørger for at du kan reagere på tilbud, og sjældnere skal sætte ting ned i pris fordi de er gået på dato, som hurtigt kan give underskud. Rent praktisk kan
man have rigtig meget på lager ad gangen. Hvis man laver en stor bestilling kan man have en palle bagerst, hvor du stapler på, så 70+ kasser, men desto mere
du har på lager, dest besværgligere er det når du skal hente nyt i køleskabene. Inklusive hvad der er i køleeskabene ville jeg nok gå efter 25 kasser. 

\subsubsection{Special Fadøl}
Special Fadøl kan vi absolut ikke anbefale at have mere end 3 fustager af. Hvis du rammer en der er dårlig eller svær at sælge, plejer de hurtigere at gå
på dato, og da vi kun har en hane er det svært at gøre med ved det. Man kan også smage når de ikke er helt så friske længere, og de skummer meget når du hælder dem.\\ \\
Hvis du har en god fustage på går den nok på 2-3 uger, hvis den ikke er så god er det nok 1-2 måneder før den er done. (kommer også meget an på pris)

\subsubsection{Jeg har fucked up. Har for meget øl/øl der er gået på dato}
Hvis du har for meget special øl, der snart går op dato kan du snakke med din PR ansvarlige, og lave plakater/ugens spotlight. Disse hjælper overraskende meget,
specielt hvis priserne forbundet med dem er fornuftige (Så ikke sæt alt for dyre ting på den). Hvis du virkelig har for meget laver du et event af en type.
Hvis det er fadøl kan du hente det lille anlæg frem, så du har flere haner. Hvis det er dåser og det kun er en type kan du lave indkøbspris eller lignende,
i skal bare informere folk om at det er over bedst før dato.\\ \\
Andre gode måder at blive af tingene på, er ved månedens øl på hjulet. Hvor man kan sætte en øl man vil af med til at være den (det er det feltet er lavet til, shhhh).
Hvis det står kritisk til kan man putte flere månedens øl på.


\subsection{Pant}
Ved skrivning af denne guide bliver alt pant håndteret af lager ansvarlig, inklusive magners fustagerne.


\section{Prislisten}
Vi går ca. efter en avance på 20\% på vores ting.\\
For at genere en prisliste er der lavet et feature på hjemmesiden under admin, hvor man kan gå ind og tilføje de forskellige øl og deres priser.
Så kan man tilføje dem de forskellige hylder og til sidst printe en prisliste man kan hænge op. For at de er i Zettle systemet skal man dog også tilføje dem der,
hvor man i optimal tilfældet også tilføjer barkoden, der gør det at bartender meget nemmere kan finde dem.\\ \\
(En mulighed hvis man selv/eller web er motiverede, er at lave en funktion på hjemmesiden der kalder zettle API`en og opretter produktet begge steder)

\section{Special tilbud jagt}
Nogle gange holder nogle danske bryggerier vildt gode tilbud på deres øl ved store bestillinger. Dette kan man overveje at gøre som engagnsting. Hvis man synes
det er for meget at bestille kan man høre fysisk fredagsbar og/eller TÅGEKAMMERET om de vil være med på det. For at pisen er lav nok skal det være maks 20 pr. dåse
ved ffb og 14-15 ved TÅGEKAMMERET. Nævneværdige tidspunkter at kigge er black friday ved GAMMA, ÅBEN, TOØL, Ebeltoft og Mikkeller.






\end{document}
%%% Local Variables:
%%% mode: latex
%%% TeX-master: t
%%% End:
