% !TeX spellcheck = da_DK
\documentclass[danish]{article}
% Ability to write input files using utf8
\usepackage[utf8]{inputenc}

% Proper font, æ, ø and å becomes copy/paste and searchable
\usepackage[T1]{fontenc}
\usepackage{lmodern}

% Enable \includegraphics, so that images can be included
\usepackage{graphicx}
\DeclareGraphicsExtensions{.pdf, .png, .jpg, .PDF, .PNG, .JPG}

% Enables use of links, and adds ToC for your PDF-reader
\usepackage{hyperref}
% A small macro for inserting clickable email adresses, e.g.,
% \email{best@fredagscafeen.dk}
\newcommand*{\email}[1]{\href{mailto:#1}{\nolinkurl{#1}}}

% Adding visible TODO's using \todo
\usepackage[obeyFinal]{todonotes}

% Better kerning etc.
\usepackage{microtype}

% Hyphenation for Danish words etc.
\usepackage{babel}

% For conditionals in commands
\usepackage{xifthen}

% Macro for Question and answers, troubleshooting etc.
% The optional argument is a label to ref to. The label will be prefixed with "qna:"
\newcommand{\QnA}[3][]{
\paragraph{#2}
% Insert label if it is provided in the optional argument
\ifthenelse{\isempty{#1}}
  {}
  {\label{qna:#1}}
  \begin{itemize}
    #3
  \end{itemize}
}

% Macros for common words, so that we can format them fancily. xspace
% inserts spaces when needed based on context.
\usepackage{xspace}

\newcommand*{\fotex}{F%
\kern -.25em%
{\raisebox{-.215em}{Ø}}%  % -.215em makes 2 Es match
\kern -.25em%
\TeX\xspace}

% Better references, automatically uses Danish names. Use \cref
% instead of \autoref. See package documentation for more details.
%
% nameinlink makes the prefix (e.g. "kapitel") a part of the hyperlink
% (as \autoref does). This might not be a good idea if multiple labels
% are used in a \cref.
\usepackage[nameinlink]{cleveref}


\title{Guide til ølbestilling og lager beholdning}
\author{Philip Athanassios Tzannis}
\date{\today}

\begin{document}

\maketitle

\section{Introduction}
Denne guide vil forklare hvordan ølbestilling og modtagelse på nuværende tidspunkt udføres, samt igennem hvilke forhandlere.

\section{Forhandlere}
På nuværende tidspunkt bruges to forhandlere; der kan dog godt tilføjes om man mener det relevant. Begge steder har varierende salgs ikke-alkoholiske drikkevarer.
\subsection{One Pint}
One Pint er der vi får de fleste af vores ting fra, da de typisk er den billigere forhandler i de fleste tilfælde, samt har et større sortiment. Det er her vi køber Høkerbajer, Ale nr. 16 og Limfjordsportere, ting der bør holdes som fast inventar. Endvidere er der her vi køber ultra pres og cider.\\
Kontakt personen ved One Pint er Heine Kristensen, der ofte sneder tilbud på sms, og som fanges på heine@onepint.dk.
\subsection{Klosterbryggeriet}
Klosterbryggeriet er hvor vi får de fleste belgiske fra, da de ud over at have mange også er billigere på den front.\\
Kontakt personen ved One Pint er Signe Palm, der sender mange mails om diverse tilbud, bestilling foregår igennem info@klosterbryggeriet.dk.

\section{Hvad/hvor meget der bør bestilles}
Det er vigtigt at vi altid har en mængde ikke alkoholiske drikkevarer stående, udover de alkoholiske. Der bør altid bestilles et par kasser ultra pres hjem når vi er på lavt niveau. Når det kommer til øl prøver jeg som regelt at bestille ret forskelligt, således der er noget til en hver smag. I forhold til hvor meget der så reelt set skal bestilles, så kommer det til dels an på hvor ofte man er interesseret i at skulle modtage og sætte ind på lageret; dette er en process der snildt kan tage et par timer. Volumens mæssigt er der plads til et sted mellem 30-40 kasser øl. Cider skal stå i rummet ved siden af rummet, og det er ofte smart at have en 6-8 stående
\subsection{Pant}
Der bør under bestilling også føres pant for både cider og paller; dette skrives blot i mailen at skal gøres.

\end{document}
%%% Local Variables:
%%% mode: latex
%%% TeX-master: t
%%% End: