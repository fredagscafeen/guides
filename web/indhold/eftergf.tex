\section{Efter Generalforsamling}
\label{sec:efter-generalforsamling}

Efter generalforsamlingen er der en række ting, der skal opdateres på hjemmesiden.
Vi tilføjer generelt den nye bestyrelse med det samme, men fjerner først den gamle bestyrelse, 
efter første bestyrelsesmøde.

\subsection{Ny Bestyrelse}
\label{sec:ny-bestyrelse}

\subsubsection{Før Første Bestyrelsesmøde}
\label{sec:foer-foerste-bestyrelsesmoede}

\begin{itemize}
    \item Tilføj den nye bestyrelse til \url{https://fredagscafeen.dk/board/}. 
    Dette kan gøres ved at kalde følgende script på serveren:
    {\small\begin{verbatim}
$ docker-compose exec app ./manage.py new_board --start-date=YEAR-MM-DD\end{verbatim}}
    \item Tilføj de nye bestyrelsesmedlemmer til bestyrelses mailinglisten \email{best@fredagscafeen.dk}:\\
    {\small\url{https://maillist.au.dk/postorius/lists/datcafe-best.cs.maillist.au.dk/}}
    \item Tilføj de nye bestyrelsesmedlemmer til Discord gruppen, og lav en ny bestyrelses rolle og kanal.
\end{itemize}

\subsubsection{Efter Første Bestyrelsesmøde}
\label{sec:efter-foerste-bestyrelsesmoede}

\begin{itemize}
    \item Opdater \url{https://fredagscafeen.dk/board/} med billeder og titler.
    % Der er tradition for at man bruger børnebilleder af sig selv, men det er ikke et krav.
    \item Tilføj de nye bestyrelsesmedlemmer til GitLab gruppen
    med ``Maintainer'' rolle, og fjern de gamle bestyrelsesmedlemmer:\\
    {\small\url{https://gitlab.au.dk/groups/fredagscafeen/-/group_members}}
    \item Fjern de gamle bestyrelsesmedlemmer fra bestyrelses mailinglisten:\\
    {\small\url{https://maillist.au.dk/postorius/lists/datcafe-best.cs.maillist.au.dk/}}
    \item Fjern admin rettigheder på hjemmesiden fra de gamle bestyrelsesmedlemmer, og tilføj de nye.
    Kan gøres ved at kalde følgende script på serveren:
    {\small\begin{verbatim}
$ docker-compose exec app ./manage.py update_board_member_accounts\end{verbatim}}
    \item Lav nye kodeord og opdater \texttt{.env} filen på serveren:\\
    {\small\url{https://fredagscafeen.dk/da/admin/admin/secrets/}}
    Undgå at lave Google passwordet til \email{datcafe@gmail.com} om, men blot ændre hvem der har 
    2-faktor enheder. Typisk er det nok at formanden og den webansvarlige har adgang.
\end{itemize}

\subsection{Ny Webansvarlig}
\label{sec:ny-webansvarlig}

Dette skal kun gøres, hvis der er valgt en ny webansvarlig.
\begin{itemize}
    \item Lav den nye webansvarlige til superbruger på hjemmesiden.
    \item Tilføj ssh adgang til serveren for den nye webansvarlige.
    Dette gøres ved at kopiere den nye webansvarliges ssh nøgle til 
    \texttt{\textasciitilde/.ssh/authorized\_keys} på serveren.
    \item Den nye webansvarlige skal have ``Owner'' rolle i GitLab gruppen.
    \item Den nye webansvarlige tilføjes til GitHub organisationen med ``Owner'' rolle:\\
    {\small\url{https://github.com/orgs/fredagscafeen/people}}
    \item Tilføj den nye webansvarlige til \texttt{ADMINS} i \texttt{base.py} filen i\\
    \texttt{fredagscafeen/settings/}, så den nye webansvarlige får emails ved serverfejl:\\
    {\small\url{https://github.com/fredagscafeen/web/blob/master/fredagscafeen/settings/base.py}}
\end{itemize}
