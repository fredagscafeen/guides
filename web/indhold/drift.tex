\section{Vedligeholdelse og Drift}
\label{sec:vedligeholdelse-og-drift}

På serveren kan man køre forskellige scripts, der gør livet lettere som webansvarlig.
Serveren tilgås med ssh:

\begin{verbatim}
$ ssh ubuntu@fredagscafeen.dk
\end{verbatim}

For at tilgå Git-repo'et, skal man ind i mappen \texttt{/web}:

\begin{verbatim}
$ cd web
\end{verbatim}

\subsection{Ny Barplan}
\label{sec:ny-barplan}

Et par uger før barplanen udløber, skal der genereres en ny barplan.
Men før man gør dette, sender man en mail til de andre bartendere (\email{alle@fredagscafeen.dk}), 
der informerer om, at der snart bliver genereret en ny barplan, og at de skal huske at vælge hvilke 
fredage, de ikke har mulighed for at stå i bar, på deres profil på hjemmesiden: 
\url{https://fredagscafeen.dk/da/profile/}. 
En template for denne mail kan ses i bilag~\ref{appendix:ny_barplan_mail_template}.

Når barplanen skal genereres, kører man følgende script på serveren:

\begin{verbatim}
$ docker-compose exec app ./manage.py generate_barshifts
\end{verbatim}

Når man kører scriptet, bliver bartendernes valg taget i betragtning, og de bliver ikke sat på de
fredage, de har fravalgt. Derudover bliver man også informeret om, hvis der er bartendere, der har
valgt de ikke kan stå i bar på noget tidspunkt i perioden, og derfor ikke bliver sat på nogen vagter.
Perioden der bliver genereret, er på $2n$ uger, hvor $n$ er antallet af bestyrelsesmedlemmer.
Scriptet forsøger at fordele vagterne således at alle dage bliver dækket, og forsøger så vidt muligt
at fordele vagterne jævnt mellem bartenderne.
Når scriptet har kørt færdigt, får man listen over fordelingerne, og man kan vælge at godkende eller
afvise den.

\subsection{Nye Pantvagter}
\label{sec:nye-pantvagter}

Før de nuværende pantvagter udløber, skal der genereres nye pantvagter.
Den webansvarlige vælger selv, hvornår de nye pantvagter skal genereres.

Når der skal genereres nye pantvagter, køres følgende script på serveren:

\begin{verbatim}
$ docker-compose exec app ./manage.py generate_deposit_shifts
\end{verbatim}

Når scriptet har kørt færdigt, får man listen over fordelingerne, og man kan vælge at godkende eller
afvise den. Man kan vælge at afvise den, hvis det f.eks. viser sig at en person har fået flere vagter 
i træk.

\subsection{Opdatering af Hjemmesiden}
\label{sec:opdatering-af-hjemmesiden}

Det kan ske, at der skal opdateres forskellige ting på hjemmesiden, hvis der f.eks. er fejl, men også
hvis der skal tilføjes nye features.
Koden til hjemmesiden ligger på GitHub: \url{https://github.com/fredagscafeen/web}.
Man bestemmer selv hvilke ændringer, der skal laves, og hvordan de skal laves. Dog er det altid en god
idé at indrage de berørte parter, hvis det er større ændringer, der skal laves. En liste over de
aktive issues/opgaver kan ses på GitHub: \url{https://github.com/fredagscafeen/web/issues}. Markér
opgaverne som færdige, når de er løst, og tilføj løbende flere.\\

En god måde at sætte sig ind I koden på, er at starte med at refactor. Man kan refactor en hel masse, 
uden først at vide hvad alting er og gør. Brug principperne fra SWEA, og ellers kan bogen 
``Five Lines of Code'' af Christian Clausen (2021) anbefaldes. Husk også at refactor efter at 
have tilføjet ændringer. \\

Før man pusher ændringer, er det vigtigt først at teste dem lokalt. Opsætningen heraf, 
kan læses i repository'ets README fil.
Efter man har testet at ændringerne virker, som de skal, pusher man ændringerne til \textbf{master}
branchen på GitHub. Derefter venter man på at GitHub Actions bygger, tester og deployer ændringerne:
\url{https://github.com/fredagscafeen/web/actions}.

Når Actions er færdig, forbinder man til serveren, henter ændringerne fra Git, og genstarter containeren:

\begin{enumerate}
    \item Pull ændringer og genstart container:
    \begin{verbatim}
$ git pull && docker-compose pull && docker-compose restart\end{verbatim}
    \item Byg og start container:
    \begin{verbatim}
$ docker-compose build && docker-compose up -d\end{verbatim}
\end{enumerate}

\subsubsection{Internationalisering}

Hjemmesiden er internationaliseret, så den kan ses på både dansk og engelsk.
Dette er implementeret ved hjælp af Django's middleware kaldet \texttt{LocaleMiddleware}.
Dette bruges nemt i python koden ved at importere \texttt{gettext\_lazy}:
\begin{verbatim}
from django.utils.translation import gettext_lazy as _
\end{verbatim}
På denne måde er det eneste man skal gøre, at omkredse teksten, der skal oversættes, med \texttt{\_()}:
\begin{verbatim}
_("Hello, world!")
\end{verbatim}
I templates bruges \texttt{\{\% translate "Hello, world!" \%\}} eller 
\texttt{\{\% blocktranslate \%\}}-\texttt{\{\% endblocktranslate \%\}}
Husk her altid at sikre at \texttt{\{\% load i18n \%\}} er tilstede i toppen af filen!

Herefter kaldes følgende kommando, for at generere oversættelsesfilerne:
\begin{verbatim}
$ ./manage.py makemessages --all
\end{verbatim}
For at gøre det nemmere at oversætte teksten, har vi tilføjet udvidelsen \texttt{rosetta}, som
gør det muligt som admin, at oversætte teksten direkte på hjemmesiden. Dette gøres ved at tilgå
\url{http://127.0.0.1:8000/rosetta/} lokalt!

Når oversættelserne er færdige, skal de kompileres:
\begin{verbatim}
$ ./manage.py compilemessages
\end{verbatim}

Læs mere om oversættelse i Django's dokumentation, 
\href{https://docs.djangoproject.com/en/5.1/topics/i18n/translation/}{her}.

\subsection{Nye Bartendere}
\label{sec:nye-bartendere}

Efter en bartenderansøgning er godkendt på et bestyrelsesmøde, skal den nye bartender tilføjes til
mailinglisten (\email{alle@fredagscafeen.dk}) og facebook gruppen 
(\url{https://www.facebook.com/groups/1200445467453617}).

\subsection{Brugbare Kommandoer}
\label{sec:brugbare-kommandoer}

Udover de allerede nævnte kommandoer, er der en række andre kommandoer, der kan være nyttige at kende til.

\begin{figure}[H]
    \centering
    \begin{tabular}{l l}
        \textbf{Beskrivelse} & \textbf{Kommando} \\
        \hline
        & \\
        \textit{Linux} & \\
        & \\
        \begin{minipage}
            {0.5\textwidth}
            Se indhold i mappe
        \end{minipage}& 
        \begin{minipage}
            {0.5\textwidth}
            \texttt{\$ ls}
        \end{minipage} \\
        & \\
        \begin{minipage}
            {0.5\textwidth}
            Se nuværende sti
        \end{minipage}& 
        \begin{minipage}
            {0.5\textwidth}
            \texttt{\$ pwd}
        \end{minipage} \\
        & \\
        \begin{minipage}
            {0.5\textwidth}
            Se størrelse på partitions
        \end{minipage}& 
        \begin{minipage}
            {0.5\textwidth}
            \texttt{\$ df -Th}
        \end{minipage} \\
        & \\
        \hline
        & \\
        \textit{Docker} & \\
        & \\
        \begin{minipage}
            {0.5\textwidth}
            Se alle containers
        \end{minipage}&
        \begin{minipage}
            {0.5\textwidth}
            \texttt{\$ docker ps -a}
        \end{minipage} \\
        & \\
        \begin{minipage}
            {0.5\textwidth}
            Slet alle ubrugte docker images og containers
        \end{minipage}&
        \begin{minipage}
            {0.5\textwidth}
            \texttt{\$ docker system prune -a}
        \end{minipage} \\
        & \\
        \hline
        & \\
        \textit{Git} & \\
        & \\
        \begin{minipage}
            {0.5\textwidth}
            Kør Git's Garbage Collector
        \end{minipage}&
        \begin{minipage}
            {0.5\textwidth}
            \texttt{\$ git gc}
        \end{minipage} \\
        & \\
    \end{tabular}
    \caption{Brugbare Kommandoer}
    \label{fig:brugbare-kommandoer}
\end{figure}
