% !TeX spellcheck = da_DK
\documentclass{article}
% Ability to write input files using utf8
\usepackage[utf8]{inputenc}

% Proper font, æ, ø and å becomes copy/paste and searchable
\usepackage[T1]{fontenc}
\usepackage{lmodern}

% Enable \includegraphics, so that images can be included
\usepackage{graphicx}
\DeclareGraphicsExtensions{.pdf, .png, .jpg, .PDF, .PNG, .JPG}

% Enables use of links, and adds ToC for your PDF-reader
\usepackage{hyperref}
% A small macro for inserting clickable email adresses, e.g.,
% \email{best@fredagscafeen.dk}
\newcommand*{\email}[1]{\href{mailto:#1}{\nolinkurl{#1}}}

% Adding visible TODO's using \todo
\usepackage[obeyFinal]{todonotes}

% Better kerning etc.
\usepackage{microtype}

% To create and control lists (itemize, enumeration, description)
\usepackage{enumitem}

% A new list environment that has both numbers (as enumeration) and
% labels (as description)
\newcounter{enumdescriptioncount}
\newlist{enumdescription}{description}{1}        % 1 means that it cannot be nested
\setlist[enumdescription]{%
  before = {\setcounter{enumdescriptioncount}{0}%
            \renewcommand*{\theenumdescriptioncount}{\arabic{enumdescriptioncount}}},
  font = {\bfseries\stepcounter{enumdescriptioncount}{\large \theenumdescriptioncount.}~},
  align = right,                % Makes the labels end, at the same position
  itemindent = 6em              % TODO: This can be done better
}

% Hyphenation for Danish words etc.
\usepackage[danish]{babel}

% Macros for common words, so that we can format them fancily. xspace
% inserts spaces when needed based on context.
\usepackage{xspace}

\newcommand*{\fotex}{F%
\kern -.25em%
{\raisebox{-.215em}{Ø}}%  % -.215em makes 2 Es match
\kern -.25em%
\TeX\xspace}


\title{Pantvagt}
\date{\formatdate{17}{12}{2025}}
\author{Anders Bruun Severinsen}

\begin{document}

\maketitle

\section{Oversigt}
\label{sec:oversigt}

Pantvagt udføres en gang om ugen, og handler hovedsageligt om, at
vaske barens brugte glas og genanvendelige plastikkopper fra sidste fredag,
i opvaskerummet ved \textit{DatKant}.

\section{Rækkefølge}
\label{sec:rakkefolge}

\begin{enumerate}
    \item Proppen sættes i opvaskeren, hvis den ikke allerede er.
    \item Opvaskeren tændes og man starter programmet ``Vandpåfyldning''.
    \item Følgende gøres:
    \begin{itemize}
        \item Brugte kopper eller glas fordeles ud på opvaskebakker. 
        Opvaskebakkerne med plastikkopper kan stables 
        to i højden, og altid med en tom bakke på toppen, for at kopperne ikke 
        flyver rundt og vælter.
        \item Brugte kander fordeles på en bakke for sig selv, og en tom bakke placeres 
        på hovedet på toppen.
        \item Brugte shotsbakker placeres med fordel på de blå opvaskebakker, så de kan stå
        delvist oprejst.
    \end{itemize}
    \item Når opvaskeren er færdig, startes den næste vask på default programmet.
    \item Vaskede effekter tørres af med et viskestykke, eller lufttørres ved at 
    bunden af effekterne først tørres af for vand, et nyt viskestykke fordeles 
    flat på toppen, og bakken stilles i en af opvaskebakkehylderne.
    \item Punkt 3 til 5 gentages indtil alle effekter er vasket op og tørt af eller 
    stillet til tørre.
    \item Opvaskeren slukkes
    \item Filteret over vandet i opvaskeren tømmes for skidt i en skraldespand og 
    skyldes af i en vask.
    \item Proppen i opvaskeren trækkes ud, så vandet kan løbe ud.
    \item Bordet ved opvaskeren tørres af for vand med en klud eller et viskestykke.
    Det skal helst se lige så godt ud eller bedre, når man forlader rummet.
\end{enumerate}

\section{Troubleshooting}
\label{sec:troubleshooting}

\QnA[opvasker-fejlkode]{Opvaskeren melder fejl}{
    \item Her er det ofte nok at trykke på `C'-knappen for at clear fejlen.
    \item Hjælper dette ikke, hængerer der en mere detaljeret guide på væggen
    bag opvaskeren.
}

\QnA[viskestykker]{Ingen (tørre) viskestykker}{
    \item Har man ingen viskestykker i opvaskerummet, eller er de alle
    våde, kan man:
    \begin{enumerate}
        \item Kigge i rengøringsrummet.
        \item Spørge rengøringen efter flere.
        \item Kigge i skuffer ved te-køkkenerne.
    \end{enumerate}
}

\QnA[afstændingsmiddel]{Der mangler afspændingsmiddel eller lign.}{
    \item Mangler der afspændingsmiddel eller lign., kan der skrives 
    til Mads Ankjær fra Matematisk Kantine (\email{mac@math.au.dk}),
    for at få noget nyt eller hjælp til at bestille noget hjem.
}

\section{Meta}
\label{sec:meta}

Man har pantvagt 2 uger i træk. Den første uge er med en, der lige
har været i oplæring ugen før, og den sidste uge er man så ansvarlig
for at oplære den næste.

Man aftaler indbyrdes, hvornår man mødes. 
Hvis man ser, at der ikke er sat nogen på som pantvagt, bør man sende 
en mail til \webmail.

\end{document}

%%% Local Variables:
%%% mode: latex
%%% TeX-master: t
%%% End:
