\documentclass[a4paper,danish]{article}

\usepackage[utf8]{inputenc}
\usepackage[danish]{babel}
\title{Guide til pizza- og burgerbestilling}
\author{}

\begin{document}

\maketitle

\section{Før I bestiller}
\begin{itemize}
  \item Noter navne for alle bestillinger
  \item Lav en oversigt, så man kan se, hvor mange af hver pizza/burger, der skal bestilles
    \item Gem begge ovenstående lister
\end{itemize}
\section{Når I bestiller}
\begin{itemize}
  \item Send en SMS til 42 47 61 62, hvor du skriver:
    \begin{itemize}
      \item alle bestillinger (pizza/burger-nummer og antal)
      \item at der skal numre på alle æsker
      \item at pizzaerne skal slices (burgerne skal ikke slices)
      \item Leveringsadressen: \textbf{Finlandsgade 21}
      \item HUSK AT SIGE, \textbf{AT DER SKAL NUMRE PÅ ÆSKERNE}
      \item HUSK AT SIGE, \textbf{AT PIZZAERNE SKAL SLICES} 
    \end{itemize}
\end{itemize}

\section{Mens I venter}
\begin{itemize}
  \item Hav telefonen klar, når den ringer
  \item Find lige penge frem (Pris i kroner = (antal pizzaer + antal burgere) $\cdot$ 55)
\end{itemize}

\section{Når pizzaerne er der}
\begin{itemize}
  \item Brug de gemte lister til at checke, om alle pizzaer/burgere er der
  \item Udlevér pizzaerne ved navneopråb, én ad gangen
\end{itemize}

\end{document}
