% !TeX spellcheck = da_DK
\documentclass[danish]{article}
% Ability to write input files using utf8
\usepackage[utf8]{inputenc}

% Proper font, æ, ø and å becomes copy/paste and searchable
\usepackage[T1]{fontenc}
\usepackage{lmodern}

% Enable \includegraphics, so that images can be included
\usepackage{graphicx}
\DeclareGraphicsExtensions{.pdf, .png, .jpg, .PDF, .PNG, .JPG}

% Enables use of links, and adds ToC for your PDF-reader
\usepackage{hyperref}
% A small macro for inserting clickable email adresses, e.g.,
% \email{best@fredagscafeen.dk}
\newcommand*{\email}[1]{\href{mailto:#1}{\nolinkurl{#1}}}

% Adding visible TODO's using \todo
\usepackage[obeyFinal]{todonotes}

% Better kerning etc.
\usepackage{microtype}

% Hyphenation for Danish words etc.
\usepackage{babel}

% For conditionals in commands
\usepackage{xifthen}

% Macro for Question and answers, troubleshooting etc.
% The optional argument is a label to ref to. The label will be prefixed with "qna:"
\newcommand{\QnA}[3][]{
\paragraph{#2}
% Insert label if it is provided in the optional argument
\ifthenelse{\isempty{#1}}
  {}
  {\label{qna:#1}}
  \begin{itemize}
    #3
  \end{itemize}
}

% Macros for common words, so that we can format them fancily. xspace
% inserts spaces when needed based on context.
\usepackage{xspace}

\newcommand*{\fotex}{F%
\kern -.25em%
{\raisebox{-.215em}{Ø}}%  % -.215em makes 2 Es match
\kern -.25em%
\TeX\xspace}

% Better references, automatically uses Danish names. Use \cref
% instead of \autoref. See package documentation for more details.
%
% nameinlink makes the prefix (e.g. "kapitel") a part of the hyperlink
% (as \autoref does). This might not be a good idea if multiple labels
% are used in a \cref.
\usepackage[nameinlink]{cleveref}


%opening
\title{Forretningsorden for \fredagscafeen~ Bestyrelse}
\author{\fredagscafeen~ Bestyrelse 2025}
\date{\formatdate{22}{3}{2025}}

\renewcommand{\thesection}{§\arabic{section}}

\begin{document}

\maketitle

\section{Bestyrelsesmøder}
Formanden indkalder til bestyrelsesmøderne. Indkaldelsen skal udsendes senest en uge før mødet, 
hvor en endelig dagsordenen også sendes med. Formanden sender et udkast til dagsordenen ud to 
uger før mødet. Det forventes, at alle bestyrelsesmedlemmer deltager i alle bestyrelsesmøder. 
En afstemning omkring næste møde udsendes umiddelbart efter hvert møde.

\section{Afstemninger til bestyrelsesmøder}
Hvor andet ikke er anført, træffes alle afgørelser ved simpelt flertal blandt de tilstedeværende 
stemmeberettigede. Bestyrelsen er beslutningsdygtig til møder, hvis mindst halvdelen af 
stemmeberettigede er til stede under afstemning. Ved stemmelighed er formandens stemme afgørende.

Hvis blot én stemmeberettiget ønsker det, skal afstemningen være hemmelig. Hvis en stemmeberettiget 
ikke er til stede, kan vedkommende afgive sin stemme pr. telefon, Discord el.lign. under mødet.

Hvis en stemmeberettiget har stærke personlige forbindelser til en sag, kan vedkommende udelukkes, 
fra at deltage, i afstemninger om denne sag. Det afgøres fra gang til gang, hvornår dette er tilfældet.

\section{Beslutninger mellem bestyrelsesmøder}
Beslutninger foretages via mail og andre af bestyrelsen godkendte kommunikationsformer. Mails internt 
i bestyrelsen besvares inden for 48 timer på hverdage, ellers 72 timer. Afstemninger afgøres ved simpelt 
flertal blandt alle bestyrelsesmedlemmer. I tilfælde, hvor alle som har svaret inden for tidsfristen er 
enige, anses tavshed som en blank stemme. Ved vigtige beslutninger skal bestyrelsen informeres via andre 
kanaler (SMS, telefonopkald mv.). Flest mulige bestyrelsesmedlemmer skal inddrages.

Hvis et bestyrelsesmedlem, i en periode, ikke forventer at kunne kontaktes indenfor førnævnte interval, 
bør vedkommende informere resten af bestyrelsen.

Beslutninger føres til referat ved næste bestyrelsesmøde. Den ansvarlige for dagsordenen er ansvarlig 
for at udarbejde en oversigt over de trufne beslutninger inden hvert bestyrelsesmøde. Henvendelser til 
bestyrelsen besvares inden for en uge, dog ikke nødvendigvis ved vigtige beslutninger. Henvendelser skal 
kvitteres snarest.

\section{Opgavestyring}
Bestyrelsen benytter referatet til administrering af interne opgaver. Det forventes at bestyrelsesmedlemmer 
tjekker det seneste mødes referat og får lavet deres opgaver efter bedste evne.

\section{Afbud}
Hvis et bestyrelsesmedlem er forhindret i at deltage i et bestyrelsesmøde, skal dette meddeles bestyrelsen 
tidligst muligt. Hvis et bestyrelsesmedlem er fraværende uden afbud, angives dette i mødereferatet.

\section{Mødereferat}
Ved begyndelse af et bestyrelsesmøde vælges en referent, som skal være et bestyrelsesmedlem. Referatet skal 
indeholde en liste over deltagende bestyrelsesmedlemmer, fraværende bestyrelsesmedlemmer, suppleanter og 
observatører. Referatet udsendes af referenten til alle bestyrelsesmedlemmer senest en uge efter 
bestyrelsesmødet til godkendelse. Hvis nogen fra bestyrelsen har indsigelser eller rettelser til referatet 
sendes disse til bestyrelsen inden for en uge efter modtagelse. Referenten lægger referatet i et arkiv på 
GitLab.

\section{Tavshedspligt}
Bestyrelsen og inviterede deltagere til bestyrelsesmøderne har tavshedspligt i forbindelse med personsager, 
private forhold og i øvrigt hvor tavshedspligt er påkrævet. Bestyrelsen kan herudover kræve tavshedspligt 
ved simpelt flertal.

\section{Økonomi}
På bestyrelsesmøderne giver kassereren en redegørelse for foreningens økonomiske situation og eventuelle 
uforudsete hændelser. Hvis det ønskes at afholde arrangementer uden for vores almene arrangementer, skal et 
budget fremlægges til bestyrelsen inden bestyrelsen vedtager, om arrangementet kan afholdes.

\section{Gyldighed}
Denne forretningsorden er vedtaget den 22. marts 2025 af bestyrelsen og er gældende for bestyrelsen og 
alle deltagere til bestyrelsesmøder, indtil en ny forretningsorden godkendes. Senest ved første ordinære 
bestyrelsesmøde efter generalforsamlingen skal bestyrelsens forretningsorden fremgå som punkt på dagsordenen 
og vedtages med evt. rettelser eller tilføjelser. Ændring af forretningsordenen kræver 2/3 flertal blandt 
bestyrelsen.

\end{document}
