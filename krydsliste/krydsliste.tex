\documentclass{article}
\usepackage{graphicx} % Required for inserting images

\usepackage{geometry}
 \geometry{
 a4paper,
 total={170mm,257mm},
 top=20mm,
 }

\title{Guide til krydslisten}
\author{Emil Sixhøj Mors}
\date{June 2023}

\begin{document}

\maketitle
Krydslisten benyttes på følgende måde:

\begin{itemize}
    \item Når en kunde betaler på krydslisten, skrives købet i "Køb"-kolonnen.
    \item Ønsker kunden at indsætte penge, skrives det indsatte beløb i "Indsat"-kolonnen. Hvis der indsættes kontanter, markeres det i kolonnen yderst til venstre (den med mønter). Der må IKKE skrives tal i denne kolonne.
    \item Er en kunde i kreditstop, kan de ikke købe på krydslisten før de har betalt deres gæld. Hvis gælden er betalt, udstreges "KREDITSTOP" i kolonnen.
    \item Hvis en ny kunde gerne vil på krydslisten, skriv da en "indgang" med følgende informationer:
    \begin{itemize}
        \item Indsat beløb
        \item Eventuelt om det er kontant
        \item Navn (med efternavne, ikke kaldenavne)
        \item Email-adresse
        \item Kundens køb den aften
    \end{itemize}
\end{itemize}

Vær opmærksom på følgende, for ikke at lave de hyppige fejl:
\begin{itemize}
    \item Skriv med en mørk kuglepen, som ikke er sort. Dvs. ikke tuscher, blyanter eller lignende
    \item Forskellige køb adskilles tydeligt med enten "+" eller "-". IKKE med komma
    \item Hvis man skriver forkert, overstreg da tydeligt det forkerte tal, og skriv det igen
    \item Hvis en kolonne udfyldes helt, lav da en ny kolonne, istedet for at skrive videre ved siden af.
    \item Skriv tydeligt med blokbogstaver/tal. Vær især opmærksom på følgende:
    \begin{itemize}
        \item "1" og "7" taller ligner hinanden
        \item "4" og "9" taller ligner hinanden
        \item Om "0"-taller ikke er lukkede, og dermed kan ligne et "6"-tal
        \item At navne og emailadresser på nye kunder skrives med blokbogstaver
    \end{itemize}
    \item Hold ALTID krydslisten tør, ISÆR i forhold til sprit.
\end{itemize}
\end{document}
