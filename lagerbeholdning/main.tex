% !TeX spellcheck = da_DK
\documentclass[danish]{article}
% Ability to write input files using utf8
\usepackage[utf8]{inputenc}

% Proper font, æ, ø and å becomes copy/paste and searchable
\usepackage[T1]{fontenc}
\usepackage{lmodern}

% Enable \includegraphics, so that images can be included
\usepackage{graphicx}
\DeclareGraphicsExtensions{.pdf, .png, .jpg, .PDF, .PNG, .JPG}

% Enables use of links, and adds ToC for your PDF-reader
\usepackage{hyperref}
% A small macro for inserting clickable email adresses, e.g.,
% \email{best@fredagscafeen.dk}
\newcommand*{\email}[1]{\href{mailto:#1}{\nolinkurl{#1}}}

% Adding visible TODO's using \todo
\usepackage[obeyFinal]{todonotes}

% Better kerning etc.
\usepackage{microtype}

% Hyphenation for Danish words etc.
\usepackage{babel}

% For conditionals in commands
\usepackage{xifthen}

% Macro for Question and answers, troubleshooting etc.
% The optional argument is a label to ref to. The label will be prefixed with "qna:"
\newcommand{\QnA}[3][]{
\paragraph{#2}
% Insert label if it is provided in the optional argument
\ifthenelse{\isempty{#1}}
  {}
  {\label{qna:#1}}
  \begin{itemize}
    #3
  \end{itemize}
}

% Macros for common words, so that we can format them fancily. xspace
% inserts spaces when needed based on context.
\usepackage{xspace}

\newcommand*{\fotex}{F%
\kern -.25em%
{\raisebox{-.215em}{Ø}}%  % -.215em makes 2 Es match
\kern -.25em%
\TeX\xspace}

% Better references, automatically uses Danish names. Use \cref
% instead of \autoref. See package documentation for more details.
%
% nameinlink makes the prefix (e.g. "kapitel") a part of the hyperlink
% (as \autoref does). This might not be a good idea if multiple labels
% are used in a \cref.
\usepackage[nameinlink]{cleveref}


\title{Lager Beholdning}
\author{Philip Athanassios Tzannis}
\date{\today}

\begin{document}

\maketitle

\section{Lager og kæleskabs beholdning}
Lageret kan groft deles op i to dele, venstre og højre side. Disse er distinkte, men kan gøres ret forskelligt.
\subsection{Venstre side}
Øl! Det er her på disse hylder hvor alle special øllene står, samt hvor kasserne med almindeligt øl står. Der skal så vidt muligt ikke være mere end 12 kasser, da det begrænser adgang til hylderne. Selve hylderne bør stå nogenlunde sorteret ølmæssigt, men det er meget op til den ansvarlige. Personligt har jeg om noget foretrukket at hold de belgiske faste (Rochefort, Westmalle) på nederste hylder, Fynsk Forår på hylden ummidelbart under hvor baren sættes til strøm, for gamle eller fåtallige øl på selve den hylde, og store samt dyre øl øverst til venstre.
\subsection{Højre side}
Her er ikke alkoholiske drikkevare, samt spiritus og andet stærkere alkohol, som mjød. Det er også her hvor sodavandskasser skal stå. Ud over dette er der ikke meget.

\section{Kæleskabene}
Der er to kæleskabe, og disse varier meget ud fra hvad der bliver sat i af bartenderne den givne vagt, men nogle ting bør holdes:\\
2 hylder i et skab dedikeret til standard pilsnerne(Top, Hoff, Tuborg) samt almene speciale (LFP, Høker, Ale 16). Endvidere er det en fordel med Fynsk Forår. I samme skab bør nederste to hylder være sodavand og special ikke alkoholiske. Det andet skab bør have nederste hylde til belgiske standard(Westmalle, Rochefort) samt stor special, inklusiv dBock, på øverste hylde.
\end{document}
%%% Local Variables:
%%% mode: latex
%%% TeX-master: t
%%% End:
