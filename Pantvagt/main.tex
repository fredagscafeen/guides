% !TeX spellcheck = da_DK
\documentclass[danish]{article}
% Ability to write input files using utf8
\usepackage[utf8]{inputenc}

% Proper font, æ, ø and å becomes copy/paste and searchable
\usepackage[T1]{fontenc}
\usepackage{lmodern}

% Enable \includegraphics, so that images can be included
\usepackage{graphicx}
\DeclareGraphicsExtensions{.pdf, .png, .jpg, .PDF, .PNG, .JPG}

% Enables use of links, and adds ToC for your PDF-reader
\usepackage{hyperref}
% A small macro for inserting clickable email adresses, e.g.,
% \email{best@fredagscafeen.dk}
\newcommand*{\email}[1]{\href{mailto:#1}{\nolinkurl{#1}}}

% Adding visible TODO's using \todo
\usepackage[obeyFinal]{todonotes}

% Better kerning etc.
\usepackage{microtype}

% Hyphenation for Danish words etc.
\usepackage{babel}

% For conditionals in commands
\usepackage{xifthen}

% Macro for Question and answers, troubleshooting etc.
% The optional argument is a label to ref to. The label will be prefixed with "qna:"
\newcommand{\QnA}[3][]{
\paragraph{#2}
% Insert label if it is provided in the optional argument
\ifthenelse{\isempty{#1}}
  {}
  {\label{qna:#1}}
  \begin{itemize}
    #3
  \end{itemize}
}

% Macros for common words, so that we can format them fancily. xspace
% inserts spaces when needed based on context.
\usepackage{xspace}

\newcommand*{\fotex}{F%
\kern -.25em%
{\raisebox{-.215em}{Ø}}%  % -.215em makes 2 Es match
\kern -.25em%
\TeX\xspace}

% Better references, automatically uses Danish names. Use \cref
% instead of \autoref. See package documentation for more details.
%
% nameinlink makes the prefix (e.g. "kapitel") a part of the hyperlink
% (as \autoref does). This might not be a good idea if multiple labels
% are used in a \cref.
\usepackage[nameinlink]{cleveref}


\title{Pantvagt}
\date{\today}
\author{Casper Freksen}

\begin{document}

\maketitle

\section{Oversigt}
\label{sec:oversigt}

Pantvagt udføres en gang om ugen, og handler hovedsageligt om, at
komme af med noget af vores pant, samt indkøbe noget af vores
sortiment.

Det pant, der skal indkasseres i Føtex er groft sagt, det pant, vi
opbevarer i kasser. Dvs. normale øl (Top, Carlsberg, \dots), normale
sodavand (Coca Cola, \dots), samt indholdet i de hvide/klare
kasser. Derimod er sørger pantvagten ikke for panten i pantsækkene
eller pant for fustager. Al panten burde ligge i Rummet.

Den del af sortimentet, der skal indkøbes, er:
\begin{itemize}
\item Normale øl
\item Normale sodavand
\item Chips
\item Sprut
\item Shotglas
\end{itemize}
Detaljer om, hvor meget der skal købes findes på sedler: Bl.a. under
trappen og bag gitteret i Nygaard.\todo{Revider sedlen og gengiv den
  her}

Disse ting købes i Føtex med købekortet, der ligger på en hylde i
Rummet over for Rummet.

Vi plejer gerne at ville købe flaskesodavand, men det er ikke altid,
at Føtex har dem. Derfor, hvis Føtex har dem, så køb gerne ekstra; de
skal nok blive drukket.

\section{Rækkefølge}
\label{sec:rakkefolge}

Her følger en rækkefølge, man kan følge\footnote{Rækker det til en MatAnal joke?}:

\begin{enumerate}
\item Start i Nygaard-kælderen, og se hvor meget der mangler af øl,
  sodavand, osv.
\item Gå op i Ada-1. I Babbage-enden, lige ved siden af
  studentermedhjælperkontoret\footnote{Så vidt jeg husker}, plejer der
  at være en indkøbsvogn, man kan låne.
\item Gå ned i Ada-0 med indkøbsvognen, og fyld pant fra Rummet op i
  den.
\item Tag købekortet i Rummet over for Rummet.
\item Kør over i Føtex. Brug gerne rampen op til parkeringspladsen på
  taget, da der heroppe er en vognvenlig indgang.
\item Indløs pant i pantautomaten.
\item Shop-amok i Føtex.
\item Betal med købekort (og pantbon) ved kassen. Dette kræver, at du
  giver dem kortet og skriver under på et stykke papir
  kasseassistenten giver dig. Husk kassebon\footnote{og dyrekort, hvis
    du vil}!
\item Kør ned i Nygaard-kælderen, og læg nyindkøbte chips under
  trappen og læg nyindkøbte drikkevarer bag gitteret.
\item Kør over i Ada-0, og læg købekortet og hvide/klare kasser på
  plads. Kassebonnen lægges ved købekortet.
\item Aflever indkøbsvognen tilbage på Ada-1.
\end{enumerate}

En ekstra note omkring indkøbsvogne: Nogle gange er der ikke nogen
indkøbsvogn i Ada-1, eller det er sent på dagen og Ada-1 er lukket. I
disse tilfælde kan man hente en vogn ovre i Føtex. Herefter, kan man
enten køre tilbage til Ada-0 og fortsætte derfra, eller man kan først
købe ind og derefter indløse panten. Pantbonnen kan indløses til
kontanter i kiosken i Føtex (Husk bon for indløsning!). Disse
kontanter ligges ved købekortet til sidst\todo{Eller skal de lægges
  depositum-boksen?}.

Hvis man ikke kan komme ind i Ada-1 til sidst, kan man nogen gange
føle sig doven, og efterlade vognen i Rummet ved siden af Rummet, i
stedet for at køre den hele vejen tilbage til Føtex. Så kig evt. her,
hvis i mangler en vogn.

\section{Meta}
\label{sec:meta}
\todo{Find en bedre titel}

Man har pantvagt 4 uger i træk. De første 2 uger er med en, der er ved
at være færdig med sin pantvagt omgang, og i de sidste 2 uger er man
så selv den, der er ved at være færdig, og man er derfor sammen med
en, der skal til at begynde.

Man aftaler indbyrdes, hvornår man mødes. Det plejer at være en god
ide, at når en ny bliver pantvagt (altså hver anden uge), så
kontakter, den der er midt i sin pantvagtsomgang, den nye. Hvis man
ser, at der ikke er sat nogen på som pantvagt, bør man sende en mail
til best-listen (\email{best@fredagscafeen.dk}) eller på en anden måde
give besked til scheduler/pantvagtplansansvarlige.

Dette system burde sørge for, at der altid er en, der har mindst 2
ugers erfaring med pantvagten, og som ved hvordan status på lagre
sådan cirka er (f.eks. om der er en indkøbsvogn fra sidste pantvagt).

\end{document}

%%% Local Variables:
%%% mode: latex
%%% TeX-master: t
%%% End:
