% !TeX spellcheck = da_DK
\documentclass[danish]{article}
% Ability to write input files using utf8
\usepackage[utf8]{inputenc}

% Proper font, æ, ø and å becomes copy/paste and searchable
\usepackage[T1]{fontenc}
\usepackage{lmodern}

% Enable \includegraphics, so that images can be included
\usepackage{graphicx}
\DeclareGraphicsExtensions{.pdf, .png, .jpg, .PDF, .PNG, .JPG}

% Enables use of links, and adds ToC for your PDF-reader
\usepackage{hyperref}
% A small macro for inserting clickable email adresses, e.g.,
% \email{best@fredagscafeen.dk}
\newcommand*{\email}[1]{\href{mailto:#1}{\nolinkurl{#1}}}

% Adding visible TODO's using \todo
\usepackage[obeyFinal]{todonotes}

% Better kerning etc.
\usepackage{microtype}

% Hyphenation for Danish words etc.
\usepackage{babel}

% For conditionals in commands
\usepackage{xifthen}

% Macro for Question and answers, troubleshooting etc.
% The optional argument is a label to ref to. The label will be prefixed with "qna:"
\newcommand{\QnA}[3][]{
\paragraph{#2}
% Insert label if it is provided in the optional argument
\ifthenelse{\isempty{#1}}
  {}
  {\label{qna:#1}}
  \begin{itemize}
    #3
  \end{itemize}
}

% Macros for common words, so that we can format them fancily. xspace
% inserts spaces when needed based on context.
\usepackage{xspace}

\newcommand*{\fotex}{F%
\kern -.25em%
{\raisebox{-.215em}{Ø}}%  % -.215em makes 2 Es match
\kern -.25em%
\TeX\xspace}

% Better references, automatically uses Danish names. Use \cref
% instead of \autoref. See package documentation for more details.
%
% nameinlink makes the prefix (e.g. "kapitel") a part of the hyperlink
% (as \autoref does). This might not be a good idea if multiple labels
% are used in a \cref.
\usepackage[nameinlink]{cleveref}


\title{Formandsguide}
\date{\formatdate{22}{3}{2025}}
\author{Casper Freksen\\
\small med tilføjelser af Oskar Haarklou Veileborg
\& Anders Bruun Severinsen}

\begin{document}

\maketitle

\section{Introduktion}
\label{sec:introduktion}

Dette er en guide omkring formandsarbejdet i \fredagscafeen.

\section{Ting, der fast skal gøres}
\label{sec:ting-der-skal}

\begin{itemize}
    \item Hver måned skal der indkaldes til bestyrelsesmøde
    \item Senest en måned inden den finder sted, skal der indkaldes til
    generalforsamling.
    \item Der gives julegaver til portnerne og rengøringen.
    \item `Porter Boarding' kort efter tiltræden.
\end{itemize}

\subsection{Bestyrelsesmøde}
\label{sec:bestyrelsesmode}

Hver måned afholdes der bestyrelsesmøde i Fredagscaféen. Jeg plejer at
sende en Doodle ud et par uger før mødet bør holdes, hvor de forskellige medlemmer kan
vælge de datoer, de kan deltage. Send gerne en påmindelse en dag før doodlen
lukkes. Endeligt bør der sendes en påmindelse på dagen for mødet.

Der tages referat til bestyrelsesmøderne af en referent. Se
evt.~referaterne i det andet git repository.

Dagsordenen bygges op mellem mødeindkaldelse og mødet. Hvis folk er
villige til at bruge git, kan de bidrage der. Ellers kan man sende
mails med forslag til punkter.

Først bestilles der mad, så det kan nå at komme inden slutningen af
mødet. Samtidig køres det store anlæg ud, og det skylles med
lilla. Dernæst vælges dirigent og referent (statistisk set bliver
formanden valgt til mindst en af delene). Så behandles der punkter
ifølge dagsordenen.

\subsection{Generalforsamling}
\label{sec:generalforsamling}

Generalforsamlingen er der, hvor du blev valgt. Da du gerne vil
genvælges, eller i det mindste gerne vil have en efterfølger, skal der
indkaldes til generalforsamlingen. Generalforsamlingen plejer at ligge
i starten af marts, så send derfor indkaldelsen i starten af februar.

Indkaldelsen skal indeholde en dagsorden, der f.eks. kunne se således
ud:
\begin{itemize}
    \item Valg af dirigent og referent
    \item Beretning fra den afgående bestyrelse
    \item Beretning om foreningens regnskab
    \item Behandling af indkomne forslag
    \item Evaluering af foreningens arbejde
    \item Valg af formand
    \item Valg af kasserer
    \item Valg af resterende bestyrelse
    \item Valg af revisor
    \item Farvel til den gamle bestyrelse
    \item Eventuelt
\end{itemize}

Beretningen fra den afgående bestyrelse plejer formanden at tage sig
af. Forbered derfor et lille slideshow med billeder fra årets løb.

Der gives en øl til stemmeberettigede.

Ift. til PR, så skal der gerne laves en plakat inden GF. Ifølge
traditionen skal der være en generalhat på plakaten.

GF kan være en fin mulighed for at tage et fællesbillede af den nye
bestyrelse.

\subsubsection{Efter generalforsamlingen}

Efter generalforsamlingen er der et par ting der skal gøres:
\begin{itemize}
    \item Virk.dk
    \item E-boks.dk
    \item Nøglefordeling
    \item Adgang med studiekort
    \item Adgang og gennemgang af bankboksen
\end{itemize}


\subsection{Julegaver}
\label{sec:julegaver}

Vi plejer at give julegaver til portnerne og
rengøringen. Ift. portnerne har vi tidligere sponsoreret øl til deres
julefrokost (en fustage), men det var vidst lidt meget øl for dem, og
det er ikke blevet gjort de sidste par år. I stedet kan der blandes en
kurv med specialøl.

Til rengøring kunne der gives en æske god chokolade eller også nogle
blandede øl (hvis man ikke skal være så kønsdiskriminerende).

Endeligt kunne man overveje, om der er andre, vi gerne vil forkæle med
lidt ekstra op til jul.

\subsection{Porter Boarding}
\label{sec:porter-boarding}

`Porter Boarding' er navnet på et møde mellem den nuværende/nye
og ældre formænd, hvor der byttes viden for portere (eller andre mørke
øl).
Du kan f.eks.~invitere de ældre formænd til Porter Boarding en fredag i ugerne efter
generalforsamlingen.

Det kan anbefales at få nogle vise ord fra gamle formænd, og man har
mulighed for at komme med spørgsmål til, hvad man lige undrer sig
over. Derudover er det hyggeligt at snakke og få et par øl.


\section{Andet}
\label{sec:andet}

Andre ting er ikke så fastlagt, men stadig en del af formandsarbejdet.

Som formand er man ansigtet udadtil. Derfor kan der komme invitationer
til diverse møder med instituttet, man bør deltage i.
Derudover kan formanden være tovholder for at introduktionen af fredagsbaren for
russerne.

Det er også vigtigt man har et overblik over, hvor de andre i
bestyrelsen er i deres opgaver. Det er ikke meningen, man skal ånde
dem i nakken, men hvis en falder bagud, er det godt at vide det og
tage en snak med dem.

\end{document}

%%% Local Variables:
%%% mode: latex
%%% TeX-master: t
%%% End:
