\documentclass[danish]{article}
% Ability to write input files using utf8
\usepackage[utf8]{inputenc}

% Proper font, æ, ø and å becomes copy/paste and searchable
\usepackage[T1]{fontenc}
\usepackage{lmodern}

% Enable \includegraphics, so that images can be included
\usepackage{graphicx}
\DeclareGraphicsExtensions{.pdf, .png, .jpg, .PDF, .PNG, .JPG}

% Enables use of links, and adds ToC for your PDF-reader
\usepackage{hyperref}
% A small macro for inserting clickable email adresses, e.g.,
% \email{best@fredagscafeen.dk}
\newcommand*{\email}[1]{\href{mailto:#1}{\nolinkurl{#1}}}

% Adding visible TODO's using \todo
\usepackage[obeyFinal]{todonotes}

% Better kerning etc.
\usepackage{microtype}

% Hyphenation for Danish words etc.
\usepackage{babel}

% For conditionals in commands
\usepackage{xifthen}

% Macro for Question and answers, troubleshooting etc.
% The optional argument is a label to ref to. The label will be prefixed with "qna:"
\newcommand{\QnA}[3][]{
\paragraph{#2}
% Insert label if it is provided in the optional argument
\ifthenelse{\isempty{#1}}
  {}
  {\label{qna:#1}}
  \begin{itemize}
    #3
  \end{itemize}
}

% Macros for common words, so that we can format them fancily. xspace
% inserts spaces when needed based on context.
\usepackage{xspace}

\newcommand*{\fotex}{F%
\kern -.25em%
{\raisebox{-.215em}{Ø}}%  % -.215em makes 2 Es match
\kern -.25em%
\TeX\xspace}

% Better references, automatically uses Danish names. Use \cref
% instead of \autoref. See package documentation for more details.
%
% nameinlink makes the prefix (e.g. "kapitel") a part of the hyperlink
% (as \autoref does). This might not be a good idea if multiple labels
% are used in a \cref.
\usepackage[nameinlink]{cleveref}


\title{Bestilling af fadøl}
\date{\today}
\author{Frederik B. Truelsen}

\begin{document}

\maketitle

\section{Lageret}

Lageret skal indeholde følgende inden en barvagt.

\begin{itemize}
\item 15 EP (pilsner)
\item 1 ks. krus
\item 2 flasker med CO$_{2}$
\item 1 - 2 specialøl
\end{itemize} 

\subsection{Lager orden}

Ved optælling af lageret er det en rigtig god ide at flytte eventuelt resterende øl frem tættest på døren.
Dette tjener to formål, vi sikre at bartenderne tager det ældste øl først og giver plads til at Aarhus Bryghus
sætter øllene ind bagerest i lokalet.

\subsection{Specialtilfælde}

Overstående lager status dækker det almindelig ugentlige behov for at drive Fredagscafeén,
men i tilfælde af andre events med udlejning af vores lille anlæg, specielle events i baren og
i forbindelse med lang barvagt skal der være ekstra øl.

\subsubsection*{Udlejning}

I tilfælde af en udlejning skal man bestille det ekstra som der er behov for i forbindelse med
udlejningen. I forbindelse store udlejninger som f.eks. Secoya og Institut for Datalogi og andre
festlige lidt større arrangementer  kan det være en god ide at have lidt ekstra fustager i baghånden.

\subsection*{Events}

Ved events såsom lange barvagter, sponsorbarer og andre events hvor firmaer udefra kommer forbi skal der forventes
at man skal have et lager på omkring 20 - 25 fustager øl. Man kan følgende uge lave en modsvarende mindre bestilling,
man behøver ikke bestille øl kan også bestille en tømning af tomme fustager.

\subsection*{Ferie}

I sommer- og juleferien er der meget stille og man skal være opmærksom på at en normal bestilling
godt kan holde flere uger, det variere meget gennem ferien og første del af ferien og
midt august ligner mere normalen dog stadig med forbehold for en pæn variation.


\section{Bestilling}

Bestilling foregår ved at man sender en mail til Niels Buchwald på \\
\href{mailto:nfb@aarhusbryghus.dk}{nfb@aarhusbryghus.dk} og specificere hvor mange tomme fustager
og CO$_{2}$ flasker vi sender retur, samt hvor meget vi ønsker at bestille svarende til at genoprette
ovenstående lager status. Aarhus Bryghus har selv en nøgle og står selv for det praktiske omkring leveringen

\subsection{Holdbarhed}

I forbindelse med store bestillinger og ferier kan det forekomme at nogle af øllene kommer til at løbe over dato.
Aarhus Bryghus kører med måned fra øllen er produceret hvilket også svarer til tæt på den dag de blev leveret.
Vi har ikke haft problemer med at sælge øllen op til en måneds tid efter den dato mærkning, men smag på øllen
om den smager rigtig. Kommer øllen tæt på at gå over den udvidet periode sælges den billigere,
eventuelt omkring købspris afrundet til nærmeste tal deleligt med fem.
Retur foregår ved at man sætter en form for mærkning på dem at de skal retur, og de stilles sammen
med de tomme fustager.

\end{document}