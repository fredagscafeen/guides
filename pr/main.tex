% !TeX spellcheck = da_DK
\documentclass{article}
% Ability to write input files using utf8
\usepackage[utf8]{inputenc}

% Proper font, æ, ø and å becomes copy/paste and searchable
\usepackage[T1]{fontenc}
\usepackage{lmodern}

% Enable \includegraphics, so that images can be included
\usepackage{graphicx}
\DeclareGraphicsExtensions{.pdf, .png, .jpg, .PDF, .PNG, .JPG}

% Enables use of links, and adds ToC for your PDF-reader
\usepackage{hyperref}
% A small macro for inserting clickable email adresses, e.g.,
% \email{best@fredagscafeen.dk}
\newcommand*{\email}[1]{\href{mailto:#1}{\nolinkurl{#1}}}

% Adding visible TODO's using \todo
\usepackage[obeyFinal]{todonotes}

% Better kerning etc.
\usepackage{microtype}

% To create and control lists (itemize, enumeration, description)
\usepackage{enumitem}

% A new list environment that has both numbers (as enumeration) and
% labels (as description)
\newcounter{enumdescriptioncount}
\newlist{enumdescription}{description}{1}        % 1 means that it cannot be nested
\setlist[enumdescription]{%
  before = {\setcounter{enumdescriptioncount}{0}%
            \renewcommand*{\theenumdescriptioncount}{\arabic{enumdescriptioncount}}},
  font = {\bfseries\stepcounter{enumdescriptioncount}{\large \theenumdescriptioncount.}~},
  align = right,                % Makes the labels end, at the same position
  itemindent = 6em              % TODO: This can be done better
}

% Hyphenation for Danish words etc.
\usepackage[danish]{babel}

% Macros for common words, so that we can format them fancily. xspace
% inserts spaces when needed based on context.
\usepackage{xspace}

\newcommand*{\fotex}{F%
\kern -.25em%
{\raisebox{-.215em}{Ø}}%  % -.215em makes 2 Es match
\kern -.25em%
\TeX\xspace}


\title{Propaganda for Fredagscaféen}
\date{\formatdate{20}{2}{2026}}

\begin{document}

\maketitle

\section{Administration af Facebook og Instagram}
Som PR, skal du have adgang til vores \href{https://www.facebook.com/fredagscafeen.dk/}{Facebook} og \href{https://www.instagram.com/fredagscafeen.dk/}{Instagram}. Vær opmærksom på at adgang gennem Meta's Business Suite er praktisk, men nogle features er ikke tilgængelige her - Ergo du skal også logge ind på selve appen. Det er PR's opgave at lave materiale og posts der skal ligge på vores SoMe, samt at besvare alle meddelelser vi får igennem Facebook siden, og, hvis nødvendigt, videreformidle det til \bestmail.\\
\\
På Facebook er coverbilledet alle vores events for det kommende semester. Det er vigtigt, at huske at opdatere denne løbende, i samarbejde med den eventansvarlige.

\section{Begivenheder}
PR skal sørge for at annoncere events og andre nyheder cirka lørdagen før Fredagscaféen (d.v.s. lige efter den Fredagscafé der ligger før). Dette er fordi at vi gerne vil undgå at annoncere det flere fredage i forvejen, så folk bliver forvirrede over hvilken fredag der er tale om.\\
Dette kan godt ændres lidt på hvis der er tale om en større begivenhed. I denne situation er det bedst at oprette begivenheden, og så lægge et plakatopslag op lørdagen før begivenheden, så folk husker det kommer
\\
Hvis begivenheden er en sponsorbar, laves der ikke et FB-begivenhed, men et-to opslag om det på Meta.

\section{Plakater og opslag på SoMe}
Op til events må der meget gerne laves plakater. Der er eksempler på disse plakater på Gitlab (\url{https://gitlab.au.dk/fredagscafeen/propaganda}). Husk ikke at benytte AI til plakater, da Meta kan vælge at shadowbanne dig for det. Desuden, da vi har en business account, kan musik på videoer være præget af ophavsret, så undgå at vælg en super populær pop banger hvis det er alfa og omega at der indgår musik i videoen.\\
En anden ting man lige skal være opmærksom på er stock billeder. Hvis ikke vi har købt rettigheder til billeder, risikerer vi en kæmpe bøde, så prøv at benytte billeder fra baren (\url{https://gitlab.au.dk/fredagscafeen/billedarkiv})\\
\\
Der er primært brugt Adobe suiten til vores plakater, men dette kræver et abonnement/andre måder at opnå adgang til programmerne. Lykkedes du ikke med det, kan man snakke med instituttet (den der er ansvarlig for csaudk/ Gøttsche) om man kan være en del af deres Canva abonnement.\\
\\
Til GF er det meget vigtigt at der indgår eksplosioner, formand og logo med general-hat.

\section{Engagement på SoMe}
Et opslag med et generic-looking billede og lidt tekst klarer sig æ' godt. Vores følgerskare engagerer sig mest med det humoristiske eller relaterbare/indsigt BTS. Hvad fanden betyder det så for hvad der skal postes? Alt efter budskabet, så klarer memes (gerne med egne sjove Djurs Sommerland billeder el. andet) sig virkelig godt. Vores følgere kan også godt lide at få et indblik i hvad vi laver, så sjove billeder fra lageret eller andet gøgl bliver belønnet. Karouseller (flere billeder) fungerer ikke så godt til promovering af øl, de skal helst være på et billede.\\
\\
Når det kommer til plakater, så er der lidt mere fri leg - Alt udover en hvid side med sort tekst klarer sig decent. Prøv at leg med farverne eller klip nogle billeder ud og smæk på :)
\\
Reels har et højt visningsantal, men desværre mest til ikke-følgere der aldrig kommer til at sætte en fod i datbar. Hellere smid en story eller to op, de retter sig mere til allerede følgere

\section{Bestilling af T-Shirts (og andet tøj)}
Af og til skal der bestilles T-Shirts. Dette sker ret sjældent, men når det skal gøres, plejer vi at gøre det igennem Customize plus Simonsen (\url{https://c-ps.dk}).\\
De er super søde og samarbejdsvillige, giver nogle gange lidt rabat på tryk mm.\\
\\
Der ligger en mere udtrykkelig guide til hvordan man bestiller T-shirts på GitLab. Husk at komme igang lige efter GF, så DatBest ikke venter på nye T-shirts i et halvt år xD.\\
\\
Vi har som noget nyt i best 25/26 valgt også at få trykt nogle hoodies/trøjer. Dette er noget man kan kigge på, evt. også udbyde til bartendere mm, så de ikke fryser på deres barvagt.

\section{Tryk af plakater mm.}
Hvis noget grafisk (på papir) skal trykkes i bedre kvalitet end en printer kan klare, så er det billigste at tage ned på Grafisk Værksted på Godsbanen. Hvis det skal være større end A3 skal man sende dem en mail, alt information ang. dette findes på deres hjemmeside.

\section{Slikposer og diverse}
Nogle gange gør vi skøre ting som at bestille 3500 slikposer. Disse slikposer bestilte vi igennem Presenta, dem har vi haft gode erfaringer med (men husk at få regningen betalt!).

Generelt, hvis der skal købes noget ekstra-ordinært, så kontakt kasseren eller resten af bestyrelsen og få det godkendt, derefter kan det blive købt fra enhver passende leverandør (men prøv at undgå interesse-konflikter I guess).

\end{document}
