% !TeX spellcheck = da_DK
\documentclass{article}
% Ability to write input files using utf8
\usepackage[utf8]{inputenc}

% Proper font, æ, ø and å becomes copy/paste and searchable
\usepackage[T1]{fontenc}
\usepackage{lmodern}

% Enable \includegraphics, so that images can be included
\usepackage{graphicx}
\DeclareGraphicsExtensions{.pdf, .png, .jpg, .PDF, .PNG, .JPG}

% Enables use of links, and adds ToC for your PDF-reader
\usepackage{hyperref}
% A small macro for inserting clickable email adresses, e.g.,
% \email{best@fredagscafeen.dk}
\newcommand*{\email}[1]{\href{mailto:#1}{\nolinkurl{#1}}}

% Adding visible TODO's using \todo
\usepackage[obeyFinal]{todonotes}

% Better kerning etc.
\usepackage{microtype}

% To create and control lists (itemize, enumeration, description)
\usepackage{enumitem}

% A new list environment that has both numbers (as enumeration) and
% labels (as description)
\newcounter{enumdescriptioncount}
\newlist{enumdescription}{description}{1}        % 1 means that it cannot be nested
\setlist[enumdescription]{%
  before = {\setcounter{enumdescriptioncount}{0}%
            \renewcommand*{\theenumdescriptioncount}{\arabic{enumdescriptioncount}}},
  font = {\bfseries\stepcounter{enumdescriptioncount}{\large \theenumdescriptioncount.}~},
  align = right,                % Makes the labels end, at the same position
  itemindent = 6em              % TODO: This can be done better
}

% Hyphenation for Danish words etc.
\usepackage[danish]{babel}

% Macros for common words, so that we can format them fancily. xspace
% inserts spaces when needed based on context.
\usepackage{xspace}

\newcommand*{\fotex}{F%
\kern -.25em%
{\raisebox{-.215em}{Ø}}%  % -.215em makes 2 Es match
\kern -.25em%
\TeX\xspace}


\title{Propaganda for Fredagscaféen}
\date{\formatdate{8}{3}{2020}}

\begin{document}

\maketitle

\section{Administration af Facebook siden}
Som PR, skal du have adgang til Facebook siden \url{https://www.facebook.com/fredagscafeen.dk/}. Det er PR's opgave at lave materiale og posts der skal ligges på Facebook siden, samt at besvare alle meddelelser vi får igennem Facebook siden og, hvis nødvendigt, videreformidle det til \bestmail.

PR skal sørge for at annoncere events og andre nyheder cirka lørdagen før Fredagscaféen (d.v.s. lige efter den Fredagscafé der ligger før). Dette er fordi at vi gerne vil undgå at annoncere det flere fredage i forvejen, så folk bliver forvirrede over hvilken fredag der er tale om.

\section{Bestilling af T-Shirts}
Af og til skal der bestilles T-Shirts. Dette sker ret sjældent, men når det skal gøres, plejer vi at gøre det igennem Werk, som også trykker T-Shirts for MatFys-Tutorforening og TK.

Werk kan lave mange andre tryk-sager, så hvis der er noget der skal til tryk, er det hos Werk.

Vi har tidligere været i kontakt med Benedikte (\url{benediktewerk@gmail.com}) når vi har skulle komme i kontakt med Werk.

\section{Slikposer og diverse}
Nogle gange gør vi skøre ting som at bestille 3500 slikposer. Disse slikposer bestilte vi igennem Presenta, dem har vi haft gode erfaringer med (men husk at få regningen betalt!).

Generelt, hvis der skal købes noget ekstra-ordinært, så kontakt kasseren eller resten af bestyrelsen og få det godkendt, derefter kan det blive købt fra enhver passende leverandør (men prøv at undgå interesse-konflikter I guess).

\section{Plakater}
Op til events må der meget gerne laves plakater. Der er materiale til disse plakater på AU's Gitlab (\url{https://gitlab.au.dk/fredagscafeen/propaganda}). Det er et værre rod pt. så der må meget gerne ryddes op!

Til GF er det meget vigtigt at der indgår eksplosioner, formand og logo med general-hat.

\end{document}
