% !TeX spellcheck = da_DK
\documentclass[danish]{article}
% Ability to write input files using utf8
\usepackage[utf8]{inputenc}

% Proper font, æ, ø and å becomes copy/paste and searchable
\usepackage[T1]{fontenc}
\usepackage{lmodern}

% Enable \includegraphics, so that images can be included
\usepackage{graphicx}
\DeclareGraphicsExtensions{.pdf, .png, .jpg, .PDF, .PNG, .JPG}

% Enables use of links, and adds ToC for your PDF-reader
\usepackage{hyperref}
% A small macro for inserting clickable email adresses, e.g.,
% \email{best@fredagscafeen.dk}
\newcommand*{\email}[1]{\href{mailto:#1}{\nolinkurl{#1}}}

% Adding visible TODO's using \todo
\usepackage[obeyFinal]{todonotes}

% Better kerning etc.
\usepackage{microtype}

% Hyphenation for Danish words etc.
\usepackage{babel}

% For conditionals in commands
\usepackage{xifthen}

% Macro for Question and answers, troubleshooting etc.
% The optional argument is a label to ref to. The label will be prefixed with "qna:"
\newcommand{\QnA}[3][]{
\paragraph{#2}
% Insert label if it is provided in the optional argument
\ifthenelse{\isempty{#1}}
  {}
  {\label{qna:#1}}
  \begin{itemize}
    #3
  \end{itemize}
}

% Macros for common words, so that we can format them fancily. xspace
% inserts spaces when needed based on context.
\usepackage{xspace}

\newcommand*{\fotex}{F%
\kern -.25em%
{\raisebox{-.215em}{Ø}}%  % -.215em makes 2 Es match
\kern -.25em%
\TeX\xspace}

% Better references, automatically uses Danish names. Use \cref
% instead of \autoref. See package documentation for more details.
%
% nameinlink makes the prefix (e.g. "kapitel") a part of the hyperlink
% (as \autoref does). This might not be a good idea if multiple labels
% are used in a \cref.
\usepackage[nameinlink]{cleveref}


\title{Kasserer Assistent Guide}
\author{Søren Hjort}
\date{February 2020}

\begin{document}

\maketitle

\section{Introduction}
Denne guide vil forklare Kassererassistentets ansvarområder.
\\
Den primære opgave er at opdatere krydslisten og optælle kontanter mellem hver
barvagt.
\\
Den primære opgave er at have styr på krydslisten og kontanter hver uge.
\\
Da man er en af de få, som har en nøgle til pengeskabet, kan man hjælpe til med at
skaffe flere penge under en barvagt, hvis der er behov.
\\
Denne post har også stået for krus, men det er stoppet efter vi mistede vores
leverandør, men hvis man har lyst, kan man prøve at finde en ny. 

\section{Krydsliste}
Hver uge skal krydslisten opdateres. Dette gøres ved at skrive alle ændringer
ind på vores admin side ved ``Add Snapshot''.
\\
Her indskrives alle ændringer: en kunde har brugt penge og/eller indsat penge og
der kan være oprettet nye brugere.
\\
Det er en god ide at tjekke igennem at man ikke har lavet nogle fejl.
\\
Man skal sørge for at printe den opdatere liste ud og lægge den i penge-rummet. 
\\
Krydslisterne arkiveres, sammen med vagten dankort-automat afstemning og pizza
kvittering i rummet under studiecafeen (i Bush).

\section{Kontanter}
Hver uge skal der tælles hvor mange kontanter er tilbage, og der skal lægges
nye i, så det passer igen.
Der skal i starten af en barvagt være 1000 kr + eventuelle 50 ører.
(3 * 100 kr.  + 5 * 50 kr. + 10 * 20 kr. + 15 * 10 kr.
\\+ 8 * 5 kr. + 20 * 2 kr. + 20 * 1 kr. = 1000 kr.)
Dette kan ses på en seddel, som ligger i den store pengekasse.
\\
Derudover holdes der også styr på hvor mange kontanter vi har liggende, i den
store pengekasse, og andetsteds.
\\
I tilfælde af mangel på kontanter, er det dit ansvar at få formanden eller
kasseren til at hæve flere (det er dem der har adgang til bankkontoen).
\\
Det er smart at have dokument, hvor man opskriver dette, på fredagscafeens
google drive kan se fra tidligere år, kaldet ``KassererAssistentDoc.''

\section{Bogføring}
Man skal bogføre ``dagens salg'', dette indebærer hvor meget salg, der sker på
hver vagt:
Antal Kontanter i lille/store pengekasser før og efter barvagt,
indsat på krydslisten, brugt af krydslisten, kontante
udgifter (pizza), indtægt fra dankort automat.
Disse lægges på fredagscafeens google drive under ``dagens salg'', her kan man
også finde skabelonen, og dokumenter fra tidligere år.
\\
Derudover skal der også tages billeder af afstemninger fra dankort-automat og
pizza-kvitteringer. Disse lægges også op på fredagscafeens google drive.

\section{GeneralForsamling}
Til den årlige generalforsamlingen har kassererassistenten et punkt, hvor man gennemgår top 10 forbrugere på krydslisten
(Dette kan ses under "count consumption på adminsiden"), og overrækker præmier til top 3.
\\
Det er sjovt at lave en meme for hver person i top 10.
\\
Præmierne kunne være:
\\
3. plads - én Top
\\
2. plads - én specialøl
\\
1. plads - én flaske sprut (/ultrapres)

\end{document}

%%% Local Variables:
%%% mode: latex
%%% TeX-master: t
%%% End: