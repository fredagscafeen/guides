% !TeX spellcheck = da_DK
\documentclass[danish]{article}
% Ability to write input files using utf8
\usepackage[utf8]{inputenc}

% Proper font, æ, ø and å becomes copy/paste and searchable
\usepackage[T1]{fontenc}
\usepackage{lmodern}

% Enable \includegraphics, so that images can be included
\usepackage{graphicx}
\DeclareGraphicsExtensions{.pdf, .png, .jpg, .PDF, .PNG, .JPG}

% Enables use of links, and adds ToC for your PDF-reader
\usepackage{hyperref}
% A small macro for inserting clickable email adresses, e.g.,
% \email{best@fredagscafeen.dk}
\newcommand*{\email}[1]{\href{mailto:#1}{\nolinkurl{#1}}}

% Adding visible TODO's using \todo
\usepackage[obeyFinal]{todonotes}

% Better kerning etc.
\usepackage{microtype}

% Hyphenation for Danish words etc.
\usepackage{babel}

% For conditionals in commands
\usepackage{xifthen}

% Macro for Question and answers, troubleshooting etc.
% The optional argument is a label to ref to. The label will be prefixed with "qna:"
\newcommand{\QnA}[3][]{
\paragraph{#2}
% Insert label if it is provided in the optional argument
\ifthenelse{\isempty{#1}}
  {}
  {\label{qna:#1}}
  \begin{itemize}
    #3
  \end{itemize}
}

% Macros for common words, so that we can format them fancily. xspace
% inserts spaces when needed based on context.
\usepackage{xspace}

\newcommand*{\fotex}{F%
\kern -.25em%
{\raisebox{-.215em}{Ø}}%  % -.215em makes 2 Es match
\kern -.25em%
\TeX\xspace}

% Better references, automatically uses Danish names. Use \cref
% instead of \autoref. See package documentation for more details.
%
% nameinlink makes the prefix (e.g. "kapitel") a part of the hyperlink
% (as \autoref does). This might not be a good idea if multiple labels
% are used in a \cref.
\usepackage[nameinlink]{cleveref}


\title{Dansk Retur System}
\date{\today}
\author{Jacob Schwartz Sørensen}

\begin{document}

\maketitle

\newcommand{\DRS}{Dansk Retur System }

\section{Introduktion}
\label{sec:Introduktion}

\DRS henter den pant vi har i pantsækkene i rummet. Dette gør de når vi ringer efter dem, og det er som udgangspunkt Panterens ansvar.
Normalt er det nok at tage pant 2-3 gange årligt, men hold øje med rummet, pant vokser eksponentiel :O
Er du ny Panter(held og lykke) så se nederste afsnit så den gamle Panter ikke får underlige opkald fra \DRS.

\section{Hvordan}
\label{sec:Hvordan}

Når det er tid til at \DRS skal hente pant, ordnes det på følgende måde:

\begin{enumerate}
	\item Tæl alle de små pantsække (dem med flasker).
	\item Tæl alle de store pantsække (dem med dåser).
	\item Book en afhæntning på \DRS hjemmeside: \url{https://www.dansk-retursystem.dk/kundeservice/book-afhentning/}
	\begin{itemize}
		\item \textbf{Firmanavn:} Fredagscafeen
		\item \textbf{Adresse:} Åbogade 34, 8200 Aarhus N
		\item \textbf{Kundenummer:} 5790001782870
		\item HUSK AT RETTE KONTAKTPERSONEN FØRST HVIS DU ER NY PANTER se nedenfor
	\end{itemize}
	\item Dansk retursystem kommer på et tidspunkt og henter det bestilte, de ringer når de kommer, vær på uni den dag.
	\item \dots
	\item Profit
\end{enumerate}

\section{Er du ny panter?}
\label{sec:Er du ny Panter?}

Hvis du er ny Panter skal du, inden din første bestilling, ringe og opdaterer kontaktoplysninger på kontaktpersonen.
Dette gøres ved at ringe til \DRS kundeservice, og sørge for at følgene oplysninger er som anført her:

\begin{itemize}
	\item \textbf{Kontaktperson:} \textit{Dit navn her}
	\item \textbf{Telefonnummer:} \textit{Dit telefonnummer her}
	\item \textbf{E-mail (Hvis de spørger):} best@fredagscafeen.dk
\end{itemize}

De skal muligvis bruge vores kundenummer for at opdatere oplysningerne, dette er \textbf{5790001782870}.

\end{document}

%%% Local Variables:
%%% mode: latex
%%% TeX-master: t
%%% End:
