\documentclass[danish]{article}
% Ability to write input files using utf8
\usepackage[utf8]{inputenc}

% Proper font, æ, ø and å becomes copy/paste and searchable
\usepackage[T1]{fontenc}
\usepackage{lmodern}

% Enable \includegraphics, so that images can be included
\usepackage{graphicx}
\DeclareGraphicsExtensions{.pdf, .png, .jpg, .PDF, .PNG, .JPG}

% Enables use of links, and adds ToC for your PDF-reader
\usepackage{hyperref}
% A small macro for inserting clickable email adresses, e.g.,
% \email{best@fredagscafeen.dk}
\newcommand*{\email}[1]{\href{mailto:#1}{\nolinkurl{#1}}}

% Adding visible TODO's using \todo
\usepackage[obeyFinal]{todonotes}

% Better kerning etc.
\usepackage{microtype}

% Hyphenation for Danish words etc.
\usepackage{babel}

% For conditionals in commands
\usepackage{xifthen}

% Macro for Question and answers, troubleshooting etc.
% The optional argument is a label to ref to. The label will be prefixed with "qna:"
\newcommand{\QnA}[3][]{
\paragraph{#2}
% Insert label if it is provided in the optional argument
\ifthenelse{\isempty{#1}}
  {}
  {\label{qna:#1}}
  \begin{itemize}
    #3
  \end{itemize}
}

% Macros for common words, so that we can format them fancily. xspace
% inserts spaces when needed based on context.
\usepackage{xspace}

\newcommand*{\fotex}{F%
\kern -.25em%
{\raisebox{-.215em}{Ø}}%  % -.215em makes 2 Es match
\kern -.25em%
\TeX\xspace}

% Better references, automatically uses Danish names. Use \cref
% instead of \autoref. See package documentation for more details.
%
% nameinlink makes the prefix (e.g. "kapitel") a part of the hyperlink
% (as \autoref does). This might not be a good idea if multiple labels
% are used in a \cref.
\usepackage[nameinlink]{cleveref}


\title{Udlejning af Fredagscaféens ting}

\begin{document}

\maketitle

\section{Udlejning af fadølsanlæg}

Der er to typer af udlejning, \textbf{opstilling} og \textbf{afhentning}.

\subsection{Afhentning}
Den simpleste udlejning. På wiki-siden for udlejning, kan man se hvem der lejer anlægget, og hvornår det skal lejes. Ved starten af udlejningen, skal man mødes med lejer ved rummet ved siden af rummet og overrække dem fadølsanlæg og fustager.

Når anlægget skal afleveres tilbage, så skal man tjekke at det er blevet gjort ordentligt rent og, afhængig af aftale med lejer, skal man rense anlægget og sætte det på plads.

D.v.s. der skal blot låses op for adgang til anlægget derefter skal det renses og låses i.

\subsection{Opstilling}
Lidt mere besværlig, her skal bestyrelsen selv tage anlægget til lokation for event og man skal afhente det der igen. Det vil aldrig ske andre steder end på instituttet så det er kun en lille tur man skal tage det.

I visse tilfælde, skal man også give en kort guide til hvordan man skifter fustage.

\section{Udlejning af grill}
Her vil der typisk være tale om \textbf{opstilling}, og det indebærer at man skal hjælpe med at tænde for grillen osv.

Mere info om hvordan man gør dette senere.

\section{Udlejning af diverse}
Her vil der udelukkende være tale om \textbf{afhentning} af den givne dims, der kan være en højtaler, en mixer, et kabel eller andet.

\end{document}
