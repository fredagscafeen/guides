\documentclass{article}
% Ability to write input files using utf8
\usepackage[utf8]{inputenc}

% Proper font, æ, ø and å becomes copy/paste and searchable
\usepackage[T1]{fontenc}
\usepackage{lmodern}

% Enable \includegraphics, so that images can be included
\usepackage{graphicx}
\DeclareGraphicsExtensions{.pdf, .png, .jpg, .PDF, .PNG, .JPG}

% Enables use of links, and adds ToC for your PDF-reader
\usepackage{hyperref}
% A small macro for inserting clickable email adresses, e.g.,
% \email{best@fredagscafeen.dk}
\newcommand*{\email}[1]{\href{mailto:#1}{\nolinkurl{#1}}}

% Adding visible TODO's using \todo
\usepackage[obeyFinal]{todonotes}

% Better kerning etc.
\usepackage{microtype}

% To create and control lists (itemize, enumeration, description)
\usepackage{enumitem}

% A new list environment that has both numbers (as enumeration) and
% labels (as description)
\newcounter{enumdescriptioncount}
\newlist{enumdescription}{description}{1}        % 1 means that it cannot be nested
\setlist[enumdescription]{%
  before = {\setcounter{enumdescriptioncount}{0}%
            \renewcommand*{\theenumdescriptioncount}{\arabic{enumdescriptioncount}}},
  font = {\bfseries\stepcounter{enumdescriptioncount}{\large \theenumdescriptioncount.}~},
  align = left,                 % Makes the labels start at the same position
}

% Be able to have multiple columns in guides and lists
\usepackage{multicol}

% Hyphenation for Danish words etc.
\usepackage[danish]{babel}

% Macros for common words, so that we can format them fancily. xspace
% inserts spaces when needed based on context.
\usepackage{xspace}

\newcommand*{\fotex}{F%
\kern -.25em%
{\raisebox{-.215em}{Ø}}%  % -.215em makes 2 Es match
\kern -.25em%
\TeX\xspace}
