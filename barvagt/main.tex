% !TeX spellcheck = da_DK
\documentclass[danish]{article}
% Ability to write input files using utf8
\usepackage[utf8]{inputenc}

% Proper font, æ, ø and å becomes copy/paste and searchable
\usepackage[T1]{fontenc}
\usepackage{lmodern}

% Enable \includegraphics, so that images can be included
\usepackage{graphicx}
\DeclareGraphicsExtensions{.pdf, .png, .jpg, .PDF, .PNG, .JPG}

% Enables use of links, and adds ToC for your PDF-reader
\usepackage{hyperref}
% A small macro for inserting clickable email adresses, e.g.,
% \email{best@fredagscafeen.dk}
\newcommand*{\email}[1]{\href{mailto:#1}{\nolinkurl{#1}}}

% Adding visible TODO's using \todo
\usepackage[obeyFinal]{todonotes}

% Better kerning etc.
\usepackage{microtype}

% Hyphenation for Danish words etc.
\usepackage{babel}

% For conditionals in commands
\usepackage{xifthen}

% Macro for Question and answers, troubleshooting etc.
% The optional argument is a label to ref to. The label will be prefixed with "qna:"
\newcommand{\QnA}[3][]{
\paragraph{#2}
% Insert label if it is provided in the optional argument
\ifthenelse{\isempty{#1}}
  {}
  {\label{qna:#1}}
  \begin{itemize}
    #3
  \end{itemize}
}

% Macros for common words, so that we can format them fancily. xspace
% inserts spaces when needed based on context.
\usepackage{xspace}

\newcommand*{\fotex}{F%
\kern -.25em%
{\raisebox{-.215em}{Ø}}%  % -.215em makes 2 Es match
\kern -.25em%
\TeX\xspace}

% Better references, automatically uses Danish names. Use \cref
% instead of \autoref. See package documentation for more details.
%
% nameinlink makes the prefix (e.g. "kapitel") a part of the hyperlink
% (as \autoref does). This might not be a good idea if multiple labels
% are used in a \cref.
\usepackage[nameinlink]{cleveref}


\title{Barvagt}
\date{\today}
\author{Casper Freksen}

\begin{document}

\maketitle

\tableofcontents

\vspace{0.4cm}
Originalt skrevet af Casper Freksen i 2016, revideret af Oskar Haarklou Veileborg i 2019.

\section{Før Barvagten}
\label{sec:pre-barvagten}

Vi mødes 14.30 ved gitteret. Baren åbner kl 15.00

Før barvagten, skal baren sættes op med alt hvad det indebærer. Her er
det sorteret efter, hvorfra ting skal hentes/ting skal ordnes.

Under barvagten har du som bartender adgang til en "bar-nøgle" denne nøgle ligger i baren og kan låse op til gitteret, rummet under trappen, tremmerummet i hopper og lageret i parkeringskælderen.

Baren lukker normalt kl. 22:00, men kan evt. lukkes tidligere, hvis der ikke er mange kunder.
4 gange om året lukker vi dog først kl 2.00, dette vil blive varslet i forvejen og der vil være to hold bartendere tilknyttet en sådan vagt.

Vask og sprit dine hænder af inden baren går i gang (sprit findes i baren).

\subsection{Bag ved Gitteret}
\label{sec:pre:bag-ved-gitteret}

\begin{itemize}
\item Baren skal køres hen til barområdet.
\item Kassen med spiritus skal hentes.
\item Køleskabene skal fyldes op med spændende øl o.lign.
  \begin{itemize}
  \item Sørg for at der er Tuborg, Carlsberg, Ceres Top og sodavand.
  \item Hav også gerne produkter som Høker bajer, Fynsk Forår, Limfjordsporter, Ale no. 16.
  \item Generelt: Hvis der er meget af det ude på lageret, så hav det med i køleskabet.
  \end{itemize}
\item Papkassen til dåser hentes.
\end{itemize}

\subsection{Ada 0}
\label{sec:pre:ada}

\begin{itemize}
\item Der skal hentes pengekasse og krydsliste i rummet ved siden af
  rummet. Gør dog gerne dette til sidst, så der ikke står en
  pengekasse i barområdet uden opsyn.
\end{itemize}

\subsection{Hopper}
\label{sec:pre:hopper}

\begin{itemize}
\item Der skal hentes fustager i parkeringskælderen. Normalt 3
  Extra Pilsner, 1 cider, 1 specialøl. Hvis der er masser af cider
  og/eller specialøl på anlægget i forvejen, så kan man i stedet hente
  en ekstra fustage Extra Pilsner. (Det handler bare om ikke at løbe
  tør).
\item Hvis der mangler pantposer til glasflasker kan disse findes i tremmerummet i Hopper 0.
\end{itemize}

\subsection{Under Trappen}
\label{sec:pre:under-trappen}

\begin{itemize}
\item Døren til under trappen skal låses op, så kunderne kan hente sine krus.
\item Brætspil skal hentes.
\item Chips skal hentes.
\item Plastkrus skal hentes (en håndfuld ruller, hvad det så end betyder).
\item Hvis det er den første fredag i måneden, så tag også lykkehjulet frem.
\end{itemize}

\subsection{Barområdet}
\label{sec:pre:baromradet}

\begin{itemize}
\item Baren skal sluttes til strøm og internet.
\item Fustager skal sluttes til, og der skal tændes for gassen:
  \begin{itemize}
  \item Pilsner sluttes til de yderste haner (1 fustage dækker begge
    haner).
  \item Cider sluttes til midt-venstre hane.
  \item Specialøl sluttes til midt-højre hane.
  \end{itemize}
\item Der sluttes strøm til de to køleskabe, og lyset i dem kan med
  fordel tændes.
\item De to barborde sættes op, og der sættes barstole op omkring dem.
\item Dankortmaskinen startes.
\item Der sættes sorte affaldssække op, hvis der ikke allerede er. Det
  er en god ide at have en skraldespand ved baren.
\item Der sættes en gennemsigtig pantsæk op til pant-A flasker. Disse
  sække plejer at være i baren, ved siden af køleren, ellers kan der
  ligge nogen under trappen.
\end{itemize}

\section{Under Barvagten}
\label{sec:intra-barvagten}

Under vagten er der diverse opgaver, der skal løses.

\subsection{Salg af varer}
\label{sec:intra:salg}

Der skal sælges diverse produkter
\begin{itemize}
\item Prisen kan enten slås op på en App, på
  \url{https://fredagscafeen.dk/prices/}, på \url{https://fredagscafeen.dk/search/} eller man kan spørge
  sine medbartendere.
\item Vi tager mod kontanter, kreditkort (Dankort, VISA, MasterCard
  m.fl.), og ``på listen''.
  \begin{itemize}
  \item Ved kreditkort trykkes OK, så indtastes beløbet på automaten, der
    trykkes OK, og kunden kan da indsætte kort osv.
    \begin{itemize}
    \item Husk at tjekke, at betalingen blev godkendt.
    \item Nogle gange kan det samme Dankort ikke bruges til det samme
      beløb, hvis betalingerne er for tæt på hinanden. Her kan beløbet
      så justeres en smule, eller kunden kan betale på en anden måde
      (kontant, liste).
    \item F.eks.\ hvis kunden betaler 5 kr.\ for en sodavand, og kort
      tid efter køber endnu en sodavand på sit Dankort, kan man nøjes
      med at tage 4,99 kr.\ for denne.
    \end{itemize}
  \item Betalinger på Listen er for folk, der har en oprettet en konto
    (``tab'') hos Fredagscaféen.
    \begin{itemize}
    \item Find kundens navn på Listen.
    \item Beløbet, der købes for, skrives til højre for personens navn
      i det lange felt.
    \item Hvis der står \emph{Kreditstop}, skylder kunden os for mange
      penge, og kan ikke købe mere, før kunden har betalt sin gæld
    \item Man kan skylde op til 0 kr.\ i starten af en barvagt, inden
      man kommer i \emph{Kreditstop}. Det er bartenderens opgave at vurdere,
      om en kunde har nok kredit til at betale for sine varer.
      Det er ikke nødvendigt at lave et regnestykke hver gang, man kan nøjes
      med at bruge sin fornuft og lidt overslagsregning.
      Det gør ikke noget at kunden kommer til at skylde os nogle penge.
      %Listen bliver opgjort mellem
      %fredage, så det er ikke noget problem for kunden, hvis
      %vedkommende går over grænsen i løbet af en fredag.
    \item Beløbet, der er angivet på Listen, er hvor mange penge baren skylder kunden.
      Derfor betyder et positivt tal, at kunden har penge til at købe for.
    \item Hvis kunden gerne vil indbetale penge, skrives dette beløb
      til venstre for kundens navn.
      Hvis der betales med kontanter skal der sættes et kryds i kontant-kolonnen.
    \item Hvis kunden betaler for at komme ud af \emph{Kreditstop},
      udstreges \emph{Kreditstop}, så man kan se, at kunden igen kan
      købe. Her skal der altid betales nok til at kunden som minimum går i 0.
    \end{itemize}
  \end{itemize}
\end{itemize}

\subsection{Opfyldning}
\label{sec:intra:opfyldning}

Under baren skal der sørges for, at vi har noget at sælge. Derfor skal
køleskabene fyldes op undervejs, hvis de er ved at løbe tør. Hvis der
er en bestemt øl, der er populær, bliver kunderne glade hvis, den hele
tiden kan fås i køleskabet, og ikke bare står ude på lageret.

Ligeledes skal der oftest også hentes nye forsyninger af fustager
under barvagten. Brug gerne samme princip som ved toiletruller: Få fat
i nogle nye, mens du stadig har en tilbage.

\subsection{Pizza}
\label{sec:intra:pizza}

Det er muligt at bestille pizza i Fredagscaféen. Der ligger en
eksplicit guide til dette i baren, men her nævnes nogle overordnede tips.

Menuer uddeles ca.\ kl. 16, og der ringes efter pizzaerne ca.\ kl. 17.

Der er en handy liste til at opskrive hvem, der har bestilt hvilke
pizzaer bag på ``Listen''. Normalt betaler man for pizzaen, når man
bliver skrevet på denne pizzaliste.

Bartendere får en gratis pizza på deres barvagt.

\subsection{Pant}
\label{sec:intra:pant}

Der er forskellige slags pant, og de skal sorteres.
\begin{description}
\item[Dåser] Alle dåser, uanset om det er almindelige eller specielle
  øl/sodavand, skal i den store hvide papkasse.
\item[Almindelig øl] Tuborg, Top osv.\ skal i en almindelig ølkasse.
  %Hvis der ikke er flere tomme ølkasser, kan de alternativt kommes i den hvide/gennemsigtige plastkasse.
\item[Almindelig sodavand] Almindelig sodavand på flaske skal i en
  sodavandskasse. Hvis man ikke har en sådan med, kan man som
  nødløsning bruge en almindelig ølkasse.%, eller den hvide/gennemsigtige plastkasse.
\item[Andre glasflasker \underline{med pant-A}] Disse skal i en gennemsigtig
  pant-plastikpose: Disse er normalt ved siden af køleren i baren.
  \begin{itemize}
    \item Disse poser skal kun fyldes op til at de kan fragtes til parkeringskælderen uden at vælte.
    \item Efter barvagten tømmes poserne ned i det grå pantkar i parkeringskælderen.
  %\item Sørg for, at der er en stregkode på denne strips.
  %\item Der burde ligge strips og stregkoder Under Trappen, ellers i Rummet over for Rummet.
  \end{itemize}
\item[Andet] Ting, der ikke hører til nogen af de ovenstående
  katagorier, skal i glascontaineren i parkeringskælderen.
  %den hvide/gennemsigtige plastkasse. Det gælder f.eks. sprutflasker.
\end{description}

\iffalse
Hvis man i slutningen af vagten har ryddet det meste væk, men så
opdager en pant-A-flaske/dåse el.lign.\ kan denne godt puttes i den
gennemsigtige/hvide plastkasse. Dog jo mere, der puttes i denne, desto
mere skal pantvagten manuelt pante, så prøv at begrænse det.
\fi

\subsection{Andet}
\label{sec:intra:andet}

Bartendere får ud over en pizza, også en pose chips samt gratis
``normal'' sodavand under deres barvagt.

Vi udlejer diverse
bræstspil/terninger/spillekort/bordtennisbolde. Kræv et ID (normalt et
studiekort) i pant for spillet. Der er en spand i baren til at
opbevare disse ID'er.

Det er muligt at afspille musik under baren, ved at bruge
mini-jack-stikket i den ene stolpe. Her skal man først bruge en nøgle
til låsen på stolpen, for at tænde for højtalerne. Vent gerne med at
tænde for musikken til ca.\ efter kl. 17, så studerende har mulighed
for at studere i Nygaard om eftermiddagen. Det er bartenderne, der er
DJ's, så det er op til dem, om fulde kunder gerne må sætte musik på,
osv.

Der ringes normalt sidste omgang ca.\ et kvarter før baren lukkes ned
(dvs.\ kl. 21:45). Der er en stor og flot klokke, den heldige bartender
får lov til at ringe med.

\section{Efter Barvagten}
\label{sec:post-barvagten}

Efter barvagten skal der hovedsageligt ske to ting: Der skal ryddes på
plads, og der skal gøres rent.

\subsection{Oprydning og nedlukning}
\label{sec:post:oprydning}

Dankortterminalen afstemmes, og det, der printes ud, kan lægges sammen
med kontanterne. Ting sættes tilbage hvor de kom fra (se
\autoref{sec:pre-barvagten}). Husk at slutte baren til timeren bag gitteret, så den
køler op til næste barvagt.

Det kan være en god ide at få
pengekassen over tidligt, så den ikke efterlades uden opsyn.

Kontanterne og afstemning lægges i deponeringsboksen i Rummet ved
siden af Rummet. Start med at lægge det i en frysepose (der burde være
en rulle ved deponeringsboksen).

\subsection{Rengøring}
\label{sec:post:rengoring}

Formålet med rengøringen er, at det skal se ordentligt ud, ca. som om
der ikke havde været nogen fredagsbar, men bare ``normal'' brug af
lokalerne. Der er en rengøringsvogn med en masse praktiske ting i det
(næst?) fjerneste toilet i Nygaard-kælderen: Det ligger over for
handicaptoiletterne, og man skal bruge en barnøgle for at komme ind.

Her er en liste over de forskellige ting, der skal gøres rent:
\begin{itemize}
\item Indsaml pant og skrald.
\item Bordene skal tøres af.
  \begin{itemize}
  \item Brug gerne sæbevand.
  \item Dette gælder både bordene kunderne har siddet ved, de to høje
    borde i baren og selve barelementet.
  \end{itemize}
\item Anlægget skal skylles igennem. Se guide om dette for flere
  detaljer. Husk at slukke for gassen bagefter.
\item Spildbakken under hanerne tømmes og tørres af.
\item Gulvet skal fejes og vaskes
  \begin{itemize}
  \item Her skal der bare fokuseres på pletter af ting, der er spildt:
    Man behøver derfor normalt ikke at feje/vaske hele gulvet.
  \item Det er dog vigtigt at fjerne ølpletter, da de kan blive meget
    klistrede. Ting som beer pong plejer at generere sådanne pletter.
  \end{itemize}
\item Toiletterne
  \begin{itemize}
  \item Skraldespandene tømmes for papir.
  \item Der tjekkes at der ikke er f.eks. opkast ud over det hele på
    et af toiletterne.
  \end{itemize}
\item Udenfor
  \begin{itemize}
  \item Tjek at der ikke er ølkrus eller lignende på den store trappe,
    hvor rygerne plejer at stå.
  \end{itemize}
\end{itemize}

\section{Troubleshooting}
\label{sec:troubleshooting}

\QnA[ol-skummer]{Fadøllen skummer}{
\item Sørg for at der er tændt for gassen.
\item Hvis anlægget ikke er koldt nok (f.eks. hvis det ikke var
  sluttet til inden baren startede), kan man prøve at give det en
  pause på ca. en time, så det kan køle ned.
\item Det kan være en dårlig fustage. Prøv at udskifte den med en
  anden.
}

\QnA[lavt-tryk]{Der kommer næsten ikke noget fadøl ud af hanen}{
\item Check at fustagen ikke er tom. Hvis den er, sættes en ny på.
  \begin{itemize}
  \item Løft håndtaget op.
  \item Drej mod uret.
  \item Sæt håndtaget på en ny fustage.
  \item Drej med uret, og sørg for at den ikke sidder skævt på.
  \item Tryk håndtaget ned.
  \end{itemize}
\item Check at koblingen/håndtaget sidder ordentligt på fustagen.
\item Check at der er tændt for gassen.
\item Check at gassen ikke er tom. Der er nogle manometre under
  baren. Der er mere gas i Rummet over for Rummet. Da det er sjældent,
  der skal udskiftes gas, må der gerne gøres opmærksom på, at der skal
  købes noget mere: Send gerne en mail til
  \email{best@fredagscafeen.dk}.
}

\QnA[uheld]{Der er sket et uheld/strømafbrydelse/\ldots/verden går under}{
\item Der hænger en seddel ved baren, der beskriver, hvem
  (drift/politi/brandvæsen etc.) der skal kontaktes ved forskellige
  situationer, og hvordan de kan kontaktes.
\item I tilfælde af evakuering\footnote{Vi har heldigvis endnu ikke
    haft en sådan situation.}, bør den arrangementsansvarlige
  bartender kende nødudgangene. Der hænger et kort med disse (flere
  steder) i Nygaard-kælderen.
}

\QnA[tom-bonrulle]{Bonrullen er lyserød}{
\item Bonrullen er snart tom, og skal snart skiftes.
\item Bonrullen starter med at være hvid. Lidt tid inden den er tom,
  bliver den lyserød. Når den så bliver hvid igen er det allersidste
  chance.
\item Der plejer at være friske bonruller i baren; ellers er der nogen
  på barens kontor i Bush-bygningen.
\item Husk at indsætte bonrullen, så der skrives på den hvide side.
}

Hvis du står i en situation, der ikke er nævnt her, er du meget
velkommen til at sende en mail til \email{best@fredagscafeen.dk} eller
\email{oskar@cs.au.dk} (nuværende forfatter), eller lave en branch
på gitlab (hvis du har adgang) med den situation, du mangler
beskrevet. Så kan guiden blive bedre.

\section{Meta}
\label{sec:meta}

Der laves med jævne mellemrum en barplan, der kan ses på
\url{https://fredagscafeen.dk/barplan/}. Derudover kan du også få en
iCalendar (.ics) fil, med netop dine barvagter på
\url{https://fredagscafeen.dk/barplan/shifts/<ditBrugernavnHer>.ics}.

Formatet er fire navne, eksempelvis ``Fredagscafe - housa + brinck +
gittep + jens3694''. Her er det bartenderne \emph{housa},
\emph{brinck}, \emph{gittep} og \emph{jens3694}, der er bartendere, og
da \emph{housa} er den første på listen, er hun den ansvarlige for
denne dag.

Det er kun bartendere på \emph{vandvognen}, der bliver sat som
ansvarlige for en vagt. \emph{vandvognen} er for bestyrelsesmedlemmer
og andre ``erfarne'' bartendere, der har lyst til og mod på at være
den ansvarlige for en barvagt. Som navnet antyder, indebærer det
blandt andet, at man ikke må drikke alkohol under vagten. Send gerne
en mail til \email{best@fredagscafeen.dk}, for at høre mere.

Hvis man har fået en vagt på et tidspunkt, man ikke kan, kan man prøve
at få den byttet. Send en mail til \email{alle@fredagscafeen.dk} og
spørg, om der er nogen, du kan bytte en vagt med.

Hvis en anden ikke kan en bestemt barvagt, og du selv ikke har haft
din første vagt endnu, eller du ikke har flere vagter tilbage på
vagtplanen, er det ikke et problem, at du overtager den: Den, du
bytter med, kan bare overtage en af dine vagter på næste barplan.

\end{document}

%%% Local Variables:
%%% mode: latex
%%% TeX-master: t
%%% End:
