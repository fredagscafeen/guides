\section{Under Barvagten}
\label{sec:intra-barvagten}

Under vagten er der diverse opgaver, der skal løses.

Husk at vaske eller sprite dine hænder af flere gange i løbet af barvagten (sprit findes i baren).

\subsection{Salg af varer}
\label{sec:intra:salg}

Der skal sælges diverse produkter
\begin{itemize}
\item Prisen kan findes i Zettle appen. Hvis den ikke kan findes heri, så spørg et bestyrelsesmedlem.
\item Vi tager mod kontanter, kreditkort (Dankort, VISA, MasterCard
  m.fl.), og ``på listen''.
\item Alle vare der sælges skal registreres gennem Zettle appen, så vi ved, hvor meget vi har på lageret.
  \begin{itemize}
  \item 
    \begin{itemize}
    \item Husk at tjekke, at betalingen blev godkendt.
    \item Man kan som det laveste beløb betale 10 kr.. Det betyder at man ikke kan betale for en enkelt sodavand. Her kan man i stedet foreslå at de køber 2 eller at de betaler for 2 og så får 5 kr. tilbage i kontanter.
    \end{itemize}
  \item Betalinger på Listen er for folk, der har en oprettet en konto
    (``tab'') hos Fredagscaféen.
    \begin{itemize}
    \item Find kundens navn på Listen.
    \item Beløbet, der købes for, skrives til højre for personens navn
      i det lange felt.
    \item Hvis der står \emph{Kreditstop}, skylder kunden os for mange
      penge, og kan ikke købe mere, før kunden har betalt sin gæld
    \item Man kan skylde op til 0 kr.\ i starten af en barvagt, inden
      man kommer i \emph{Kreditstop}. Det er bartenderens opgave at vurdere,
      om en kunde har nok kredit til at betale for sine varer.
      Det er ikke nødvendigt at lave et regnestykke hver gang, man kan nøjes
      med at bruge sin fornuft og lidt overslagsregning.
      Det gør ikke noget at kunden kommer til at skylde os nogle penge.
      %Listen bliver opgjort mellem
      %fredage, så det er ikke noget problem for kunden, hvis
      %vedkommende går over grænsen i løbet af en fredag.
    \item Beløbet, der er angivet på Listen, er hvor mange penge baren skylder kunden.
      Derfor betyder et positivt tal, at kunden har penge til at købe for.
    \item Hvis kunden gerne vil indbetale penge, skrives dette beløb
      til venstre for kundens navn.
      Hvis der betales med kontanter skal der sættes et kryds i kontant-kolonnen.
    \item Hvis kunden betaler for at komme ud af \emph{Kreditstop},
      udstreges \emph{Kreditstop}, så man kan se, at kunden igen kan
      købe. Her skal der altid betales nok til at kunden som minimum går i 0.
    \item Husk at indtaste salget i Zettle appen og sæt en rabat på 100\% så vi bliver ved med at lave lagerstyring
    \end{itemize}
  \end{itemize}
\end{itemize}

\subsection{Opfyldning}
\label{sec:intra:opfyldning}

Under baren skal der sørges for, at vi har noget at sælge. Derfor skal
køleskabene fyldes op undervejs, hvis de er ved at løbe tør. Hvis der
er en bestemt øl, der er populær, bliver kunderne glade hvis, den hele
tiden kan fås i køleskabet, og ikke bare står ude på lageret.

Ligeledes skal der oftest også hentes nye forsyninger af fustager
under barvagten. Brug gerne samme princip som ved toiletruller: Få fat
i nogle nye, mens du stadig har en tilbage.

\subsection{Pizza}
\label{sec:intra:pizza}

Det er muligt at bestille pizza i Fredagscaféen. Der ligger en
eksplicit guide til dette i baren, men her nævnes nogle overordnede tips.

Menuer uddeles ca.\ kl. 16, og der ringes efter pizzaerne ca.\ kl. 17.

Der er en handy liste til at opskrive hvem, der har bestilt hvilke
pizzaer bag på ``Listen''. Normalt betaler man for pizzaen, når man
bliver skrevet på denne pizzaliste.

Bartendere får en gratis pizza på deres barvagt.

Husk at få en kvittering når man tager imod pizzaerne. Den skal i pengekassen.

\subsection{Pant}
\label{sec:intra:pant}

Der er forskellige slags pant, og de skal sorteres.
\begin{description}
\item[Dåser] Alle dåser, uanset om det er almindelige eller specielle
  øl/sodavand, skal i den store hvide papkasse.
\item[Almindelig øl] Tuborg, Top osv.\ skal i en almindelig ølkasse.
  %Hvis der ikke er flere tomme ølkasser, kan de alternativt kommes i den hvide/gennemsigtige plastkasse.
\item[Almindelig sodavand] Almindelig sodavand på flaske skal i en
  sodavandskasse. Hvis man ikke har en sådan med, kan man som
  nødløsning bruge en almindelig ølkasse.%, eller den hvide/gennemsigtige plastkasse.
\item[Andre glasflasker \underline{med pant-A}] Disse skal i en gennemsigtig
  pant-plastikpose: Disse er normalt ved siden af køleren i baren.
  \begin{itemize}
    \item Disse poser skal kun fyldes op til at de kan fragtes til parkeringskælderen uden at vælte.
    \item Efter barvagten tømmes poserne ned i det grå pantkar i parkeringskælderen.
  %\item Sørg for, at der er en stregkode på denne strips.
  %\item Der burde ligge strips og stregkoder Under Trappen, ellers i Rummet over for Rummet.
  \end{itemize}
\item[Andet] Ting, der ikke hører til nogen af de ovenstående
  katagorier, skal i glascontaineren i parkeringskælderen.
  %den hvide/gennemsigtige plastkasse. Det gælder f.eks. sprutflasker.
\end{description}

\iffalse
Hvis man i slutningen af vagten har ryddet det meste væk, men så
opdager en pant-A-flaske/dåse el.lign.\ kan denne godt puttes i den
gennemsigtige/hvide plastkasse. Dog jo mere, der puttes i denne, desto
mere skal pantvagten manuelt pante, så prøv at begrænse det.
\fi

\subsection{Andet}
\label{sec:intra:andet}

Bartendere får ud over en pizza, også en pose chips samt gratis
``normal'' sodavand under deres barvagt.

Vi udlejer diverse
bræstspil/terninger/spillekort/bordtennisbolde. Kræv et ID (normalt et
studiekort) i pant for spillet. Der er en spand i baren til at
opbevare disse ID'er.

Det er muligt at afspille musik under baren, ved at bruge
mini-jack-stikket i den ene stolpe. Her skal man først bruge en nøgle
til låsen på stolpen, for at tænde for højtalerne. Vent gerne med at
tænde for musikken til ca.\ efter kl. 17, så studerende har mulighed
for at studere i Nygaard om eftermiddagen. Det er bartenderne, der er
DJ's, så det er op til dem, om fulde kunder gerne må sætte musik på,
osv.

Der ringes normalt sidste omgang ca.\ et kvarter før baren lukkes ned
(dvs.\ kl. 21:45). Der er en stor og flot klokke, den heldige bartender
får lov til at ringe med.
