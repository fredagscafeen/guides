\section{Under Barvagten}
\label{sec:intra-barvagten}

Under vagten er der diverse opgaver, der skal løses.

Husk at vaske eller sprite dine hænder af flere gange i løbet af barvagten
(sprit findes i baren).

\subsection{Salg af varer}
\label{sec:intra:salg}

Der skal sælges diverse produkter.
\begin{itemize}
    \item Prisen kan findes i \textit{Zettle}\index{Zettle} appen. Hvis den ikke kan findes heri, 
    så prøv at scan stregkoden på varen, eller spørg et bestyrelsesmedlem.
    \item Vi tager imod kontanter, kreditkort (Dankort, VISA, MasterCard
    m.fl.), og ``på listen''.
    %\item Alle vare der sælges skal registreres gennem Zettle appen, så vi ved, hvor meget vi har på lageret.
    \begin{itemize}
        \item Husk at tjekke at betalingen blever godkendt.
        \item Man kan som det laveste beløb betale 10 kr..
    \end{itemize}
    \item Betalinger på \textit{Krydslisten}\index{Krydslisten} er for folk, der har oprettet en konto
    (``tab'') hos \fredagscafeen. Læs den tilhørende \href{https://media.fredagscafeen.dk/guides/krydsliste.pdf}{guide} til Krydslisten.
    \item Kopper og kander fra kunder må ikke fyldes op igen, da vi ikke kan garantere, at de er rene. 
    Fyld derfor i stedet en ny, og sæt de andre til vask.
    \item Ved salg af flasker, skal de åbnes med det samme.
\end{itemize}

Husk, vi må IKKE sælge alkohol til børn under 18 år. 
Det er sjælent det hænder, men er du det mindste i tvivl så spørg efter ID.
De skal kunne vise gyldigt billed ID, og spørg hellere en gang for meget, end en gang for lidt.

\subsection{Opfyldning}
\label{sec:intra:opfyldning}

Under barvagten skal der sørges for, at vi har noget at sælge. Derfor skal
køleskabene fyldes op undervejs, hvis de er ved at løbe tør. Hvis der
er en bestemt øl der er populær, bliver kunderne glade, hvis den
kan fås i køleskabet, og ikke bare står ude på lageret.

Ligeledes skal der oftest også hentes nye forsyninger af fustager
under barvagten. 

\begin{quote}
  Brug gerne samme princip som ved toiletruller: Få fat
  i nogle nye, mens du stadig har en tilbage.
\end{quote}

\subsection{Aftensmad}
\label{sec:intra:aftensmad}
\begin{center}
$\dotfill$ HUSK KVITTERING! $\dotfill$
\end{center}
På et tidspunkt skal der også noget mad indenbords.
Sammen vælger man et sted (typisk i Storcenter Nord)
hvor man gerne vil have mad fra.
Beløbet pr. person skal ligge på maks 70kr..
Dette passer bl.a. med:
\begin{itemize}
    \item En sandwich fra Anettes Sandwich
    \item En rulle, pita eller bowl fra Pita Planet
    \item En falafelrulle fra Rulle Far
    \item Fire cheeseburgere fra Carls
    \item Salat eller sandwich fra DatKant
\end{itemize}
Hvis beløbbet er over 70 kr., betaler man selv differencen. 
Dvs. køber man aftensmad for 90 kr., betaler man 20 kr. i baren.,
og det noteres på kvitteringen.
Det er typisk lettest at blive enige om ét sted at handle ved,
men det er op til dem der er på vagt.

Når man har valgt hvorfra der skal købes,
vælger man én til at notere de andres præferencer,
og smutte afsted og købe det.
Hertil skal man huske at medbringe kontankter fra
\textit{pengekassen}\index{Pengekassen} til at betale for maden.
Byttepenge og kvittering lægges efterfølgende tilbage i pengekassen.
\begin{center}
  $\dotfill$ HUSK KVITTERING! $\dotfill$
\end{center}

\subsection{Pant}
\label{sec:intra:pant}

Der er forskellige slags \textit{pant}\index{Pant}, og det skal sorteres.
\begin{description}
\item[Dåser] Alle dåser, uanset om det er almindelige eller specielle
  øl/sodavand, skal i den store hvide pantkasse.
\item[Almindelige ølflasker] Sportcola, Tuborg, Top osv.\ skal i en almindelig ølkasse.
\item[Hancock sodavandsflasker] Hancock sodavand skal i deres egne lysegrønne kasser. Disse kan findes i træburet.
\item[Almindelig sodavand på flaske] Almindelig sodavand på flaske, som f.eks. Schweppes, skal i en
  rød sodavandskasse. Hvis man ikke har sådan en med, kan man som
  nødløsning bruge en almindelig ølkasse, og senere placere den korrekt ude i Træburet.
\item[Andre glasflasker \underline{med pant}] Disse skal i en gennemsigtig
  pant-plastikpose: Disse er normalt oven på køleanlægget i baren. Hvis ikke, er der flere i Træburet.
  \begin{itemize}
    \item Disse poser skal kun fyldes op til at de stadig kan bæres, og fragtes til parkeringskælderen, 
    uden at vælte eller gå i stykker.
    \item Efter barvagten tømmes poserne ned i det grå pantkar i parkeringskælderen.
  \end{itemize}
\item[Andet] Ting, der ikke hører til nogen af de ovenstående
  katagorier, skal i glascontaineren i parkeringskælderen,
  og samles i en sort affaldssæk ved baren. Det gælder f.eks. sprutflasker.
\end{description}

Hvis pantkassen bliver fyldt, skal den tømmes, og en ny pose sættes i. 
Den fyldte pose fragtes til Træburet og lukkes med en strips.

\subsection{Andet}
\label{sec:intra:andet}

Bartendere får ud over aftensmad, også én pose chips, samt gratis
sodavand under deres barvagt.

Vi udlejer diverse brætspil/terninger/spillekort/bordtennisbolde/pool. 
Kræv et ID (normalt et studiekort) i pant for bordtennisbolde og pool, 
så vi er sikre på at få disse tilbage. 
Der er en kasse i baren til at opbevare disse ID'er.

Det er muligt at afspille \textit{musik}\index{Musik} under barvagten, ved at bruge
jack-stikket i den ene stolpe, forbundet til iPad'en på baren.
Hertil skal man først bruge en barnøgle
til låsen på stolpen, for at tænde for højtalerne.
Spotify er installeret på iPad'en, så det handler bare om at vælge noget at spille.
Det er bartenderne, der er DJ's, så det er op til dem,
om fulde kunder gerne må sætte musik i kø, osv.
Vent gerne med at tænde for musikken til efter kl. 17, så studerende har mulighed
for at studere i Nygaard om eftermiddagen.

Husk at tømme bordene undervejs for affald og tomme flasker, så nye
kunder kan komme til.

Der ringes normalt \textit{sidste omgang}\index{Sidste omgang}
ca. et kvarter før baren lukkes ned
(dvs. kl. 21:45). Der er en stor og flot klokke, den heldige bartender
får lov til at ringe med.

Kl. 22:00 spilles ``skrub af''-musik og der sælges ikke mere.
