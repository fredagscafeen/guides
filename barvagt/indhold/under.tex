\section{Under Barvagten}
\label{sec:intra-barvagten}

Under vagten er der diverse opgaver, der skal løses.

Husk at vaske eller sprite dine hænder af flere gange i løbet af barvagten (sprit findes i baren).

\subsection{Salg af varer}
\label{sec:intra:salg}

Der skal sælges diverse produkter
\begin{itemize}
    \item Prisen kan findes i Zettle appen. Hvis den ikke kan findes heri, så spørg et bestyrelsesmedlem.
    \item Vi tager mod kontanter, kreditkort (Dankort, VISA, MasterCard
    m.fl.), og ``på listen''.
    %\item Alle vare der sælges skal registreres gennem Zettle appen, så vi ved, hvor meget vi har på lageret.
    \begin{itemize}
        \item Husk at tjekke at betalingen blever godkendt.
        \item Man kan som det laveste beløb betale 10 kr..
    \end{itemize}
    \item Betalinger på Listen er for folk, der har oprettet en konto
    (``tab'') hos \fredagscafeen.
    \begin{itemize}
        \item Find kundens navn på Listen\index{Listen}.
        \item Beløbet, der købes for, skrives til højre for personens navn
        i det lange felt.
        \item Hvis der står \emph{Kreditstop}\index{Kreditstop}, skylder kunden os for mange
        penge, og kan ikke købe mere, før kunden har betalt sin gæld.
        \item Man kan skylde op til 0 kr.\ i starten af en barvagt, inden
        man kommer i \emph{Kreditstop}. Det er bartenderens opgave at vurdere,
        om en kunde har nok kredit til at betale for sine varer.
        Det er ikke nødvendigt at lave et regnestykke hver gang, man kan nøjes
        med at bruge sin fornuft og lidt overslagsregning.
        Det gør ikke noget at kunden kommer til at skylde os nogle penge.
        %Listen bliver opgjort mellem
        %fredage, så det er ikke noget problem for kunden, hvis
        %vedkommende går over grænsen i løbet af en fredag.
        \item Beløbet, der er angivet på Listen, er hvor mange penge baren skylder kunden.
        Derfor betyder et positivt tal, at kunden har penge til at købe for.
        \item Hvis kunden gerne vil indbetale penge, vælges ``Amount'' i Zettle og beløbet indtastes, hvorefter kunden betaler.
        Derefter skrives dette beløb til venstre for kundens navn på Listen.
        Hvis der betales med kontanter, skal der i stedet sættes et kryds i kontant-kolonnen.
        \item Hvis kunden betaler for at komme ud af \emph{Kreditstop},
        udstreges \emph{Kreditstop}, så man kan se, at kunden igen kan
        købe. Her skal der altid betales nok til at kunden som minimum går i 0.
        %\item Husk at indtaste salget i Zettle appen og sæt en rabat på 100\% så vi bliver ved med at lave lagerstyring
    \end{itemize}
\end{itemize}

\subsection{Opfyldning}
\label{sec:intra:opfyldning}

Under barvagten skal der sørges for, at vi har noget at sælge. Derfor skal
køleskabene fyldes op undervejs, hvis de er ved at løbe tør. Hvis der
er en bestemt øl, der er populær, bliver kunderne glade, hvis den hele
tiden kan fås i køleskabet, og ikke bare står ude på lageret.

Ligeledes skal der oftest også hentes nye forsyninger af fustager
under barvagten. Brug gerne samme princip som ved toiletruller: Få fat
i nogle nye, mens du stadig har en tilbage.

\subsection{Aftensmad}
\label{sec:intra:aftensmad}
På et tidspunkt skal der også noget mad indenbords. 
Sammen vælger man et sted (typisk i Storcenter Nord) 
hvor man gerne vil have mad fra.
Beløbet pr. person skal helst ligge omkring 70kr..
Dette passer bl.a. med:
\begin{itemize}
    \item En sandwich fra Anettes Sandwich
    \item En rulle, pita eller bowl fra Pita Planet
    \item En falafelrulle fra Rulle Far
\end{itemize}
Det er typisk lettest at blive enige om ét sted at handle ved, 
men det er op til den enkelte.\\
Når man har valgt hvad der skal købes, 
vælger man én til at notere de andres præferencer, 
og smutte afsted og købe det.
Hertil skal man huske at medbringe kontankter fra pengekassen til at betale for maden. 
Byttepenge og kvittering lægges tilbage i pengekassen.

\subsection{Pant}
\label{sec:intra:pant}

Der er forskellige slags pant, og de skal sorteres.
\begin{description}
\item[Dåser] Alle dåser, uanset om det er almindelige eller specielle
  øl/sodavand, skal i den store hvide papkasse.
\item[Almindelig øl] Tuborg, Top osv.\ skal i en almindelig ølkasse.
  %Hvis der ikke er flere tomme ølkasser, kan de alternativt kommes i den hvide/gennemsigtige plastkasse.
\item[Almindelig sodavand] Almindelig sodavand på flaske skal i en
  sodavandskasse. Hvis man ikke har sådan en med, kan man som
  nødløsning bruge en almindelig ølkasse.%, eller den hvide/gennemsigtige plastkasse.
  \item[Hancock sodavand] Hancock sodavandsflasker skal i deres egne lysegrønne kasser. Disse kan findes bag gitteret.
\item[Andre glasflasker \underline{med pant-A}] Disse skal i en gennemsigtig
  pant-plastikpose: Disse er normalt ved siden af køleren i baren.
  \begin{itemize}
    \item Disse poser skal kun fyldes op til at de kan fragtes til parkeringskælderen uden at vælte.
    \item Efter barvagten tømmes poserne ned i det grå pantkar i parkeringskælderen.
  %\item Sørg for, at der er en stregkode på denne strips.
  %\item Der burde ligge strips og stregkoder Under Trappen, ellers i Rummet over for Rummet.
  \end{itemize}
\item[Andet] Ting, der ikke hører til nogen af de ovenstående
  katagorier, skal i glascontaineren i parkeringskælderen, 
  og samles i en sort affaldssæk ved baren. Det gælder f.eks. sprutflasker.
\end{description}

\subsection{Andet}
\label{sec:intra:andet}

Bartendere får ud over aftensmad, også en pose chips samt gratis
``normal'' sodavand under deres barvagt.

Vi udlejer diverse
bræstspil/terninger/spillekort/bordtennisbolde. Kræv et ID (normalt et
studiekort) i pant for spillet. Der er en kasse i baren til at
opbevare disse ID'er.

Det er muligt at afspille musik under barvagten, ved at bruge
jack-stikket i den ene stolpe, forbundet til iPad'en på baren. 
Hertil skal man først bruge en bar-nøgle
til låsen på stolpen, for at tænde for højtalerne. 
Spotify er installeret på iPad'en, så det handler bare om at vælge noget at spille.
Det er bartenderne, der er DJ's, så det er op til dem, 
om fulde kunder gerne må sætte musik på, osv.
Vent gerne med at tænde for musikken til efter kl. 17, så studerende har mulighed
for at studere i Nygaard om eftermiddagen. 

Der ringes normalt sidste omgang ca. et kvarter før baren lukkes ned
(dvs. kl. 21:45). Der er en stor og flot klokke, den heldige bartender
får lov til at ringe med.
