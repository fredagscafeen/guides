\section{Før Barvagten}
\label{sec:pre-barvagten}

Vi mødes 14.30 ved gitteret. Gitteret er ved spiraltrappen ned til Nygaard kælderen.

Baren åbner kl. 15.00, og det er vigtigt
at vi IKKE sælger alkohol før, da det først er der vi har vores alkoholbevilling.

Før barvagten, skal baren sættes op med alt hvad det indebærer. Her er
det sorteret efter, hvorfra ting skal hentes/ting skal ordnes.

Under barvagten har du som bartender adgang til en \textit{barnøgle}\index{Barnøgle}.
Denne nøgle ligger i baren og kan låse op til \textit{Bag Gitteret}\index{Bag Gitteret},
\textit{Rummet under Trappen}\index{Rummet under Trappen}, 
\textit{Rengøringsrummet}\index{Rengøringsrummet} overfor handicaptoiletterne i Nygaard -1,
\textit{Træburet}\index{Træburet} i Hopper -1
og lageret i \textit{Parkeringskælderen}\index{Parkeringskælderen}.

Baren lukker normalt kl. 22:00, men kan i nogle tilfælde lukkes tidligere, hvis der er under
$n$ kunder, hvor $n \le$ få. 
Ved fester eller lign. kan vi dog lukke senere, dette vil blive varslet i forvejen og der vil
være flere hold bartendere tilknyttet en sådan vagt.

\subsection{Bag Gitteret}
\label{sec:pre:bag-ved-gitteret}

\begin{itemize}
	\item Baren skal køres hen til barområdet med tilhørende \textit{drypspand}\index{Drypspand}.
	\item Køleskabende eller \textit{Kæleskabene}\index{Kæleskabene} skal være fyldte, og køres hen på hver sin side af køleskabene i
	\textit{Barområdet}\index{Barområdet}. Vær forsigtig når du kører med køleskabene, da de ikke er fastmonteret på hjulene,
	samt at de kan falde forover, hvis man skubber for hårdt, og for højt oppe. Se \autoref{fig:køleskab-rip}. Derudover skal
	du passe på, at du ikke kører ind i noget på vejen. Skulle noget af indholdet ligge sig op af døren, kan man lægge en jakke
	ned under, og forsigtigt åbne, mens man griber flasker inde mellem døren.
	\item Rene plastkopper, kander og en opvaskebakke hentes.
	\item Papkassen til pant hentes.
	\item Hvis det er den første fredag i måneden, så tag også lykkehjulet frem.
\end{itemize}

\begin{figure}[H]
	\centering
	\includegraphics[scale=0.3]{billeder/kæleskab_rip.jpg}
	\caption{R.I.P. Kæle-2 (?-2015)}
	\label{fig:køleskab-rip}
\end{figure}

\subsection{Parkeringskælderen}
\label{sec:pre:hopper}

\begin{itemize}
	\item Der skal hentes fustager i Parkeringskælderen. Normalt 1
	Extra Pilsner, 1 cider og 1 specialøl.
	\item Derudover skal der hentes flaskeøl. Normalt 2 kasser Top, 2 kasser Tuborg Classic,
	1 kasse Odence Classice, 1 kasse Tuborg Grøn og 1 kasse Sport Cola.
	\item Et stort stykke pap til at stille opvaskebakken til brugte kopper på.
\end{itemize}

\subsection{Under Trappen}
\label{sec:pre:under-trappen}

\begin{itemize}
	\item Chips skal hentes.
	\item \textit{Ståmåtten}\index{Ståmåtten} skal hentes.
	\item \textit{Brætspilsvognen}\index{Brætspilsvognen} skal køres ind.
	\item Vognen med spiritus, shotglas og shotsbakker skal køres ind.
\end{itemize}

\subsection{DatKant}
\label{sec:pre:datkant}

\begin{itemize}
	\item Opvaskestativet hentes fra opvaskerummet i DatKant, og køres ned til barområdet.
	\item En affaldssæk eller et stykke pap placeres under opvaskebakken til brugte plastkopper.
\end{itemize}

\subsection{Barområdet}
\label{sec:pre:baromradet}

Barområdet stilles typisk op i Nygaard-K02, som set på figur \ref{fig:baromraadet}.

\begin{figure}[t]
	\centering
	\def\svgwidth{\columnwidth}
	{\small\import{billeder/}{baromraadet.pdf_tex}}
	\caption{Barområdet i Nygaard-K02}
	\label{fig:baromraadet}
\end{figure}

\begin{itemize}
    \item \textit{Hængelåsene}\index{Hængelåsene} omkring køleskabene i barområdet låses op med barnøglen.
	\item Der sluttes strøm til køleskabene og tændes for lyset i køleskabene.
	\item Efter baren er kørt på plads, sættes de to barborde op, og der sættes barstole op omkring dem.
	\item En skraldespand stilles ved siden af baren.
	\item Baren skal sluttes til strøm.
	\item Fustager skal sluttes til, og der skal tændes for gassen:
	\begin{itemize}
		\item Pilsner sluttes til de yderste haner (her dækker 1 fustage begge haner).
		\item Cider sluttes til midt-venstre hane.
		\item Specialøl sluttes til midt-højre hane.
	\end{itemize}
\end{itemize}

\subsection{Rengøringsrummet}
\label{sec:pre:rengøring}
\begin{itemize}
	\item Fra rengøringsrummet kan man med fordel hente en spand med sæbevand og en klud.
	\item Derudover, findes der en nøgle oven i papirdispenseren. Denne kan man bruge til
	at åbne ind til papirrullen, og man kan hente en rulle ``\textit{Spæns}\index{Spæns}''
	fra et af de åbne toiletter, hvis der mangler i baren. 
	Husk at fjerne det blå stykke plastik i enden.
\end{itemize}
