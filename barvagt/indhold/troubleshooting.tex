\section{Troubleshooting}
\label{sec:troubleshooting}

\QnA[ol-skummer]{Fadøllen skummer}{
    \item Sørg for at der er tændt nok op for gassen.
    \item Hvis anlægget ikke er koldt nok (f.eks. hvis det ikke var
    sluttet til inden baren startede), kan man prøve at give det en
    pause på ca. en time, så det kan køle ned.
    \item Det kan være en dårlig fustage. Prøv at udskifte den med en
    anden.
}

\QnA[lavt-tryk]{Der kommer næsten ikke noget fadøl ud af hanen}{
    \item Check at fustagen ikke er tom. Hvis den er, sættes en ny på.
    \begin{itemize}
        \item Løft håndtaget op.
        \item Drej mod uret.
        \item Sæt håndtaget på en ny fustage.
        \item Drej med uret, og sørg for at den ikke sidder skævt på.
        \item Tryk håndtaget ned.
    \end{itemize}
    \item Check at koblingen/håndtaget sidder ordentligt på fustagen.
    \item Check at \textit{quickconnect}\index{Quickconnect} er forbundet korrekt.
    \item Check at der er tændt for gassen.
    \item Check at gassen ikke er tom. Der er nogle manometre under
    baren. Der er mere gas i parkeringskælderen. Da det er sjældent,
    der skal udskiftes gas, må der gerne gøres opmærksom på, at der skal
    købes noget mere: Send gerne en mail til
    \bestmail.
}

\QnA[uheld]{Der er sket et uheld/strømafbrydelse/\ldots/verden går under}{
    \item Der hænger en seddel ved baren, der beskriver, hvem
    (drift/politi/brandvæsen etc.) der skal kontaktes ved forskellige
    situationer, og hvordan de kan kontaktes.
    \item I tilfælde af evakuering\footnote{Vi har heldigvis endnu ikke
        haft en sådan situation.}, bør den ansvarlige
    bartender kende nødudgangene. Der hænger et kort med disse (flere
    steder) i Nygaard-kælderen.
}

Hvis du står i en situation, der ikke er nævnt her, er du meget
velkommen til at sende en mail til \bestmail,
eller lave en pull request på \href{https://github.com/fredagscafeen/guides}{github}
med den situation, du mangler beskrevet. Så kan guiden blive bedre.
