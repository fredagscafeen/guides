\section{Efter Barvagten}
\label{sec:post-barvagten}

Efter barvagten skal der hovedsageligt ske to ting: Der skal ryddes på
plads, og der skal gøres rent.

\subsection{Oprydning og nedlukning}
\label{sec:post:oprydning}

\begin{itemize}
	\item Ting sættes tilbage hvor de kom fra (se
	\autoref{sec:pre-barvagten}).
	\item Husk at sætte strøm til køleskabene Bag Gitteret,
	men tag strøm fra køleskabene i Barområdet.
	Køleskabene i Barområdet skal stå åbne på klem,
	ved at stikke en træklods i klemme, når køleskabene låses igen.
	\item Husk at slutte baren til timeren bag gitteret, så den
	er klar til næste barvagt.
	\item Kasser med tomme flasker skal ud i Træburet i Hopper -1.
	\item Pantflasker skal ud i pantkaret i parkeringskælderen.
	\item Almindelige glasflasker uden pant, skal sorters i glasaffald i parkeringskælderen.
	\item Højtalerne slukkes med låsen på stolpen, så det ikke brummer.
\end{itemize}

\subsection{Rengøring}
\label{sec:post:rengoring}

Formålet med rengøringen er, at det skal se ordentligt ud, ca. som om
der ikke havde været nogen fredagsbar, men bare ``normal'' brug af
lokalerne. Der er en rengøringsvogn i
rengøringsrummet overfor handicaptoiletterne i Nygaard,
og man skal bruge en barnøgle for at låse sig ind.

Her er en liste over de forskellige ting, der skal gøres rent:
\begin{itemize}
	\item Indsaml pant og skrald. Bundslatter tømmes over i en slatspand.
	\item Bordene skal tøres af med sæbevand.
	\begin{itemize}
		\item Dette gælder både bordene kunderne har siddet ved, de to høje
			borde i baren, spritvognen og selve baren.
	\end{itemize}
	\item Anlægget skal skylles igennem. Se \href{https://media.fredagscafeen.dk/guides/rensningafanlaeg.pdf}{guiden} til dette for flere
	detaljer. Husk at slukke for gassen bagefter.
	\item Spildbakken under hanerne vaskes og tørres af.
	\item Gulvet skal fejes og vaskes.
	\begin{itemize}
		\item Her skal der bare fokuseres på pletter af ting, der er spildt:
			Man behøver derfor normalt ikke at vaske hele gulvet.
		\item Det er dog vigtigt at fjerne ølpletter, da de kan blive meget
			klistrede. Ting som beer pong plejer at generere sådanne pletter.
		\item Det er også en god idé at vaske ståmåtten, der har ligget bag baren.
	\end{itemize}
	\item Skænkepropper vaskes igennem med sæbevand og tørres af med spæns. 
	\item Oplukkere der er klamme vaskes.
	\item Skraldespandene i Nygaard kælderen skal tømmes og skiftes med nye affaldsposer.
	\item Toiletterne:
	\begin{itemize}
		\item Det tjekkes, at der er skyllet ud, og ulækre situationer gøres rent.
		\item Gulvet vaskes hvis nødvendigt, og papir på gulvet kommes i skraldespanden.
		\item Skraldespandene tømmes, og nye poser skal sættes i.
	\end{itemize}
	\item Udenfor:
	\begin{itemize}
		\item Tjek at der ikke er flasker eller lignende på den store trappe,
			hvor rygerne plejer at sidde.
	\end{itemize}
\end{itemize}
