\section{Efter Barvagten}
\label{sec:post-barvagten}

Efter barvagten skal der hovedsageligt ske to ting: Der skal ryddes på
plads, og der skal gøres rent.

\subsection{Oprydning og nedlukning}
\label{sec:post:oprydning}

\begin{itemize}
	\item Ting sættes tilbage hvor de kom fra (se
	\autoref{sec:pre-barvagten}).
	\item Husk at sætte strøm til de køleskabene Bag Gitteret,
	men tag strøm fra køleskabene i Barområdet.
	Køleskabene i Barområdet skal stå åbne på klem,
	ved at stikke træklodsen i klemme, når køleskabene låses igen.
	%Husk at slutte baren til timeren bag gitteret, så den
	%køler op til næste barvagt.
	\item Kasser med tomme flasker skal ud i Træburet i Hopper -1.
	\item Pantflasker skal ud i pantkaret i parkeringskælderen.
	\item Almindelige glasflasker uden pant, skal sorters i glasaffald i parkeringskælderen.
	\item Højtalerne slukkes med låsen på stolpen.
\end{itemize}

\subsection{Rengøring}
\label{sec:post:rengoring}

Formålet med rengøringen er, at det skal se ordentligt ud, ca. som om
der ikke havde været nogen fredagsbar, men bare ``normal'' brug af
lokalerne. Der er en rengøringsvogn med en masse praktiske ting i
rengøringsrummet overfor handicaptoiletterne i Nygaard,
og man skal bruge en barnøgle for at komme ind.

Her er en liste over de forskellige ting, der skal gøres rent:
\begin{itemize}
	\item Indsaml pant og skrald. Bundslatter tømmes over i en slatspand.
	\item Bordene skal tøres af.
	\begin{itemize}
		\item Brug gerne sæbevand.
		\item Dette gælder både bordene kunderne har siddet ved, de to høje
			borde i baren og selve barelementet.
	\end{itemize}
	\item Anlægget skal skylles igennem. Se \href{https://media.fredagscafeen.dk/guides/rensningafanlaeg.pdf}{guide} om dette for flere
	detaljer. Husk at slukke for gassen bagefter.
	\item Spildbakken under hanerne vaskes og tørres af.
	\item Gulvet skal fejes og vaskes
	\begin{itemize}
		\item Her skal der bare fokuseres på pletter af ting, der er spildt:
			Man behøver derfor normalt ikke at feje/vaske hele gulvet.
		\item Det er dog vigtigt at fjerne ølpletter, da de kan blive meget
			klistrede. Ting som beer pong plejer at generere sådanne pletter.
		\item Det er også en god idé at vaske måtten der har ligget bag baren.
	\end{itemize}
	\item Skænkepropper vaskes igennem med sæbe og tørres. Her kan man med fordel komme dem i en opvaskespand med sæbevand.
	\item Oplukkere
	\item Skraldespandene i Nygaard kælderen skal tømmes og skiftes med nye affaldsposer.
	\item Toiletterne
	\begin{itemize}
		\item Skraldespandene tømmes, og nye poser skal sættes op.
		\item Der tjekkes at der ikke er f.eks. opkast ud over det hele på
			et af toiletterne.
	\end{itemize}
	\item Udenfor
	\begin{itemize}
		\item Tjek at der ikke er flasker eller lignende på den store trappe,
			hvor rygerne plejer at stå.
	\end{itemize}
\end{itemize}
