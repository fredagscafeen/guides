% !TeX spellcheck = da_DK
\documentclass[danish]{article}
% Ability to write input files using utf8
\usepackage[utf8]{inputenc}

% Proper font, æ, ø and å becomes copy/paste and searchable
\usepackage[T1]{fontenc}
\usepackage{lmodern}

% Enable \includegraphics, so that images can be included
\usepackage{graphicx}
\DeclareGraphicsExtensions{.pdf, .png, .jpg, .PDF, .PNG, .JPG}

% Enables use of links, and adds ToC for your PDF-reader
\usepackage{hyperref}
% A small macro for inserting clickable email adresses, e.g.,
% \email{best@fredagscafeen.dk}
\newcommand*{\email}[1]{\href{mailto:#1}{\nolinkurl{#1}}}

% Adding visible TODO's using \todo
\usepackage[obeyFinal]{todonotes}

% Better kerning etc.
\usepackage{microtype}

% Hyphenation for Danish words etc.
\usepackage{babel}

% For conditionals in commands
\usepackage{xifthen}

% Macro for Question and answers, troubleshooting etc.
% The optional argument is a label to ref to. The label will be prefixed with "qna:"
\newcommand{\QnA}[3][]{
\paragraph{#2}
% Insert label if it is provided in the optional argument
\ifthenelse{\isempty{#1}}
  {}
  {\label{qna:#1}}
  \begin{itemize}
    #3
  \end{itemize}
}

% Macros for common words, so that we can format them fancily. xspace
% inserts spaces when needed based on context.
\usepackage{xspace}

\newcommand*{\fotex}{F%
\kern -.25em%
{\raisebox{-.215em}{Ø}}%  % -.215em makes 2 Es match
\kern -.25em%
\TeX\xspace}

% Better references, automatically uses Danish names. Use \cref
% instead of \autoref. See package documentation for more details.
%
% nameinlink makes the prefix (e.g. "kapitel") a part of the hyperlink
% (as \autoref does). This might not be a good idea if multiple labels
% are used in a \cref.
\usepackage[nameinlink]{cleveref}


\title{Eventguide}
\date{\formatdate{26}{5}{2020}}
\author{Christian Zhuang-Qing Nielsen (Ching Chong)}

\begin{document}

\maketitle
\noindent\textbf{Disclaimer:} This guide is in Danish. If for some reason the event manager of Fredagscaféen cannot read Danish, then he/she should probably learn to.
\\ \\ \\ \\
\noindent Kære eventansvarlige,
\\ \\
Velkommen til en post, hvor der er plads til både seriøse ting, men også en masse sjov og ballade. I den her guide gennemgår vi de årlige events, håndtering af eksterne kontakter, ekstraordinære events, planlægning af events, håndtering af events på dagen, og sidst men ikke mindst at lave tilmeldinger som bartenderne kan tilmelde sig. Denne guide behøver ikke nødvendigvis følges til punkt og prikke. Du kan jo egentlig gøre som du vil fordi jeg er her ikke til at holde dig i ørene, men lad dette dokument være en spirituel vejledning hvis du bliver i tvivl.

\section*{Det essentielle}
\label{sec:det-essentielle}

Det overordnede mål er at planlægge og arrangere events, både for gæsterne
og bartenderne. Her skal der skelnes mellem de arrangementer som kun er for bartenderne og de arrangementer som afholdes nede i fredagsbaren som alle kan komme med til. Events skal være sjove for alle der er med. Heldigvis er det nemt at underholde bartenderne fordi de plejer som regel at være gode venner så de er gode til at underholde sig selv. Førhen har det været lidt besværligt at få sørge for at der var nok som meldte sig til begivenhederne, men på det seneste har vi nærmest haft det omvendte problem. Ikke desto mindre er det altid en fornuftig idé at påminde folk konstant at de skal huske at melde sig til og at det bliver skide skægt. Som eventansvarlig er det altid dig som har ansvaret (med mindre andet er aftalt) for et pågældende event (Det fungerer lidt anderledes for sponsorbarer, mere om det senere). Det betyder at du altid burde lægge arrangementerne de dage hvor først og fremmest SELV KAN DELTAGE. Det betyder også, at man aldrig skal drikke sig så fuld til de begivenheder man selv arrangerer at man ikke kan holde styr på hvad der sker og derfor ikke kan tage ansvar for tingene. Når man har sagt det så betyder det overhovedet ikke at man ikke kan have det sjovt sammen med de andre. Tværtimod så er min erfaring at man sagtens kan gå rundt og have det gevaldig sjovt selvom man samtidig har ansvaret. Her handler det jo så om at have styr på tingene og sørge for at have en ordentlig plan.

\section*{Planlægning}
\label{sec:det-praktiske}

Events, der skal afholdes, skal planlægges først. Derfor bør et event være et punkt til det/de forgående bestyrelsesmøde(r). Det er dit ansvar at sætte punktet på dagsordenen og planlægge kort hvad du vil sige til punktet. Her plejer man som eventansvarlig (afhængigt af selve eventet) at præsentere hvad det handler om, og så ens plan for hvordan det kommer til at foregå. Før noget som helst planlægges, så skal events godkendes af bestyrelsen og dermed gennemgås til et bestyrelsesmøde (hvis muligt).

Når punktet over et event er overstået, bør følgende være klart:
\begin{itemize}
\item Hvad eventet går ud på
\item Hvornår det foregår, og
\item Hvem der står for div. praktiske ting.
\end{itemize}

\noindent Efterfølgende skal en dato fastslås. Til at finde en dato har jeg fundet ud af at den bedste måde er først og fremmest at finde alle de datoer du selv kan, og så lave en Doodle til resten af bestyrelsen for at bruge dem som en slags stikprøve. Fra doodlen vælger man den dag hvor de fleste kan og så bliver det dén dato man sender ud til bartenderne. \textbf{Sørg for at tjekke TÅGEKAMMERETS og Mat/Fys-Tutorgruppens kalendere så man ikke kommer til at vælge en dato hvor der allerede ligger noget andet.} Man skal selvfølgelig heller ikke ligge begivenheder oven på Kapsejladsen, Danmarks største fredagsbar, og andre større begivenheder som de studerende hellere vil til.

Hernæst skal arrangementet planlægges nærmere. F.eks. til en sommerfest så er det rart at have nogle til at gennemgå hvor meget mad der skal købes ind, samt for lavet en tjekliste af alle andre ting som service, badebassiner, osv. som også skal købes ind. Selvom dette ofte godt kan klares af dig alene, så er det ofte rart lige at hive fat i en eller to fra bestyrelsen for lige at have nogen at vende den med. Med hensyn til maden til forskellige arrangementer, så er det bedst hvis man kan blive enige til bestyrelsesmødet forinden om hvilken mad bartenderne skal have på dagen.

Til nogle arrangementer vil der være en del arbejde i form af opsætning, madlavning og oprydning. Her er det vigtigt at man ikke prøver at gøre det hele selv,
men derimod udnytter den gratis arbejdskraft i form af frivillige bartendere. Her vil man til tilmeldingen af eventet nok lave ekstra valg som bartenderne kan lave, hvor de som minimum \textit{skal} tilmelde sig en anden post, som f.eks. opsætning af julefrokost (så de møder tidligere på dagen) eller oprydning (så de møder dagen efter). Det er dog stadig dit ansvar som eventansvarlig at folk rent faktisk kommer til deres poster, så man kan med fordel lave konsekvenser hvis folk ikke kommer og laver deres ting. Det er derfor også vigtigt på dagen at du tjekker at folk rent faktisk kommer og hjælper så de ikke bare er nogle små nasserøve.

Når et event er planlagt, så skal dette næsten altid informeres til bartenderne. Dette sker ofte i form at emails. Her er det igen vigtigt at man informerer om:
\begin{itemize}
    \item Hvad eventet handler om,
    \item Hvornår det finder sted,
    \item Hvordan man tilmelder sig,
    \item Hvem der er velkommen: Er det kun bartendere eller også gamle bestyrelsesmedlemmer eller en +1 fra bartenderne?,
    \item Hvad der kommer til at foregå til eventet (f.eks. en tidsplan) og hvor længe det tager,
    \item Hvorvidt eventet kommer til at koste noget for bartenderne,
    \item Hvor bartenderne skal mødes henne på dagen,
    \item Om der vil være en ekstraaktivitet dagen efter (bowling, skøjtning, osv) eller senere på dagen (Thomas \& Tim), samt informationer omkring dette.
    \item Evt. flere ting
\end{itemize}

\noindent Det er vigtigt at mails til bartenderne kommer ud i god tid, og at det også kommer \textbf{en masse reminder mails}. Man bør \textbf{fremhæve vigtige informationer med fed skrift}. Man kan med fordel koordinere med den PR-ansvarlige hvis det er begivenheder som alle er velkommen til. På den måde kan man få lavet en fed poster der reklamerer gratis øl, generalforsamling, eller lignende.

\subsection*{Udgifter og Indkøb}
Hvis der er udgifter for begivenheden (hvilket der næsten altid er), så husk at vende disse med kasseren for fredagsbaren. Hvis der skal købes noget i på dagen, så er det klart nemmest at bruge købekortet i Føtex. Hvis man ikke skal i Føtex så har fredagsbaren også et Mastercard eller VISA som kan benyttes af kasseren (f.eks. til Anettes Sandwich).

\subsection*{Oprydning}
\label{sec:oprydning}

Det vigtigste ved oprydningen er at det slet ikke tager så lang tid når man er mange omkring det. Hvis oprydningen finder sted dagen efter, så kan man passende lokke med noget gratis morgenmad. Føtex har morgenmadspakker til fornuftig pris som kan bestilles nogle dage før til afhenting på dagen. Prøv også at lave et tidsestimat på oprydningen. Husk at lægge vægt på at det slet ikke er så galt at rydde op, for folk plejer ofte at tro at det er helvede selv. I virkeligheden er det kun slemt hvis man er meget få om det. Man skal nogle gange overveje om antallet af opsætning / madlavningspladser skal begrænses så resten er tvunget til at hjælpe med oprydningen. \textbf{Hvis man er den eventansvarlige så er man selvfølgelig med til BÅDE OPSÆTNING OG OPRYDNING}

Mht. rengøring af service efter julefrokost, se sektion omkring Bestilling af Mad og Catering.

\section*{Tilmeldingsformularer}
\label{sec:tilmelding}

Førhen har tilmeldinger altid foregået udelukkende over Doodle. Dette system er næsten helt blevet erstattet af vores eget tilmeldingssystem som kører over hjemmesiden \url{https://fredagscafeen.dk/admin/events/}. Bemærk det kræver at man har admin-rettigheder til Events-funktionaliteten på hjemmesiden for at man kan oprette nye events og ændre i de gamle. Dette system sørger helt automatisk for, at det udelukkende er bartendere som kan tilmelde sig begivenheder, og at de er aktive inde på hjemmesiden. Af og til vil man gerne have at enkelte folk, der ikke er bartendere (længere) også gerne må komme med. Hvis de har været bartendere førhen, så kan disse nemt whitelistes inde på hjemmesiden. Ligeledes kan man blackliste bartendere hvis man synes de har været træls gennem en længere periode. Systemet inde på hjemmesiden er lidt funky, så man skal lige lege lidt med det før man helt forstår det. Idéen er at man kan oprette et Event-objekt, og en række Event-choices-objekter som kan forbindes med dette event. De popper så op inde på hjemmesiden for bartenderne på \url{https://fredagscafeen.dk/events/}, og når de svarer inde på siden bliver der så lavet et Event-response objekt. Man kan også ændre på disse inde på hjemmesiden hvis nu en bartender kommer til at lave en fejltilmelding eller lignende.

Til hvert event er der automatisk et choice om hvorvidt bartenderen deltager. Andre muligheder er f.eks. om de hjælper med at sætte op, med at rydde op eller begge dele. Der bør også være et valg til hvis man er vegetar/veganer eller lider af allergier. Sidst men ikke mindst skal der til Aarhus Bryghusturen være et felt hvor folk kan vælge en sandwich at tage med.

Til nogle events skal der folk med som ikke er bartendere. Dette er f.eks. Nytår i datbar, hvor folk plejer at måtte have en +1 med. Til denne begivenhed plejer jeg at bruge en Doodle, hvor bartenderne skriver både dem selv samt deres +1 med på.

En tommelfingerregel for hvem der må komme med er nuværende bartendere (eller folk som lige er stoppet med at være bartendere) samt forrige års bestyrelse og nogle gamle formænd.

\section*{Bestilling af Mad og Catering}
Til Aarhus Bryghusturen plejer vi at bestille nogle sandwich igennem Anettes sandwich i Storcenter Nord. Husk at gøre dette i god tid. Det er nogle kæmpe idioter, som arbejder dernede som altid laver fejl i bestillingen, så husk at tjekke grundigt efter om der er sandwiches nok.

Til julefrokost plejer vi at bruge \textbf{Kokken \& Jomfruen}. De leverer fin mad i store mængder til billige penge. Når man bestiller fra disse skal der ALTID bestilles færre kuverter end der er mennesker hvis man vil undgå en overflod af mad. F.eks. hvis man er 35 personer ville jeg nok bestille omkring 28-30 kuverter. Man kan få sammensat vilkårlige elementer fra menuerne hvis man ringer til dem. De leverer maden og henter servicen igen. Servicen bør være tømt for madrester og skyllet let af. Man behøver SLET ikke vaske det grundigt op da de kun tager et mindre beløb for at gøre dette for en selv. Samtidig optjener man kredit på deres hjemmeside, og de plejer bare at tage en del af denne kredit for selv at rengøre servicen.

\section*{Transport}
Til nogle events som f.eks. Aarhus Bryghustur eller Djurs Sommerlandstur skal bartenderne fragtes frem og tilbage. Dette kan enten gøres gennem Midttrafik gruppebestilling inde på \url{https://midttrafikbestilling.dk/}. Vær opmærksom på at man SKAL bestille en grupperejse hvis man er mange af sted, ellers så er det ikke sikkert der er plads inde i busserne. Til Aarhus Bryghusturen kan man nøjes med at bestille en-to zonebilletter så det passer med antallet af mennesker. Her kan man passende tage Letbanen mod Odder som kører forbi Aarhus Bryghus ude ved Viby.

Hvis økonomien går godt det pågældende år, så kan man snakke med kasseren og resten af bestyrelsen om at leje en bus til at køre alle bartenderne frem og tilbage. Her kan man passende lave underholdning i bussen i form af Eventansvarlig \& Monopolet, hvor man lader bartenderne få ens telefonnummer og de så sender en række spørgsmål man svarer på. Det er morskab for hele familien og kan klart anbefales.
\\ \\
\textbf{HUSK HUSK HUSK AT SIGE TIL FOLK AT DE SKAL VÆRE DER I GOD TID!!!!!}

\section*{Events}
\label{sec:arskalender}
Her er en række events og hvornår på året de skal foregå cirka. I hver følgende subsection vil der uddybes hvad hvert event handler om samt nogle gode råd til disse. \\ \\
\textbf{Forår}
  \begin{itemize}
  \item Generalforsamling
  \item Aarhus Bryghustur
  \item Sponsorbar (måske)
  \end{itemize}
\textbf{Sommer}
  \begin{itemize}
  \item Djurs Sommerlandstur
  \item Sommerfest
  \item Sponsorbar (måske)
  \end{itemize}
\textbf{Efterår}
  \begin{itemize}
  \item Fødselsdag
  \item Sponsorbar (måske)
  \end{itemize}
\textbf{Vinter}
  \begin{itemize}
  \item Julefrokost
  \item Sponsorbar (måske)
  \end{itemize}

\noindent Sponsorbarer kan ligge når som helst på året. Foruden de nævnte events er du altid velkommen til at finde på flere events, men der bør som minimum være ovenstående.

\subsection*{Aarhus Bryghustur}
\label{sec:aarhus-bryghus}

Et af de traditionelle events er den årlige tur til Aarhus Bryghus!
Her skal du sørge for at kontakte Aarhus Bryghus og aftale en
ølsmagningstur for x antal personer. Det kan i den her case være en
god idé først at få datoen på plads med Aarhus Bryghus, før der sendes
informationer og åbnes for tilmeldingsblanket. Det er meget god stil at udmelde til Aarhus Bryghus hvor stort det endelige antal personer som kommer med er.

Aarhus Bryghus plejer i modstætning til andre foreninger ikke at tage imod betaling for vores tur, og vi får også lov til at smage på en masse øl efterfølgende. Vi er en god kunde for dem, men vi skal stadig være anstændige og behandle dem godt. Husk altid at tilbyde betaling for ølsmagningen og en eventuel faktura skal selvfølgelig betales hvis den kommer.

Mht. mad så plejer jeg at lave en liste over sandwiches fra Anettes sandwich som vi bestiller på forhånd og tager med derud. Man kan overveje at få noget leveret direkte derud. Tidspunktet for ølsmagningen plejer at være fra eftermiddagen til om aftenen. De sidste par år har vi brugt Midttrafik som transport. Det passer fint med at tage letbanen derud, hvis man bestiller en zonebillet (op til 14 mennesker men man kan bare bestille flere billetter). Husk det at ølsmagningen ikke kan ligge om fredagen, og gerne må ligge en hverdag.

Vi plejer at mødes ved DatBar og så gå i fællesskab hen til Letbanen. Husk at tjekker kørselsplanen!!!

\subsection*{Sommerfest}
Sommerfesten er et relativt nyt koncept. Her køber vi selv ind til mad (ofte grill, salat og brød) og sørger selv for at lave det. Vi plejer at tage køleskabene med ud og fadølsanlægget og så vil der bare være fri mad og bar. Vi har også købt en pool som vi plejer at bruge. Husk at snakke med Drift omkring alarmer og sørg selvfølgelig for at booke Fitts Lawn så I har et sted at være. Folk skal melde sig på opsætning, oprydning eller madlavning. Det meste bliver købt ind på dagen. Servicen plejer at være éngangsservice. Der bør være 4-6 på madlavning for at det ikke er alt for træls (dog ikke mange flere). Sidste år tog vi 50 kroner for sommerfesten da det er et ekstraordinært event. Dette fungerede fint og betød at maden var inddækket. Alkohollen var gratis så det var stadig et fint tilbud. Da oprydningen plejer at foregå sent om aftenen, så er det bedst at det bliver gjort løbende af oprydningsholdet. Man kan overveje at tage de dyre/vigtige ting med indenfor om aftenen og så sørge for oprydningen dagen efter ligesom til julefrokost.

\subsection*{Djurs Sommerland}
\label{sec:djurs-sommerland}

Hvert år i sommerferien arrangeres der en årlig tur for bartenderne
til Djurs Sommerland. Til det skal der arrangeres transport, entré, mad, og is. Turen plejer at ligge søndagen inden studiestart (for det er dér de fleste kan).

Man kan enten leje en hel bus eller lave en grupperejse inde på Midttrafikbestilling (Husk at bestille grupperejse \textbf{både til og fra Djurs Sommerland}). Man skal også ringe til Djurs Sommerland for at lave en gruppebestilling til entré. Det kan man spare en fin mængde på. Det er tradition at spise på pizzastedet (det er også det billigste), husk at sørge for at der er reserveret pladser til dette i god tid. Et par timer før maden plejer man også at gå i fællesskab hen for at få en is i western land. Det er siden 2018 tradition at den eventansvarlige skal spise den største Djurs-vaffel med det hele på tid.
\\ \\
\textbf{Leaderboards er som følger:}
\begin{table}[h]
\begin{tabular}{|l|l|l|ll}
\cline{1-3}
\textbf{År} & \textbf{Navn} & \textbf{Tid}  &  &  \\ \cline{1-3}
2018        & Ching Chong   & 10 min 50 sek &  &  \\ \cline{1-3}
2019        & Ching Chong   & 10 min 24 sek &  &  \\ \cline{1-3}
2020        & ?             & ?             &  &  \\ \cline{1-3}
\end{tabular}
\end{table}

Det er gratis for bartenderne pånær isen, som de selv skal betale. Tager man Midttrafik plejer man at mødes nede ved rutebilstationen og tage den første bus derhen og den sidste bus hjem. Husk at lave gruppebestillinger!!!
\\ \\
Efterfølgende er det tradition at dem der har lyst sætter sig ned i DatBar og ser Thomas og Tim (playlisten findes på Youtube) på det store lærred. Hertil vil der være gratis fadøl og snacks på barens regning.

\subsection*{Julefrokost}
\label{sec:julefrokost}

Julefrokosten er for alle bartenderne i Fredagscaféen. Ligesom alle de andre events skal information sendes ud i god tid. Sørg for at der er nok tid til at maden kan bestilles. Vi plejer at afholde julefrokosten i slutningen af Januar dog før at Kokken \& Jomfruens julefrokost menu udgår. Tal med bestyrelsen om hvilke retter der skal med. Kig tilbage i gamle mails efter lister hvis du bliver i tvivl.

Der er to poster: Opsætning og Oprydning. Det er bedst hvis der er flest på oprydning men der er meget at lave på begge poster. Opsætningsholder plejer at møde klokken 14:00 på dagen, hvor selve julefrokosten starter kl 16:00. Oprydningsholdet møder klokken 11:00 dagen efter til morgenmad (oprydning starter klokken 12:00).

De sidste år er arrangementet blevet afholdt i Kaffestuen i Hopper. Husk at booke lokalet i god tid nede hos Tina og sørge for at alarmen er slået fra vinduerne (snak med drift) så man kan åbne disse. Menuen plejer at være julemad, og der skal i hvert fald være risalamande. Vi plejer at sætte bordene i en hestesko og lægger engangsduge hen over bordet og pynter op med engangsservice. Husk også at indkøbe shotsglas og sørg for at der er snaps nok (cirka én flaske pr mand). Vi plejer at tage ét køleskab op samt fadølsanlæg og cider. Vi tager også højtalere, beerpongbolde, osv. med op så der er til god stemning.

Man kan evt. overveje at lave goodiebags som folk kan tage med hjem hvis man har tid/lyst.

\textbf{Der skal selvfølgelig også være pakkeleg}. Få folk til at tage 2 pakker med hver til højst 50 kroner i alt. Pakkelegen bør ligge efter forretten men før hovedretten så folk ikke er alt for fulde. Husk at have dette med i informationsmailen så folk husker deres pakker. Der er altid nogle som glemmer dette så de kan passende nå i Føtex og købe et par pakker inden pakkelegen.

Til oprydningen så er det vigtigt at alt sættes tilbage som det var før man kom derop. Servicen skal skylles af og gøres klar til afhenting. Man behøver ikke at vaske tingene grundigt men der skal heller ikke sidde så mange madrester at der kommer til at gro meget mug på dem. Alle bordflader skal tørres af og hele gulvet skal vaskes. Toiletterne skal også vaskes, gøres rene og have tømt skraldespande. Husk også at lufte ud så (evt. hold vinduerne åbne natten over). \\ \\

Efter oprydningen er det tradition at bartenderne laver noget sammen så de kan snakke om alle de sjove ting der skete aftenen forinden. Vi har haft stor succes med en fællestur ud til skøjtehallen i Aarhus N, hvor baren betaler for billet og leje af skøjter hvis folk ikke selv har med. Vi har også prøvet at tage ned til bowlinghallen ved Aros. Det var meget sjovt, men en del dyrere end at skøjte, så dette kan også overvejes.

\subsection*{Fødselsdag}
Barens fødselsdag ligger omkring Fredag den 13. september hvert år. Husk at købe nogle flag ind fra Føtex og sørg for at der er et godt fødselsdagstilbud. Sidste gang havde vi f.eks. klosterbryg til 15 kroner, og så kom Niels ude fra Aarhus Bryghus ud med et par ekstra gratis fustager til vores tilbud, hvilket var pænt af ham.


\section*{Ekstern Kommunikation}
\label{sec:ekst-komm}

Som eventansvarlig er det dit ansvar at kommunikere med virksomheder
der gerne vil afholde sponsorbar. Sørg derfor for at holde en liste
med sponsorbar-kunder. Der kan nemt opstå kø blandt virksomheder, og
hvis du ikke holder styr på det, bliver det noget rod.

Listen skal indeholde følgende:
\begin{itemize}
\item Dato for første henvendelse
\item Header på e-mailen (så du kan finde den)
\item Navn på virksomhed og kontakt person
\end{itemize}

Virksomheder vil ofte gerne lave en masse reklame for dem selv. De skal vide at dette absolut ikke er gratis. Vi tager som minimum en sponsorbar for at de kan komme og reklamere sig selv.

\subsection*{Sponsorbarer}
\label{sec:sponsorbarer}

Studerende kan godt lide gratis øl, men gider ikke skulle betale for
dem med opmærksomhed eller tvungen modtagelse af flyers eller alt
muligt andet gøjl. Virksomheder har mange penge og ikke særligt mange
gode måder at bruge dem på, hvis de vil have opmærksomhed fra
studerende.

Målet er at begynde at holde \emph{én-to sponsorbarer pr. semester},
hvilket gerne må foregå oven i de vagter hvor vi har længe åbent.

De umiddelbare rammer for sponsorbarer er derfor som følger:
\begin{itemize}
\item De skal give \textit{mindst} 10 fustager fadøl af 750 kr. stykket -> 7500kr i
  alt \textbf{eller} give gratis øl i \textit{mindst 2 timer}.
\item For den pris køber de op til et medium langt oplæg (~20 min)
  \begin{itemize}
  \item Vi stiller PA-anlæg til rådighed og sørger for opsætning. Hvis de siger at de ikke har brug for mikrofon men stadig vil snakke, så er det vigtigt at de får en mikrofon ellers kan man overhovedet ikke høre hvad de siger.
  \item Vi sørger for PR om nødvendigt
  \end{itemize}
\item Gæstebartendere er en god ide, hvis man gerne vil have kontakt
  til de studerende. De er velkommen til at tage nogle pop-ups med og evt nogle visitkort, quizzer (med præmier) eller andet swag.
\item Når der har snakket færdig, så skal øllen være gratis. Altså ikke noget med ølbilletter eller at folk skal komme op og snakke med dem for at få en øl, sådan noget pis gider vi ikke.
\item Sørg for at booke det mellem-anlæg så vi kan bruge begge anlæg når det går stærkt med øllen! På den måde undgår man stor kø.
\end{itemize}


En standardmail for sponsorbar:
\begin{quote}
Hej x,

Vi er glade for at I har lyst til at holde et oplæg og give øl til de studerende samtidigt. Rammerne for sponsorbar er som følger:

I køber øl til de studerende. Det plejer som regel at være minimum 10 fustager af 750,- kr eller et tidsrum på 2+ timer (hvor I selvfølgelig kun betaler for de fustager der bliver brugt). 10 fustager svarer til omkring 500 halvliters fadøl. Til den pris får I lov til at afholde et (højst) 20 minutters oplæg inden den gratis øl bliver uddelt. Vi sørger for et PA-anlæg, projektor og PR om nødvendigt. Oplægget plejer at starte kl. 16:00, men det er op til jer hvornår I vil begynde.

I er velkommen til at tage jeres pop-up ting med og stille op ved et bord så folk kan komme hen og tale med jer efterfølgende. Her kan I evt. uddele flyers eller have en slags quiz/konkurrence hvor folk svarer på nogle spørgsmål på papir og kan vinde en slags præmie ved at deltage. I er også velkomne til at være gæstebartendere hvis I ønsker dette.


[Noget med dato om hvornår de kan og om der på nuværende tidspunkt er kø / ledigt]


Har I yderligere spørgsmål må I endelig skrive eller ringe til mig.

Mvh [Eventansvarlig]
\end{quote}

\noindent Netcompany plejer at afholde sponsorbar to gange om året, bare lige så I er opmærksomme på det. De plejer at give et langt tidsrum hvilket er godt for økonomien

\section*{Dødsquake}
\label{sec:dodsquake}
De studerende på IT og Datalogi kan godt lide events hvor det er
muligt at game med og mod hinanden. Der holdes derfor Dødsquake ca. én
gang hvert semester. Vær opmærksom på, at det ikke ligger for tæt på
en sponsorbar eller samtidigt med et andet event

\section*{Tour de Fredagsbar}
\label{sec:tour-de-fredagsbar}
Erfaringer har vist sig at det er en god ide at holde sig indenfor
normale studie perioder (læs: lad være med at arrangere disse i
eksamens- eller ferie-perioder og vær opmærksom på at andre fakulteter
har forskellige kalendere i forhold til eksamener osv.). Det har været lidt problematisk at afholde en tour de fredagsbar da vi har åbent hver fredag, hvilket betyder at ikke alle bartenderne kan komme med.

\section*{Slutord}
\label{sec:epilog}
Jeg håber, at du fandt denne guide hjælpsom. Er du i tvivl eller har
spørgsmål omkring hvordan et event skal holdes, eller hvordan du får
de praktiske ting på plads, kan du altid spørge de andre
bestyrelsesmedlemmer, tage det op til næste bestyrelsesmøde, eller
sende en mail til tidligere eventansvarlig i baren på mailen
\email{christian@czn.dk}. \\ \\
Jeg håber, du får holdt nogle fede events med god underholdning og morskab!
\end{document}

%%% Local Variables:
%%% mode: latex
%%% TeX-master: t
%%% End:
