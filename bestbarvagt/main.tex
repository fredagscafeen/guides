% !TeX spellcheck = da_DK
\documentclass{article}
% Ability to write input files using utf8
\usepackage[utf8]{inputenc}

% Proper font, æ, ø and å becomes copy/paste and searchable
\usepackage[T1]{fontenc}
\usepackage{lmodern}

% Enable \includegraphics, so that images can be included
\usepackage{graphicx}
\DeclareGraphicsExtensions{.pdf, .png, .jpg, .PDF, .PNG, .JPG}

% Enables use of links, and adds ToC for your PDF-reader
\usepackage{hyperref}
% A small macro for inserting clickable email adresses, e.g.,
% \email{best@fredagscafeen.dk}
\newcommand*{\email}[1]{\href{mailto:#1}{\nolinkurl{#1}}}

% Adding visible TODO's using \todo
\usepackage[obeyFinal]{todonotes}

% Better kerning etc.
\usepackage{microtype}

% To create and control lists (itemize, enumeration, description)
\usepackage{enumitem}

% A new list environment that has both numbers (as enumeration) and
% labels (as description)
\newcounter{enumdescriptioncount}
\newlist{enumdescription}{description}{1}        % 1 means that it cannot be nested
\setlist[enumdescription]{%
  before = {\setcounter{enumdescriptioncount}{0}%
            \renewcommand*{\theenumdescriptioncount}{\arabic{enumdescriptioncount}}},
  font = {\bfseries\stepcounter{enumdescriptioncount}{\large \theenumdescriptioncount.}~},
  align = right,                % Makes the labels end, at the same position
  itemindent = 6em              % TODO: This can be done better
}

% Hyphenation for Danish words etc.
\usepackage[danish]{babel}

% Macros for common words, so that we can format them fancily. xspace
% inserts spaces when needed based on context.
\usepackage{xspace}

\newcommand*{\fotex}{F%
\kern -.25em%
{\raisebox{-.215em}{Ø}}%  % -.215em makes 2 Es match
\kern -.25em%
\TeX\xspace}


\title{Barvagt for bestyrelsesmedlem\\ i \fredagscafeen}
\date{\today}
\author{Anders Bruun Severinsen}

\begin{document}

\maketitle

\tableofcontents \

Denne guide indeholder de ekstra opgaver, 
der er for den baransvarlige på barvagt.

\section{Før Barvagten}
\label{sec:pre-barvagten}

Det er det ansvarlige bestyrelsesmedlems ansvar at være klar Bag Gitteret\index{Bag Gitteret} senest kl 14:30.
Derudover kan der være nogle ekstra opgaver, så sørg for at være i god tid.

\subsection{Ada 0}
\label{sec:pre:ada}

\begin{itemize}
    \item Der skal hentes pengekasse og Krydsliste\index{Krydsliste} i Rummet ved siden af 
    Rummet\index{Rummet ved siden af Rummet}. Pengekassen\index{Pengekassen} må aldrig stå uden opsyn!
\end{itemize}

\subsection{Bag ved Gitteret}
\label{sec:pre:bag-ved-gitteret}
\begin{itemize}
    \item Mød med de andre bartendere bag gitteret, og fordel opgaverne beskrevet i den standard barvagt guide.
\end{itemize}

\subsection{Barområdet}
\label{sec:pre:baromradet}

\begin{itemize}
    \item Hvis der er rykket på borde og stole, skal de stilles på plads.
    \item Hængelåsene omkring køleskabene i Barområdet\index{Barområdet} låses op. 
    Koden kan findes i secrets på hjemmesidens \href{https://fredagscafeen.dk/admin}{admin-side}.
\end{itemize}

\section{Under Barvagten}
\label{sec:intra-barvagten}
Under barvagten skal den ansvarlige kunne springe til, 
derfor anbefaldes det, at det ikke er den baransvarlige 
der løber erner eller køber aftensmad, men i stedet uddelegerer
opgaverne til de andre bartendere.
\begin{itemize}
    \item På et tidspunkt under barvagten kommer G4S vagten forbi, 
    for at se et studiekort, og spørge hvor længe vi har åbent.
\end{itemize}

\section{Efter Barvagten}
\label{sec:post-barvagten}

\begin{itemize}
    \item Fadølsanlægget skal renses, og dette gøres fordelagtigt af den baransvalige.
    \item Når man tror alle opgaver er gjort, gennemgås den laminerede check-liste i baren.
    \item Når alle andre opgaver er gjort, går den baransvarlige op med pengekassen. 
    %Det kan være en god ide at få pengekassen over tidligt, så den ikke efterlades uden opsyn.
    Kontanter, kviteringer og Krydslisten lægges i deponeringsboksen i Rummet ved
    siden af Rummet. Kontanter og kviteringer lægges i den vedlagte frysepose.
\end{itemize}

\printindex
\end{document}

%%% Local Variables:
%%% mode: latex
%%% TeX-master: t
%%% End:
