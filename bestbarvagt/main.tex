% !TeX spellcheck = da_DK
\documentclass[danish]{article}
% Ability to write input files using utf8
\usepackage[utf8]{inputenc}

% Proper font, æ, ø and å becomes copy/paste and searchable
\usepackage[T1]{fontenc}
\usepackage{lmodern}

% Enable \includegraphics, so that images can be included
\usepackage{graphicx}
\DeclareGraphicsExtensions{.pdf, .png, .jpg, .PDF, .PNG, .JPG}

% Enables use of links, and adds ToC for your PDF-reader
\usepackage{hyperref}
% A small macro for inserting clickable email adresses, e.g.,
% \email{best@fredagscafeen.dk}
\newcommand*{\email}[1]{\href{mailto:#1}{\nolinkurl{#1}}}

% Adding visible TODO's using \todo
\usepackage[obeyFinal]{todonotes}

% Better kerning etc.
\usepackage{microtype}

% Hyphenation for Danish words etc.
\usepackage{babel}

% For conditionals in commands
\usepackage{xifthen}

% Macro for Question and answers, troubleshooting etc.
% The optional argument is a label to ref to. The label will be prefixed with "qna:"
\newcommand{\QnA}[3][]{
\paragraph{#2}
% Insert label if it is provided in the optional argument
\ifthenelse{\isempty{#1}}
  {}
  {\label{qna:#1}}
  \begin{itemize}
    #3
  \end{itemize}
}

% Macros for common words, so that we can format them fancily. xspace
% inserts spaces when needed based on context.
\usepackage{xspace}

\newcommand*{\fotex}{F%
\kern -.25em%
{\raisebox{-.215em}{Ø}}%  % -.215em makes 2 Es match
\kern -.25em%
\TeX\xspace}

% Better references, automatically uses Danish names. Use \cref
% instead of \autoref. See package documentation for more details.
%
% nameinlink makes the prefix (e.g. "kapitel") a part of the hyperlink
% (as \autoref does). This might not be a good idea if multiple labels
% are used in a \cref.
\usepackage[nameinlink]{cleveref}


\title{Barvagt for bestyrelsesmedlem\\ i \fredagscafeen}
\date{\formatdate{25}{10}{2025}}
\author{Anders Bruun Severinsen}

\begin{document}

\maketitle

\tableofcontents \

Denne guide indeholder de ekstra opgaver, der er for den baransvarlige på barvagt.

\section{Før Barvagten}
\label{sec:pre-barvagten}

Det er vigtigt at man et par timer inden baren åbner, sætter strøm til køleskabene i
\textit{Barområdet}\index{Barområdet}, så vi kan sælge kolde vare, så snart baren åbner.\\

Det er det ansvarlige bestyrelsesmedlems ansvar, at være klar Bag Gitteret senest kl. 14:30,
og låse op for de andre bartendere.

Derudover, kan der være en række opgaver, der med fordel kan gøres inden barvagten og de andre bartendere møder op.
Disse opgaver kan være:

\begin{itemize}
    \item Checke at køleskabene \textit{Bag Gitteret}\index{Bag Gitteret} og i \textit{Barområdet} er fyldt op. 
    \item Checke at vi har nok \textit{chips}\index{Chips} i \textit{Rummet under Trappen}\index{Rummet under Trappen}. 
    Hvis vi er løbet tør eller har for få, kan man tage en tur i \foetex\ med 
    \textit{købekortet}\index{Købekortet} og købe nogle flere.
\end{itemize}

\subsection{Ada 0}
\label{sec:pre:ada}

\begin{itemize}
    \item Der skal hentes \textit{pengekasse}\index{Pengekassen} i \textit{Rummet ved siden af Rummet}\index{Rummet ved siden af Rummet} 
    (Ada-029). Man bestemer selv om man henter pengekassen før eller efter baren er sat op, 
    den må bare aldrig stå uden opsyn!
\end{itemize}

\subsection{Træburet}
\label{sec:pre:træburet}

\begin{itemize}
    \item Nøglen til pengekassen og \textit{krydslisten}\index{Krydslisten} hentes.
\end{itemize}

\subsection{Bag ved Gitteret}
\label{sec:pre:bag-ved-gitteret}
\begin{itemize}
    \item For at vagten kan starte godt ud, er det altid godt at mødes ved gitteret.
    Her kan man lige få hilst på de andre bartendere, fordele opgaverne beskrevet i den standard
    barvagt guide og eventuelt give dem en bartender \textit{T-shirt}\index{T-shirt}, hvis det nu er deres første vagt
    eller de har glemt deres egen.
    (Findes i rummet \textit{Under Trappen}\index{Under Trappen}).
    \item Baren er vores hjertebarn og hjem til de gyldne dråber.
    For at vi kan få serveret noget lækkert øl, skal vi have gjort følgende.
    \begin{itemize}
        \item Åbne for CO$_2$ gassen. Vores CO$_2$ patron skulle helst være lukket fredagen
        inden da vi ellers spilder gassen uden grund.
        \item Skift adapteren på rød slange til vores SS kobling, den skulle gerne ligge i baren.
        Når du skifter adapter er det rart at klikke slangerne af, så du ikke behøver at ligge på knæ.
        Når adapteren skiftes, er det vigtigt at få plastikdimsen med fra den anden koblings CO$_2$ rør.
        Efter du har fået skiftet over, køres cideren igennem indtil der kommer en klar stråle.
        I det tilfælde at der kommer vand ud i starten, SKAL der køres \textit{lilla}\index{Lilla} igennem.
        Lilla blandes 1 til 10, og køres igennem slangen. Når der kun kommer lilla farve ud, ventes i
        mindst 15 minutter, hvorefter der køres vand igennem. Hvis der kommer grøn farve ud, skal den
        have en gang mere, indtil at det kun er lilla der kommer ud.
        
        \textit{Se denne \href{https://media.fredagscafeen.dk/guides/rensningafanlaeg.pdf}{guide} til rensning af fadølsanlægget.}
        \item Er der ÅBEN på fad (Plastik fustager) så skal vi have KeyKeg kobling på blå slange.
        Dette gøres på samme måde som i punkt 2. eneste forskel er at plastikdimsen ikke skal med.
        (Der er heller ikke plads til den).
        Efter den nye kobling er kommet på, køres specialøllen igennem.
        Hvis der nu også er vand i den her slang, skal der køres lilla igennem her.
        \item Kør pilsner igennem sort slange.
        \item Nu hvor vi har fået kørt lækkert øl og cider igennem vores system kan baren køres ud.
        Husk at tømme og skylle \textit{drypspanden}\index{Drypspanden} du har brugt og tag en med ud.
    \end{itemize}
\end{itemize}

\subsection{Barområdet}
\label{sec:pre:baromradet}

\begin{itemize}
    \item Hvis der er rykket på borde og stole, skal de stilles på plads.
\end{itemize}

\subsection{A-skilte}
\label{sec:pre:a-skilte}

\fredagscafeen\ har fået nogle seje \textit{A-skilte}\index{A-skilte},
som med fordel kan sættes udenfor,
for at gøre opmærksom på at baren er åben.
De står udenfor træburet i Hopper -1, og
køres med elevatoren op til stueetagen.

I tilfælde af (eller hvis der er chance for) regn, sne eller orkan
frarådes det dog.
I tilfælde af at baren er rykket til en anden lokation end Nygaard kælderen,
kan man sætte andre passende plakater i, for at lede folk det rigtige sted hen.

\section{Under Barvagten}
\label{sec:intra-barvagten}

Under barvagten skal den ansvarlige kunne springe til,
derfor anbefaldes det, at det ikke er den baransvarlige
der løber erner eller køber aftensmad, men i stedet uddelegerer
opgaverne til de andre bartendere.

\begin{itemize}
    \item På et tidspunkt under barvagten kommer vagten, fra \textit{Securitas}\index{Securitas}, forbi,
    for at se et studiekort, og spørge hvornår vi regner bed at være færdige med at rydde op.
\end{itemize}

\section{Efter Barvagten}
\label{sec:post-barvagten}

\begin{itemize}
    \item Fadølsanlægget skal renses, og dette gøres fordelagtigt af den baransvalige.
    \item For at kunne dokumentere overfor rengøringen, at vi har gjort toiletterne rene,
    skal der tages billeder heraf. De kan så ligges op på Discord'en, i en tråd under
    ``post-bar-writeup'' med dato i titlen.
    \item Når man tror alle opgaver er gjort, 
    gennemgås fælles den laminerede \textit{check-liste}\index{Check-liste} i baren.
    \item Herefter holdes en lille \textit{tale}\index{Tale}, hvor man reviderer dagens vagt. 
    Har der været travlt? Hvad har fungeret godt, og hvad har fungeret mindre godt.
    Her kan man også med fordel takke de andre barendere, for deres arbejde.
    \item Når alle andre opgaver er gjort, går den baransvarlige op med pengekassen.
    Det kan også være en god ide at få pengekassen over tidligt, så den ikke efterlades uden opsyn.
    Krydslisten og nøglen til pengekassen ligges i \textit{deponeringsboksen}\index{Deponeringsboksen} 
    i Rummet ved siden af Rummet, og pengekassen stilles aflåst ovenpå. 
\end{itemize}

\newpage
\section{Troubleshooting}
\label{sec:troubleshooting}

\QnA[akut-situation]{Akut situation}{
    \item[1.] Stands ulykken, hvis muligt, uden fare for dig selv
    \item[2.] Ring 112 og fortæl
    \begin{itemize}
        \item Dit navn
        \item Hvor du ringer fra
        \item Hvad der er sket
        %\item Hvilket telefonnummer du ringer fra
    \end{itemize}
    \item[3.] Evakuér hvis nødvendigt. 
    \item[•] I tilfælde af brand og ulykker, aktiver varslingsanlæg (brandalarm).
    \item[•] Du har som baransvarlig ansvaret for, at kende
    flugtvejsplanerne for den pågældende bygning, baren befinder sig i.
    \item[•] I hvert evakueringsområde vil der være opsat evakueringsmateriale. Holderen
    indeholder en gul og orange vest, som skal bæres af henholdsvis evakueringslederen og
    samlepladslederen. Derudover følger specifikke instrukser til begge roller.
    I forbindelse med at påtage sig rollen som evakuerings- eller samlepladsleder, skal 
    egen sikkerhed altid overvejes.
    \item[4.] Ring til AU's alarmeringsnummer: 87 15 16 17
    \item[$\ast$] Husk på at du som frivillig, ikke skal kunne klare alle situationer på egen hånd.
    Hvis du står med en svær situation, hvor der er behov for, at du griber ind, er
    det altid godt at inddrage en person mere fra foreningen, så I er to om det.
    \item[$\ast\ast$] I nogle festforeninger og fredagsbarer har man sine egne dørmænd til festen,
    som kan inddrages i håndteringen af svære situationer.
    Da vi ikke har dørmænd (fordi vi ikke har haft brug for det),
    kan det være nødvendigt enten at tilkalde politiet eller AU's vagtselskab, Securitas.
    Securitas-vagterne er tilknyttet campus efter kl. 17 på hverdage og i hele weekenden, og de kan bl.a.
    assistere os, hvis vi har ubudne gæster, der skal bortvises, hvis der sprænger et vandrør,
    eller der sker andre svære situationer, som I har brug for hjælp til at håndtere.
    Securitas-vagterne kan tilkaldes på telefon: 70 26 36 50.
    \item[$\ast\ast\ast$] \textbf{Du kan læse mere om AU's gældende brandreglement og evakueringsprocedurer, 
    \href{https://medarbejdere.au.dk/administration/bygninger/beredskab/akuttesituationer}{her}.}
}
\QnA[codeofconduct]{Adfærdskodeks}{
    \item[1.] Hvis personer opfører sig i strid med vores adfærdskodeks, som kan læses på \url{https://fredagscafeen.dk/about/\#codeofconduct/},
    kan man tage dem til side og tage en seriøs snak med dem om det.
    Vi forbeholder os retten til at bortvise dem fra baren, hvis de ikke vil rette ind.
}

\newpage
\section{Meta}
\label{sec:meta}

For at støtte vores værdifulde arbejde, har Aarhus Universitet
udviklet et frivilligt e-læringskursus på Brightspace,
for fredagsbarer og festforeninger, som kan være en god ide at gennemgå.
Kurset består af de fire moduler:
\begin{description}
    \item[AU's rammer] Rammer og regler og hvad gør man, hvis der opstår en krise.
    \item[At blive en del af en forening] Inklusion og fællesskaber, den gode
    onboarding i en forening, opmærksomhed på ulige relationer, egne og andres grænser og etik.
    \item[Din betydning for studiemiljøet] Få indblik i, hvordan du og din forening påvirker
    studiekulturen og hvordan I kan være med til at skabe et inkluderende og
    trygt studiemiljø for de andre studerende.
    \item [Uoverensstemmelser og konflikthåndtering] Du får indblik over hvordan konflikter
    kan udvikle sig, og hvordan du kan håndtere dem.
\end{description}
Kurset er tilgængeligt i Brightspace og kan findes ved at søge
på ``Fredagsbarer og festforeninger'' under ``Alle kurser''.

\printindex
\end{document}

%%% Local Variables:
%%% mode: latex
%%% TeX-master: t
%%% End:
