% !TeX spellcheck = da_DK
\documentclass[danish]{article}
% Ability to write input files using utf8
\usepackage[utf8]{inputenc}

% Proper font, æ, ø and å becomes copy/paste and searchable
\usepackage[T1]{fontenc}
\usepackage{lmodern}

% Enable \includegraphics, so that images can be included
\usepackage{graphicx}
\DeclareGraphicsExtensions{.pdf, .png, .jpg, .PDF, .PNG, .JPG}

% Enables use of links, and adds ToC for your PDF-reader
\usepackage{hyperref}
% A small macro for inserting clickable email adresses, e.g.,
% \email{best@fredagscafeen.dk}
\newcommand*{\email}[1]{\href{mailto:#1}{\nolinkurl{#1}}}

% Adding visible TODO's using \todo
\usepackage[obeyFinal]{todonotes}

% Better kerning etc.
\usepackage{microtype}

% Hyphenation for Danish words etc.
\usepackage{babel}

% For conditionals in commands
\usepackage{xifthen}

% Macro for Question and answers, troubleshooting etc.
% The optional argument is a label to ref to. The label will be prefixed with "qna:"
\newcommand{\QnA}[3][]{
\paragraph{#2}
% Insert label if it is provided in the optional argument
\ifthenelse{\isempty{#1}}
  {}
  {\label{qna:#1}}
  \begin{itemize}
    #3
  \end{itemize}
}

% Macros for common words, so that we can format them fancily. xspace
% inserts spaces when needed based on context.
\usepackage{xspace}

\newcommand*{\fotex}{F%
\kern -.25em%
{\raisebox{-.215em}{Ø}}%  % -.215em makes 2 Es match
\kern -.25em%
\TeX\xspace}

% Better references, automatically uses Danish names. Use \cref
% instead of \autoref. See package documentation for more details.
%
% nameinlink makes the prefix (e.g. "kapitel") a part of the hyperlink
% (as \autoref does). This might not be a good idea if multiple labels
% are used in a \cref.
\usepackage[nameinlink]{cleveref}


\title{Barvagt for bestyrelsesmedlem\\ i \fredagscafeen}
\date{\today}
\author{Anders Bruun Severinsen}

\begin{document}

\maketitle

\tableofcontents \

Denne guide indeholder de ekstra opgaver, 
der er for den baransvarlige på barvagt.

\section{Før Barvagten}
\label{sec:pre-barvagten}

Det er det ansvarlige bestyrelsesmedlems ansvar at være klar \textit{Bag Gitteret}\index{Bag Gitteret} senest kl 14:30, 
og låse op for de andre bartendere.
Derudover kan der være nogle ekstra opgaver, så sørg for at være i god tid. 

\subsection{Ada 0}
\label{sec:pre:ada}

\begin{itemize}
    \item Der skal hentes pengekasse og \textit{Krydsliste}\index{Krydsliste} i \textit{Rummet ved siden af 
    Rummet}\index{Rummet ved siden af Rummet}. Man bestemer selv om man henter 
    \textit{Pengekassen}\index{Pengekassen} før eller efter baren er sat op, den må bare aldrig stå uden opsyn!
\end{itemize}

\subsection{Bag ved Gitteret}
\label{sec:pre:bag-ved-gitteret}
\begin{itemize}
    \item For at vagten kan starte godt ud, er det altid godt at mødes ved gitteret. 
    Her kan man lige få hilst på de andre bartendere, fordele opgaverne beskrevet i den standard 
    barvagt guide og eventuelt give dem en bartender \textit{T-shirt}\index{T-shirt} hvis nu det er deres første vagt 
    (Findes i rummet \textit{Under Trappen}\index{Under Trappen}). 
    \item Baren er vores hjertebarn og hjem til de gyldne dråber. 
    For at vi kan få serveret noget lækkert øl, skal vi have gjort følgende.  
    \begin{itemize}
        \item Åbne for CO$_2$ gassen. Vores CO$_2$ patron skulle helst være lukket fredagen 
        inden da vi ellers spilder gassen uden grund.
        \item Skift adapteren på rød slange til vores SS kobling, den skulle gerne ligge i baren. 
        Når du skifter adapter er det rart at klikke slangerne af, så du ikke behøver at ligge på knæ. 
        Når adapteren skiftes, er det vigtigt at få plastikdimsen med fra den anden koblings CO$_2$ rør. 
        Efter du har fået skiftet over, køres cideren igennem indtil der kommer en klar stråle. 
        I det tilfælde at der kommer vand ud i starten, SKAL der køres lilla igennem. 
        Se \href{https://media.fredagscafeen.dk/guides/rensningafanlaeg.pdf}{guiden} til rensning af fadølsanlægget. 
        \item Er der ÅBEN på fad (Plastik fustager) så skal vi have Keykeg kobling på blå slange. 
        Dette gøres på samme måde som i punkt 2. eneste forskel er at plastikdimsen ikke skal med. 
        (Der er heller ikke plads til den). 
        Efter den nye kobling er kommet på, køres specialøllen igennem. 
        Hvis der nu også er vand i den her slang, skal der køres vand igennem her. 
        \item Kør Pilsner igennem hvid slange. 
        \item Nu hvor vi har fået kørt lækkert øl og cider igennem vores system kan baren køres ud. 
        Husk at tømme og skylle \textit{drypspanden}\index{drypspanden} du har brugt og tag en med ud. 
    \end{itemize}
\end{itemize}

\subsection{Barområdet}
\label{sec:pre:baromradet}

\begin{itemize}
    \item Hvis der er rykket på borde og stole, skal de stilles på plads.
    \item \textit{Hængelåsene}\index{Hængelåsene} omkring køleskabene i \textit{Barområdet}\index{Barområdet} låses op. 
    Koden kan findes i secrets på hjemmesidens \href{https://fredagscafeen.dk/admin/admin/secrets/}{admin-side}.
\end{itemize}

\subsection{A-skilte}
\label{sec:pre:a-skilte}
\fredagscafeen\ har fået nogle seje \textit{A-skilte}\index{A-skilte}, 
som med fordel kan sættes udenfor, 
for at gøre opmærksom på at baren er åben.
De står udenfor træburet i Hopper -1, og 
køres med elevatoren op til stueetagen.

I tilfælde af (eller hvis der er chance for) regn, sne eller orkan
frarådes det dog.
I tilfælde af at baren er rykket til en anden lokation end Nygaard kælderen, 
kan man sætte andre passende plakater i, for at lede folk det rigtige sted hen.

\section{Under Barvagten}
\label{sec:intra-barvagten}
Under barvagten skal den ansvarlige kunne springe til, 
derfor anbefaldes det, at det ikke er den baransvarlige 
der løber erner eller køber aftensmad, men i stedet uddelegerer
opgaverne til de andre bartendere.
\begin{itemize}
    \item På et tidspunkt under barvagten kommer G4S vagten forbi, 
    for at se et studiekort, og spørge hvor længe vi vil være der.
\end{itemize}

\section{Efter Barvagten}
\label{sec:post-barvagten}

\begin{itemize}
    \item Fadølsanlægget skal renses, og dette gøres fordelagtigt af den baransvalige.
    \item Når man tror alle opgaver er gjort, gennemgås den laminerede \textit{check-liste}\index{check-liste} i baren.
    \item Når alle andre opgaver er gjort, går den baransvarlige op med pengekassen. 
    %Det kan være en god ide at få pengekassen over tidligt, så den ikke efterlades uden opsyn.
    Kontanter, kvitteringer og Krydslisten ligges i \textit{deponeringsboksen}\index{deponeringsboksen} i Rummet ved
    siden af Rummet. Kontanter og kvitteringer ligges i den vedlagte frysepose.
\end{itemize}

\printindex
\end{document}

%%% Local Variables:
%%% mode: latex
%%% TeX-master: t
%%% End:
